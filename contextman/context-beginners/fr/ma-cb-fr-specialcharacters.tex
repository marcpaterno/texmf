\startcomponent ma-cb-fr-specialcharacters

\product ma-cb-fr

\chapter[caracteres speciaux]{Caract�res sp�ciaux}

\index{caract�res sp�ciaux}

Vous avez remarqu� que les commandes \CONTEXT\ �taient pr�c�d�es par
un \tex{} (antislash). \tex{} � donc une signification sp�ciale pour
\CONTEXT. � part \tex{}, il existe d'autres caract�res qui n�cessite
une attention particuli�re si vous souhaitez qu'ils apparaissent en
mode verbatim ou en mode texte. Le \in{tableau}[tab:specchars]
pr�sente une vue d'ensemble de ces caract�res ainsi que ce que vous
aurez � taper pour les utiliser dans vos documents.

\let\normalunderscore=\_
\let\normaltilde     =\~

\placetable[here,force][tab:specchars]
  {Caract�res sp�ciaux (1).}
  \starttable[|c|c|c|c|c|]
  \HL
  \NC \bf \LOW{Carart�res sp�ciaux} \NC \use2 \bf Verbatim  \NC \use2 \bf Texte \NC\FR
  \NC                         \NC \bf Saisie \NC \bf Rendu \NC \bf Saisie \NC \bf Rendu \NC\LR
  \HL
  \NC \type{#} \NC \type{\type{#}} \NC \type{#} \VL \type{\#} \NC \# \NC\FR
  \NC \type{$} \NC \type{\type{$}} \NC \type{$} \VL \type{\$} \NC \$ \NC\MR
  \NC \type{&} \NC \type{\type{&}} \NC \type{&} \VL \type{\&} \NC \& \NC\MR
  \NC \type} \NC \type \NC \% \NC\LR
  \HL
  \stoptable

D'autres caract�res sp�ciaux poss�dent une signification sp�ciale lors
de la composition d'expressions math�matiques et certains d'entre eux
ne peuvent �tre utilis�s que dans le mode math
(cf. \in{chapitre}[formulas]). 

\let\normalbar=|
\placetable
  [here,force]
  [tab:special chars]
  {Caract�res sp�ciaux (2).}
  \starttable[|c|c|c|c|c|]
  \HL
  \NC \bf \LOW{Caract�res sp�ciaux} \NC \use2 \bf Verbatim  \NC \use2 \bf Texte \NC\FR
  \NC                         \NC \bf Saisie \NC \bf Rendu \NC \bf Saisie \NC \bf Rendu \NC\LR
  \HL
  \NC \type{+} \NC \type{\type{+}} \NC \type{+} \VL \type{$+$} \NC $+$ \NC\FR
  \NC \type{-} \NC \type{\type{-}} \NC \type{-} \VL \type{$-$} \NC $-$ \NC\MR
  \NC \type{=} \NC \type{\type{=}} \NC \type{=} \VL \type{$=$} \NC $=$ \NC\MR
  \NC \type{<} \NC \type{\type{<}} \NC \type{<} \VL \type{$<$} \NC $<$ \NC\MR
  \NC \type{>} \NC \type{\type{>}} \NC \type{>} \VL \type{$>$} \NC $>$ \NC\LR
  \HL
  \stoptable

\stopcomponent
