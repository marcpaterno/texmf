\startcomponent ma-cb-fr-descriptions

\product ma-cb-fr

%\chapter{Definitions}
\chapter{D�finitions}

\index{d�finitions}

\Command{\tex{definedescription}}
\Command{\tex{setupdescriptions}}

% If you want to display notions, concepts and ideas in a
% consistent manner you can use:

Si vous voulez exposer des notions, des concepts ou des id�es d'un
mani�re coh�rente, vous pouvez utiliser :

\shortsetup{definedescription}

% For example:
Par exemple, le code ci-dessous :

\startbuffer
\definedescription
  [concept]
  [location=serried,headstyle=bold,width=broad]

\concept{Hasselter juffer} A sort of biscuit made of puff pastry and
covered with sugar. It tastes very sweet. \par
\stopbuffer

\typebuffer

%It would look like this:
aurait l'allure suivante :

\getbuffer

% But you can also choose other layouts:
Mais on peut �galement choisir d'autres formes de pr�sentation :

\startbuffer
\definedescription
  [concept]
  [location=top,
   headstyle=bold,
   width=broad,
   style=slanted]

\concept{Hasselter bitter} A very strong alcoholic drink (up to 40\%)
mixed with herbs to give it a special taste. It is sold in a stone
flask and it should be served {\em ijskoud} (as cold as ice). \par

\definedescription
  [concept]
  [location=inmargin,headstyle=bold,width=broad]

\concept{Euifeest} A harvest home to celebrate the end of a period of
hard work. The festivities take place in the last week of August.
\par

\stopbuffer

\start
\getbuffer
\stop

% If you have more than one paragraph in such a definition you can use
% a \type{\start...}||\type{\stop...} pair.

Si une description est constitu�e de plusieurs paragraphes, on doit les
placer dans la paire de commandes \type{\start...}||\type{\stop...}.

\startbuffer
\definedescription
  [concept]
  [location=right,
   headstyle=bold,
   width=broad]

\startconcept{Euifeest} A harvest home to celebrate the end of a
period of hard work.
This event takes place at the end of August and lasts one week. The
city is completely illuminated and the streets are decorated. This
feast week ends with a {\em Braderie}.
\stopconcept
\stopbuffer

\typebuffer

% This would become:

Ceci aurait pour r�sultat :

\getbuffer

% Layout is set up within the second bracket pair of
% \type{\definedescription[][]}. But you can also use:

La pr�sentation est ici d�finie dans la seconde paire de crochets de la
commande \type{\definedescription[][]}. Mais on peut �galement utiliser
la commande :

\shortsetup{setupdescriptions}

\stopcomponent