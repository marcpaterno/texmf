\startcomponent ma-cb-fr-enumerations

\product ma-cb-fr

%\chapter{Numbered definitions}
\chapter{Listes num�rot�es}

\index{listes num�rot�es}

\Command{\tex{defineenumeration}}
\Command{\tex{setupenumerations}}

% With \type{\defineenumeration} you can number text elements
% like remarks or questions. If you want to make numbered
% remarks in your document you use:

Avec la commande \type{\defineenumeration}, il est possible de num�roter
certains �l�ments de texte tels que les remarques ou les questions. Pour
num�roter des remarques dans un document, on utilise :

\shortsetup{defineenumeration}

%For example:

Par exemple :

\startbuffer[a]
\defineenumeration
  [remark]
  [location=top,
   text=Remark,
   inbetween=\blank,
   after=\blank]
\stopbuffer

\typebuffer[a]

% Now the new commands \type{\remark}, \type{\subremark},
% \type{\resetremark} and \type{\nextremark} are available and
% you can type remarks like this:

Maintenant, les nouvelles commandes \type{\remark}, \type{\subremark},
\type{\resetremark} et \type{\nextremark} sont disponibles et l'on peut
saisir des remarques de la mani�re suivante :

\startbuffer[b]
\remark In the early medieval times Hasselt was a place of
pilgrimage. The {\em Heilige Stede} (Holy Place) was torn down during
the Reformation. In 1930, after 300 years the {\em Heilige Stede} was
reopened.

\subremark Nowadays the {\em Heilige Stede} is closed again but once
a year an open air service is held on the same spot. \par
\stopbuffer

\typebuffer[b]

\start
\getbuffer[a]\getbuffer[b]
\stop

% You can reset numbering with \type{\resetremark} or
% \type{\resetsubremark} or increment a number with
% \type{\nextremark} or \type{\nextsubremark}. This is
% normally done automatically per chapter, section or
% whatever.

On utilise \type{\resetremark} ou \type{\resetsubremark} pour remettre �
z�ro la num�rotation. Pour incr�menter la num�rotation, on utilise
\type{\nextremark} ou \type{\nextsubremark}. Tout ceci est a priori
effectu� automatiquement lors de chaque nouveau chapitre, section, etc.

% You can set up the layout of \type{\defineenumeration} with:

On peut configurer la pr�sentation de \type{\defineenumeration} avec :

\shortsetup{setupenumerations}

% You can also vary the layout of {\bf Remark} and {\bf
% Subremark} in the example above by:

Il est �galement possible de modifier la pr�sentation de {\bf Remark} et
de {\bf Subremark} dans l'exemple ci-dessus, avec les commandes
suivantes :

\starttyping
\setupenumeration[remark][headstyle=bold]
\setupenumeration[subremark][headstyle=slanted]
\stoptyping

%If a number becomes obsolete you can type:

Si un nombre devient obsol�te, on peut saisir :

\starttyping
\remark[-]
\stoptyping

% If the remark contains more than one paragraph you will
% have to use the command pair
% \type{\startremark} $\cdots$ \type{\stopremark} that becomes
% available after defining {\bf Remark} with
% \type{\defineenumeration[remark]}.

Si une remarque devait contenir plusieurs paragraphes, on devrait
la saisir au sein de la paire de commandes \type{\startremark}
$\cdots$ \type{\stopremark}. Ces commandes sont disponibles apr�s avoir
d�fini {\bf Remark}, par le biais de \type{\defineenumeration[remark]}.

%So the example above would look like this:

Ainsi, l'exemple ci-dessus ressemblerait � cela :

\startbuffer[c]
\startremark
In the early medieval times Hasselt was a place of pilgrimage. The
{\em Heilige Stede} (Holy Place) was torn down during the
Reformation.

After 300 years in 1930 the {\em Heilige Stede} was reopened.
Nowadays the {\em Heilige Stede} is closed again but once a year an
open air service is held on the same spot.
\stopremark
\stopbuffer

\typebuffer[c]

\start
\getbuffer[a]\getbuffer[c] \par
\stop

\stopcomponent