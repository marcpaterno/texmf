\startcomponent ma-cb-vn-units
\project ma-cb
\product ma-cb-vn
\environment ma-cb-env-vn

%\chapter[units]{Units}
\chapter[units]{Đơn vị đo lường}

%\index{units}
\index{đơn vị}
%\index[siunit]{\cap{SI}--unit}
\index[siunit]{\cap{SI}--unit}

\Command{\tex{unit}}
\Command{\tex{permille}}
\Command{\tex{percent}}

%To force yourself to use dimensions and units consistently
%throughout the document you can make your own list with
%units. These are specified in the set up area of your
%input file.
Để dùng các đơn vị đo lường phù hợp xuyên suốt tài liệu, bạn có thể tự tạo một
danh sách các đơn vị đo lường. Danh sách này được xác định trong vùng thiết
lập của tập tin nhập liệu.

%In \CONTEXT\ there is an external module available that
%contains almost all \SI||units. When this module is loaded
%with \type{\usemodule[units]} you can call units with:
Trong \CONTEXT\ có một mođun bên ngoài chứa đựng hầu hết các đơn vị||SI. Khi
mođun được nạp với lệnh \type{\usemodule[units]} bạn có thể gọi các đơn vị đó
với các lệnh:

\startbuffer
\Meter \Per \Square \Meter
\Cubic \Meter \Per \Sec
\Square \Milli \Meter \Per \Inch
\Centi \Liter \Per \Sec
\Meter \Inverse \Sec
\Newton \Per \Square \Inch
\Newton \Times \Meter \Per \Square \Sec
\stopbuffer

\typebuffer

%It looks like a lot of typing but it does guarantee a
%consistent use of units. The command \type{\unit} also
%prevents the separation of value and unit at line breaks,
%because a number typeset at the end of a line and the unit
%at the beginning of the next one, is far from perfect.
%These examples come out as:
Trông thì thấy phải nhập vào nhiều nhưng nó đảm bảo một sự sử dụng các đơn vị
đó phù hợp. Lệnh \type{\unit} cũng ngăn ngừa phá vỡ giá trị và đơn vị lúc
xuống dòng chẳng hạn như một số được sắp tại cuối dòng và đơn vị lại được sắp
đầu dòng kế tiếp. Đây là cách hiển thị của các lệnh trên:

\startnarrower
\startlines
\getbuffer
\stoplines
\stopnarrower

%You can define your own units with:
Bạn có thể định nghĩa các đơn vị đo lường riêng bạn với lệnh:

\starttyping
\unit[Ounce]{oz}{}
\stoptyping

\unit[Ounce]{oz}{}

%Later on in the document you can type \type{15.6 \Ounce}
%that will be displayed as 15.6 \Ounce.
Sau đó trong tài liệu, bạn có thể nhập vào \type{15.6 \Ounce} sẽ hiển thị 15.6
\Ounce.

%In order to write \percent\  and \permille\ in a consistent
%way there are two specific commands:
Để viết \percent\ và \permille\ bằng cách thích hợp có hai lệnh được xác định:

\type{\percent} \crlf
\type{\permille}

\stopcomponent
