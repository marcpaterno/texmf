\startcomponent ma-cb-vn-document
\project ma-cb
\product ma-cb-vn
\environment ma-cb-env-vn

%\chapter{How to create a document}
\chapter{Phương thức tạo một tài liệu}

%\index{input file}
\index{tập tin nhập liệu}

%Let's assume you want to create a simple document. It has
%some structure and contains a title page, a few chapters,
%sections and sub sections. Of course there is a table of
%contents and an index.
Chúng ta giả sử rằng bạn muốn tạo một tài liệu đơn giản. Nó có một vài cấu trúc
và có một trang tiêu đề, vài chương, mục và các mục con. Dĩ nhiên cũng có một
bảng nội dung và bảng tra cứu.

%\CONTEXT\ can create such a document automatically if you
%offer the right input by means of a file. So first you have
%to create an input file. An input file consists of a name
%and an extension. You can choose any name you want but the
%extension has to be \type{tex}. If you create a file with
%the name \type{myfile.tex} you will find no difficulties in
%running \CONTEXT.
\CONTEXT\ có thể tạo tài liệu tự động nếu bạn chỉ ra phần nhập liệu đúng bằng
cách định nghĩa trong tập tin. Vì vậy, đầu tiên bạn phải tạo một tập tin nhập
liệu. Một tập tin nhập liệu bao gồm tên và phần mở rộng. Bạn có thể đặt tên tập
tin theo ý bạn nhưng phần mở rộng phải là \type{tex}. Nếu bạn tạo tập tin với
tên là \type{taptin.tex} bạn sẽ không thấy khó khăn khi sử dụng \CONTEXT.

%An \pagereference[inputfile] input file could look like
%this:
Một \pagereference[inputfile] tập tin nhập liệu có thể trông như thế này:

%\startbuffer
%\starttext
%
%\startstandardmakeup
%  \midaligned{How to make a document.}
%  \midaligned{by}
%  \midaligned{The Author}
%\stopstandardmakeup
%
%\completecontent
%
%\chapter{Introduction}
%
%... your text\index{indexentry} ...
%
%\chapter{One Chapter}
%
%\section[firstsection]{The first section}
%
%... your text ...
%
%\section{The second section}
%
%\subsection{the first sub section}
%
%... your text\index{another indexentry} ...
%
%\subsection{the second sub section}
%
%... your text ...
%
%\section{The third section}
%
%... your text ...
%
%\chapter{Another Chapter}
%
%... your text ...
%
%\chapter[lastchapter]{The Last Chapter}
%
%... your text ...
%
%\completeindex
%
%\stoptext
%\stopbuffer
\startbuffer
\starttext

\startstandardmakeup
  \midaligned{Tạo một tài liệu.}
  \midaligned{bởi}
  \midaligned{Tác giả}
\stopstandardmakeup

\completecontent

\chapter{Lời nói đầu}

... văn vẻ của bạn\index{chỉ mục} ...

\chapter{Chương Một}

\section[firstsection]{Mục thứ nhất}

... văn vẻ của bạn ...

\section{Mục thứ hai}

\subsection{mục con thứ nhất}

... văn vẻ của bạn\index{chỉ mục khác} ...

\subsection{mục con thứ hai}

... văn vẻ của bạn ...

\section{Mục thứ ba}

... văn vẻ của bạn ...

\chapter{Chương khác}

... văn vẻ của bạn ...

\chapter[lastchapter]{Chương cuối}

... văn vẻ của bạn ...

\completeindex

\stoptext
\stopbuffer

{\switchtobodyfont[9pt]\typebuffer}

%\CONTEXT\ expects a plain \ASCII\ input file. Of course you
%can use any texteditor or wordprocessor you want, but you
%should not forget that \CONTEXT\ can only read \ASCII\
%input. Most texteditors or wordprocessors can export your
%file as plain \ASCII.
\CONTEXT\ chấp nhận thuần ASCII trong tập tin nhập liệu. Dĩ nhiên, bạn có
thể dùng bất cứ trình biên tập hay xử lí văn bản nào bạn muốn nhưng đừng quên
rằng \CONTEXT\ chỉ có thể đọc ASCII trong dữ liệu. Hầu hết các trình biên
tập và xử lí văn bản đều có thể trích xuất ra dữ liệu ASCII.

%The input file should contain the text you want to be
%processed by \CONTEXT\ and the \CONTEXT\ commands. A
%\CONTEXT\ command begins with a backslash~\tex{}. With
%the command \type{\starttext} you indicate the beginning of
%your text. The area before \type{\starttext} is called the
%set up area and is used for defining new commands and setting up
%the layout of your document.
Tập tin nhập liệu chứa dữ liệu văn bản bạn muốn được thực hiện bởi \CONTEXT\
và các lệnh của \CONTEXT. Một lệnh \CONTEXT\ bắt đầu bằng một dấu chéo 
ngược~\type{\}. Lệnh \type{\starttext} cho biết bắt đầu phần văn bản của bạn.
Phần trước \type{\starttext} được gọi là phần thiết lập và được dùng cho việc
định nghĩa các lệnh mới và thiết lập bộ khung nền cho tài liệu.

%A command is usually followed by a left and right bracket
%pair \type{[]} and/or a left and right brace \type{{}}. In
%\type{\chapter[lastchapter]{The Last Chapter}} the command
%\type{\chapter} for example tells \CONTEXT\ to perform a
%few actions concerning design, typography and structure.
%These actions may be:
Một lệnh luôn được theo sau bởi một cặp dấu ngoặc vuông \type{[]} và|/|hoặc một
cặp dấu ngoặc kép \type{{}}. Trong ví dụ
\type{\chapter[lastchapter]{Chương cuối}} lệnh \type{\chapter} báo cho 
\CONTEXT\ trình diễn các hành động thiết kế, trình bày bản in và thể hiện cấu
trúc tài liệu.
Các hành động này có thể là:

%\startitemize[n,packed]
%\item start a new page
%\item increase chapter number by one
%\item place chapter number in front of chapter title
%\item reserve some vertical space
%\item use a big font
%\item put chapter title (and page number) in table of contents
%\stopitemize
\startitemize[n,packed]
\item bắt đầu trang mới
\item tăng số của chương
\item đặt số của chương trước tiêu đề chương
\item thêm vài khoảng trống phía trên
\item dùng font chữ lớn hơn
\item đặt tiêu đề chương (và số trang) trong bản nội dung
\stopitemize

%These actions will be performed on the argument that is
%given between the left and right braces: {\em The Last
%Chapter}.
Những hành động này sẽ được thực hiện bởi đối số được đặt trong cặp ngoặc móc:
{\em Chương cuối}.

%The \type{[lastchapter]} between brackets has not been
%mentioned yet. This is a label with a logical name that can
%be used for refering to that specific chapter. This can be
%done with yet some other \CONTEXT\ commands:
%\type{\in{chapter}[lastchapter]} that typesets the chapter
%number, while \type{\about[lastchapter]} returns the title.
Phần \type{[lastchapter]} chưa được đề cập đến. Đây là một nhãn với một cái tên
hợp lý có thể được dùng để xem xét đến từng chương riêng biệt. Việc này có thể
được thực hiện với một số lệnh \CONTEXT\ khác: \type{\in{chapter}[lastchapter]}
sắp số trang trong khi \type{\about[lastchapter]} mô tả tiêu đề.

%So now the list of actions can be extended with:
Bây giờ, danh sách các hành động có thể được mở rộng ra:

%\startitemize[continue]
%\item let label \type{lastchapter} be chapter number
%      and title (and store these for later use)
%\stopitemize
\startitemize[continue]
\item để \type{lastchapter} là số chương và tiêu đề (và giữ lại dùng sau này)
\stopitemize

%Other actions concerning running heads, number resetting and
%interactivity are disregarded at this moment.
Các hành động tạo phần đầu trang, sắp xếp số và các hành động tương tác khác
chưa được nói đến lúc này.

%If you have \CONTEXT\ process this example file, you would
%obtain a very simple document with a few numbered chapters
%and section headers.
Nếu bạn cho \CONTEXT\ thực hiện tập tin ví dụ này, bạn sẽ nhận được một tài
liệu rất đơn giản với một vài chương được đánh số và các phần đầu đề mục.

%While processing the file \CONTEXT\ takes care of many
%things. One of these things is for example page numbering.
%But in order to make a table of contents \CONTEXT\ needs
%page numbers that are not yet known to \CONTEXT\ at the
%first run. So you have to process this file twice (a two
%pass job). \CONTEXT\ will produce a few auxilliary files to
%store this kind of information. These are to be processed by
%\TEXUTIL. In some instances you have to proces an input file
%thrice (a three pass job). One can use \TEXEXEC\ to run
%\CONTEXT\ from the command line. This \PERL\ script also
%takes care of the multiple passes.  \TEXEXEC\ is part of the
%standard \CONTEXT\ distribution.
Trong khi thực hiện tập tin, \CONTEXT\ làm rất nhiều việc. Ví dụ, một trong
các việc này là đánh số trang. Nhưng khi tạo bảng nội dung, \CONTEXT\ cần số
trang mà \CONTEXT\ chưa biết trong lần chạy đầu tiên. Vì vậy, bạn phải thực
hiện tập tin này hai lần (cả hai đều hoàn tất).\CONTEXT\ sẽ tạo một vài tập tin
bổ trợ để lưu trữ các thông tin này. Trong một vài trường hợp cá biệt, bạn
phải thực hiện tập tin nhập liệu đến lần thứ ba (cả ba đều hoàn tất). Chúng ta
có thể dùng \TEXEXEC\ để chạy \CONTEXT\ từ dòng lệnh. Đoạn mã \RUBY\ kiểm soát
các lần hoàn tất và quá trình thực hiện các tập tin bổ trợ. \TEXEXEC\ là phần
của bản phân phối \CONTEXT\ chính thức.

\stopcomponent
