\startcomponent ma-cb-en-composedcharacters

\product ma-cb-en

\chapter{Composite characters}

\section{Introduction}

In \in{chapter}[special chars] you have already seen that
you have to type more than one token to obtain special
characters like \# \$ \% \& \_ $\{$ and $\}$. Accented
characters or foreign symbols have to be composed to produce
the right typesetted character.

It is not within the scope of this manual to go into
accented characters in math mode. See the {\TEX Book by
Donald E. Knuth} on that subject.

\section{Accents}

\index{accents}

Accented characters have to be composed in \CONTEXT.
\in{Table}[tab:accents] shows you how to do this. The
character {\em u} is just an example here.

% VZ 2006-11-15 (add \r, \k)
\let\MySign=\=
\placetable
  [here,force]
  [tab:accents]
  {Accents.}
\starttable[|c|c|c|c|]
\HL
\NC \bf You type \NC \bf You get \VL \bf You type  \NC \bf You get \NC\SR
\HL
\NC \type{\`{u}} \NC \`{u}       \VL \type{\u{u}}  \NC \u{u}       \NC\FR
\NC \type{\'{u}} \NC \'{u}       \VL \type{\v{u}}  \NC \v{u}       \NC\MR
\NC \type{\^{u}} \NC \^{u}       \VL \type{\H{u}}  \NC \H{u}       \NC\MR
\NC \type{\"{u}} \NC \"{u}       \VL \type{\b{u}}  \NC \b{u}       \NC\MR
\NC \type{\~{u}} \NC \~{u}       \VL \type{\d{u}}  \NC \d{u}       \NC\MR
\NC \type{\={u}} \NC \MySign{u}  \VL \type{\c{u}}  \NC \c{u}       \NC\MR
\NC \type{\.{u}} \NC \.{u}       \VL \type{\k{u}}  \NC \k{u}       \NC\LR
\NC \type{\r{u}} \NC \r{u}       \VL               \NC             \NC\LR
\HL
\stoptable

You don't want \`{i} or \'{j} so for an
accented {\em i} and {\em j} you compose the characters as
follows:

\type{\"{\i}} ~~~ \"{\i} \crlf
\type{\^{\j}} ~~~ \^{\j}

\section{Foreign symbols}

\index{foreign symbols}

The composition of characters that appear in foreign
languages is shown in \in{table}[tab:foreign symbols].

%
% be ware of the spanish questionmark and exclamation
%

\placetable
  [here,force]
  [tab:foreign symbols]
  {Foreign characters.}
\starttable[|c|c|c|c|]
\HL
\NC \bf You type \NC \bf You get \VL \bf You type     \NC \bf You get \NC\SR
\HL
\NC \type{\oe}   \NC \oe         \VL \type{\O}        \NC \O          \NC\FR
\NC \type{\OE}   \NC \OE         \VL \type{\l}        \NC \l          \NC\MR
\NC \type{\ae}   \NC \ae         \VL \type{\L}        \NC \L          \NC\MR
\NC \type{\AE}   \NC \AE         \VL \type{\SS}       \NC \SS         \NC\MR
\NC \type{\aa}   \NC \aa         \VL \type{?}\type{`} \NC ?`          \NC\MR
\NC \type{\AA}   \NC \AA         \VL \type{!}\type{`} \NC !`          \NC\MR
\NC \type{\o}    \NC \o          \VL                  \NC             \NC\LR
\HL
\stoptable

\stopcomponent
