\startcomponent ma-cb-en-headers

\product ma-cb-en

\chapter{Page headers and footers}

\index{page header}
\index{pagefooter}

\Command{\tex{setupfootertexts}}
\Command{\tex{setupheadertexts}}
\Command{\tex{setupheader}}
\Command{\tex{setupfooter}}
\Command{\tex{noheaderandfooterlines}}

In some cases you want to give your document a page header
and footer. The commands to do this are:

\shortsetup{setupfootertexts}
\shortsetup{setupheadertexts}

The first bracket pair is used for the location of the
footer or header (\type{text}, \type{edge} etc). Footer and
header are placed within the second and third bracket pairs.
In a double sided document a fourth and fifth bracket
pair is used for footer and header on the left-hand side
page and the right-hand side page. In most cases you can omit
these last two bracket pairs.

\startbuffer
\setupfootertexts[Manual][section]
\stopbuffer

\typebuffer

In this case the text {\em Manual} will appear in the
left-hand side corner and the title of the actual section on
the right-hand side of the page. This footer will change with
the beginning of a new section.

You can set up the head- and footline with:

\shortsetup{setupheader}
\shortsetup{setupfooter}

If you want to leave out the page header and footer you can
type:

\starttyping
\noheaderandfooterlines
\stoptyping

\stopcomponent
