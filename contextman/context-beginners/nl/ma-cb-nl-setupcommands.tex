\startonderdeel ma-cb-nl-setupcommands

\produkt ma-cb-nl

\hoofdstuk{Instellingen}

\index{instellingen}
\index{layout}

Instellingen van commando's worden in het instelgebied van de
invoerfile geplaatst. De commando's hebben een globaal
karakter en zijn van toepassing op het volledige document.

In \in{bijlage}[comdefs] is een compleet overzicht gegeven
van alle commando's en de mogelijke parameters en
instellingen.

De \type{stel...in}||commando's hebben allen dezelfde
structuur en zien er bijvoorbeeld als volgt uit:

\startbuffer
\setup{stelalineasin}
\stopbuffer

\haalbuffer


Een instellingscommando bestaat uit een min of meer logische
naam en een aantal vierkante haken. De vierkante haken
kunnen optioneel zijn. In dat geval zijn de \type{[ ]} in de
commandodefinitie schuin gedrukt {\tt \sl [ ]}.

\starttypen
\steleencommandoin[.1.][.2.][..,..=..,..]
\stoptypen

De komma's geven aan dat er een lijst van parameters kan
worden ingegeven. De lijst met opties die bij de definitie
is opgenomen, begint met \type{.1.} en \type{.2.}. Deze geven
de mogelijke opties aan die in het eerste en tweede paar
haken kunnen worden opgenomen. Vervolgens worden parameters
en hun mogelijke waarden in het derde paar haken geplaatst.

De standaardopties en -waarden zijn in de definitie
onderstreept. Bovendien zijn enkele waarden schuin gedrukt:
{\sl sectie}, {\sl naam}, {\sl maat}, {\sl
getal}, {\sl commando} en {\sl tekst}. Deze waarden kunt u
zelf invoeren.

\starttabulatie[|S||]
\NC sectie   \NC  verwacht een sectienaam, zoals hoofdstuk, paragraaf enz. \NC\NR
\NC naam     \NC verwacht een logische naam \NC\NR
\NC maat     \NC verwacht een getal met eenheid in \type{cm pt em ex sp in} \NC\NR
\NC getal    \NC verwacht een getal \NC\NR
\NC commando \NC verwacht een commando, omgeven door \type{{}} \NC\NR
\NC tekst    \NC verwacht tekst \NC\NR
\stoptabulatie

\stoponderdeel
