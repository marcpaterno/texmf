\startonderdeel ma-cb-nl-synonyms

\produkt ma-cb-nl

\hoofdstuk[synoniemen]{Synoniemen}

\index{synoniemen}

\Command{\tex{definieersynoniemen}}
\Command{\tex{stelsynoniemenin}}
\Command{\tex{afkorting}}
\Command{\tex{voluit}}
\Command{\tex{eenheid}}
\Command{\tex{laadafkortingen}}
\Command{\tex{plaatslijstmetafkortingen}}
\Command{\tex{volledigelijstmetafkortingen}}

In een document dienen sommige woorden consistent te worden
weergegeven. Denk daarbij bijvoorbeeld aan afkortingen. Om
die consistentie af te dwingen is het onderstaande commando
beschikbaar:

\shortsetup{definieersynoniemen}

Het eerste paar haken bevat de naam van het synoniemcommando
in enkelvoud, het tweede paar in meervoud. Het derde paar
haken bevat een commando.

Het commando \type{\afkortingen} wordt gedefinieerd door:

\starttypen
\definieersynoniemen[afkorting][afkortingen][\voluit]
\stelsynoniemenin[letter=kap]
\stoptypen

Vervolgens kan het commando \type{\afkorting} worden gebruikt
om afkortingen te defini\"eren:

\starttypen
\afkorting{ANWB}{Algemene Nederlandse Wielrijders Bond}
\afkorting{VVV}{Vereniging voor Vreemdelingen Verkeer}
\afkorting{NS}{Nederlandse Spoorwegen}
\stoptypen

\afkorting{VVV}{Vereniging voor Vreemdelingen Verkeer}

Als u typt:

\startbuffer
De \VVV\ (\voluit{VVV}) kan u voorzien van alle toeristische
informatie over Hasselt.
\stopbuffer

\typebuffer

Dan krijgt u dit:

\haalbuffer

De lijsten met synoniemen of afkortingen kunnen het best
worden gedefinieerd in het instelgebied van de invoerfile. U
kunt dergelijke informatie ook opslaan in een externe file
(\type{afkort.tex}) die u vervolgens apart laadt:

\type{\input afkort.tex} \quad \quad (zie \in{paragraaf}[externe texfiles])

Indien u een lijst met afkortingen in uw document wilt
opnemen, typt u:

\starttypen
\plaatslijstmetafkortingen
\stoptypen

of

\starttypen
\volledigelijstmetafkortingen
\stoptypen

Een complete en gesorteerde lijst met afkortingen wordt
vervolgens gegenereerd.

De weergave van lijsten kan worden be\"{\i}nvloed met:

\starttypen
\stelsynoniemenin
\stoptypen

In het \in{hoofdstuk}[eenheden] vindt u een andere
toepassing van synoniemen.

\stoponderdeel
