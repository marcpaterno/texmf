\startonderdeel ma-cb-nl-pagebackgrounds

\produkt ma-cb-nl

\hoofdstuk{Achtergronden op paginavlakken}

\index{paginavlakken}
\index{achtergrond+paginavlakken}

\Command{\tex{stelachtergrondenin}}

De achtergrond van ieder paginavlak kan worden ingesteld.
Het commando daarvoor is:

\shortsetup{stelachtergrondenin}

De eerste twee paren haken worden gebruikt om de
paginavlakken te defini\"eren. Het laatste paar wordt gebruikt
om de instellingen vast te leggen.

\plaatsfiguur
  [hier]
  [fig:paginavlakken]
  {De paginavlakken ingesteld met \type{\stelachtergrondenin}.}
{\hbox{\omlijnd[breedte=1.5cm,kader=uit]{}
\omlijnd[breedte=2cm,kader=uit]{\tt linker}
\omlijnd[breedte=2.5cm,kader=uit]{\tt linker}
\omlijnd[breedte=3cm,kader=uit]{\tt tekst}
\omlijnd[breedte=2.5cm,kader=uit]{\tt rechter}
\omlijnd[breedte=2cm,kader=uit]{\tt rechter}}
\hbox{\omlijnd[breedte=1.5cm,kader=uit]{}
\omlijnd[breedte=2cm,kader=uit]{\tt rand}
\omlijnd[breedte=2.5cm,kader=uit]{\tt marge}
\omlijnd[breedte=3cm,kader=uit]{\tt tekst}
\omlijnd[breedte=2.5cm,kader=uit]{\tt marge}
\omlijnd[breedte=2cm,kader=uit]{\tt rand}}
\hbox{\omlijnd[breedte=1.5cm,kader=uit]{\tt boven}
\omlijnd[breedte=2cm]{}
\omlijnd[breedte=2.5cm]{}
\omlijnd[breedte=3cm]{}
\omlijnd[breedte=2.5cm]{}
\omlijnd[breedte=2cm]{}}
\hbox{\omlijnd[breedte=1.5cm,kader=uit]{\tt hoofd}
\omlijnd[breedte=2cm]{}
\omlijnd[breedte=2.5cm,achtergrond=raster]{}
\omlijnd[breedte=3cm,achtergrond=raster]{}
\omlijnd[breedte=2.5cm,achtergrond=raster]{}
\omlijnd[breedte=2cm]{}}
\hbox{\omlijnd[breedte=1.5cm,kader=uit,hoogte=3cm]{\tt tekst}
\omlijnd[breedte=2cm,hoogte=3cm]{}
\omlijnd[breedte=2.5cm,hoogte=3cm,achtergrond=raster]{}
\omlijnd[breedte=3cm,hoogte=3cm,achtergrond=raster]{}
\omlijnd[breedte=2.5cm,hoogte=3cm,achtergrond=raster]{}
\omlijnd[breedte=2cm,hoogte=3cm]{}}
\hbox{\omlijnd[breedte=1.5cm,kader=uit]{\tt voet}
\omlijnd[breedte=2cm]{}
\omlijnd[breedte=2.5cm,achtergrond=raster]{}
\omlijnd[breedte=3cm,achtergrond=raster]{}
\omlijnd[breedte=2.5cm,achtergrond=raster]{}
\omlijnd[breedte=2cm]{}}
\hbox{\omlijnd[breedte=1.5cm,kader=uit]{\tt onder}
\omlijnd[breedte=2cm]{}
\omlijnd[breedte=2.5cm]{}
\omlijnd[breedte=3cm]{}
\omlijnd[breedte=2.5cm]{}
\omlijnd[breedte=2cm]{}}}

Indien u de achtergronden in de grijze gebieden van
\in{figuur}[fig:paginavlakken] wilt wijzigen, typt u:

\startbuffer
\stelachtergrondenin
  [hoofd,tekst,voet]
  [linkermarge,tekst,rechtermarge]
  [achtergrond=raster]
\stopbuffer

\typebuffer

\stoponderdeel
