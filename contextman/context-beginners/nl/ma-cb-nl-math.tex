\startonderdeel ma-cb-nl-math

\produkt ma-cb-nl

\hoofdstuk[formules]{Wiskundig zetwerk}

\paragraaf{Inleiding}

\index {wiskundig zetwerk}
\index {mathematisch zetwerk}

\TEX\ is h\'et typesetting programma voor wiskundig zetwerk.
Ondanks dat is dit niet het uitgebreide hoofdstuk over
wiskundig zetwerk dat u zou verwachten. We raden u de
volgende boeken aan om het typesetten van formules onder de
knie te krijgen:\voetnoot{In deze inleiding op
wiskundig zetwerk is gebruik gemaakt van {\em \TEX niques}
van Arthur Samuel.}

\startopsomming[opelkaar]
\som {\em The \TeX Book} door D.E. Knuth
\som {\em The Beginners Book of \TeX} door S. Levy and R. Seroul
\stopopsomming

Bovendien zal in de toekomst een \CONTEXT\ math module en de
bijbehorende handleiding het licht zien.


\paragraaf{Het typesetten van wiskundig zetwerk}

\index {math mode}
\index {display mode}
\index {text mode}

Bij het typesetten van mathematisch zetwerk gelden
gewoonlijk andere regels dan bij het zetten van normale
tekst. Deze regels zijn in grote lijnen bekend bij \TEX\ en
we kunnen er daardoor op vertrouwen dat \TEX\ correct
mathematisch zetwerk oplevert.

We noemen hier een aantal regels voor mathematisch zetwerk:

\startopsomming[n,opelkaar]

\som Karakters worden gezet in $math\ italic$ (niet te
     verwarren met de normale {\it italic karakters} in een
     font).

\som Er worden symbolen gebruikt, zoals griekse letters
     ($\alpha$, $\chi$) en de gebruikelijke mathematische
     symbolen ($\leq$, $\geq$, $\in$).

\som De spati\"ering verschilt sterk van die van normale
     tekst.

\som Voor het uitlijnen van mathematische expressies gelden
     andere regels.

\som De sub|-| en superscripts worden automatisch verkleind,
     zoals in $a^{b}_{c}$.

\som De wijze van typesetten van sommige symbolen in de
     lopende tekst (text mode) verschilt van die van
     in losstaande formules (display mode).

\stopopsomming

Bij het opmaken van een wiskundige expressie werkt u in de
zogenaamde math mode en kunnen formules worden gedefinieerd
met behulp van plain \TEX||commando's.

Math mode kent twee verschijningsvormen: textmode en
displaymode. Wiskundige expressies in textmode worden tussen
\type{$} en \type{$} geplaatst, in displaymode tussen
\type{$$} en \type{$$}.

\startbuffer
De gemeente Hasselt beslaat een gebied van 42,05 \Square
\Kilo \Meter. Bij een denkbeeldige cirkel met de Markt als
middelpunt $M$ kan de diameter van de cirkel worden
berekend met ${{1}\over{4}} \pi r^2$.
\stopbuffer

\typebuffer

Dit wordt:

\haalbuffer

De vele \type{{}} in ${{1}\over{4}} \pi r^2$ zijn essentieel
om operaties in de expressie te scheiden. Als de buitenste
accolades worden weggelaten: \type{${1}\over{4} \pi r^2$},
wordt een ongewenst resultaat verkregen: ${1}\over{4} \pi
r^2$.

De letters en cijfers in een mathematische expressie worden
door \TEX\ in drie groottes gezet, namelijk
text size         $a+b$,
script size       $\scriptstyle a+b$ en
scriptscript size $\scriptscriptstyle a+b$.
We gebruiken hier de engelse uitdrukkingen, omdat we
invloed op \TEX\ kunnen uitoefenen met de commando's
\type{\scriptstyle} en \type{\scriptscriptstyle}.

Symbolen als $\int$ en $\sum$ worden in textmode
anders gezet dan in displaymode. Als u \type
{$\sum_{n=1}^{m}$} of \type {$\int_{-\infty}^{+\infty}$}
intypt, krijgt u {$\sum_{n=1}^{m}$} en
{$\int_{-\infty}^{+\infty}$}. Maar als u in de display
mode \type {$$\sum_{n=1}^{m}$$} en \type
{$$\int_{-\infty}^{+\infty}$$} intypt, krijgt u:

\startformule
\sum_{n=1}^{m} \quad {\rm en} \quad \int_{-\infty}^{+\infty}
\stopformule

Met de commando's \type {\nolimits} en \type{\limits}
kunnen we de weergave van \type{\sum} en \type{\int}
be\"{\i}nvloeden:

\startformule
\sum_{n=1}^{m}\nolimits \quad {\rm en} \quad \int_{-\infty}^{+\infty}\limits
\stopformule

Voor het zetten van breuken is het commando \type
{\over} beschikbaar. In \CONTEXT\ gebruiken we echter het
alternatief \type {\frac}. Om ${\frac{a}{1+b}}+c$ te
verkrijgen, typen we het volgende in
\type {${\frac{a}{1+b}}+c$}.

Andere commando's om expressies boven elkaar te zetten,
zijn:

\startbuffer[atop]
${a} \atop {b}$
\stopbuffer
\startbuffer[choose]
${n+1} \choose {k}$
\stopbuffer
\startbuffer[brack]
${m} \brack {n}$
\stopbuffer
\startbuffer[brace]
${m} \brace {n-1}$
\stopbuffer

\starttabulatie[|l|l|l|l|]
\NC \type {\atop}
\NC \typebuffer[atop]
\NC \mathstrut\haalbuffer[atop]
\NC
\NC\NR
\NC \type {\choose}
\NC \typebuffer[choose]
\NC
\NC \mathstrut\haalbuffer[choose]
\NC\NR
\NC \type {\brack}
\NC \typebuffer[brack]
\NC \mathstrut\haalbuffer[brack]
\NC
\NC\NR
\NC \type {\brace}
\NC \typebuffer[brace]
\NC
\NC \mathstrut\haalbuffer[brace]
\NC\NR
\stoptabulatie

\TEX\ vergroot expressiescheiders als (~) en $\{~\}$
automatisch als de linker en rechter scheider worden
voorafgegaan door respectievelijk de commando's \type
{\left} en \type {\right}. Als we \type
{$$1+\left(\frac{1}{1-x^{x-2}}\right)^3$$} typen, krijgen
we:

\startformule
1+\left(\frac{1}{1-x^{x-2}}\right)^3
\stopformule

Het realiseren van een sub|| en superscript gebeurt met de
karakters \citeer{\type{_}} en \citeer{\type{^}}. Ze hebben
betrekking op het eerstvolgende karakter, zodat bij meerdere
karakters gegroepeerd moet worden met $\{$~$\}$.

In specifieke situaties kunnen expressiescheiders ook worden
voorafgegaan door de commando's \type{\bigl}, \type{\Bigl},
\type{\biggl} en \type{\Biggl} en zijn rechtse tegenhangers.
Nog grotere scheiders kunnen worden gemaakt door
\type{\left} en \type{\right} in een
\type{\vbox}||constructie te plaatsen. Als we een
fantasie||expressie als \type{$$\left(\vbox to
16pt{}x^{2^{2^{2^{2}}}}\right)$$} typen, krijgen we:

\startformule
\left(\vbox to 16pt{}x^{2^{2^{2^{2}}}}\right)
\stopformule

In de display mode voldoen de volgende scheiders aan
dit mechanisme:

\starttabulatie[|l|l|l|l|l|l|l|l|]
\NC \type{\lfloor}      \NC $\lfloor$
\NC \type{\langle}      \NC $\langle$
\NC \type{\vert}        \NC $\vert$
\NC \type{\downarrow}   \NC $\downarrow$
\NC\NR
\NC \type{\rfloor}      \NC $\rfloor$
\NC \type{\rangle}      \NC $\rangle$
\NC \type{\Vert}        \NC $\Vert$
\NC \type{\Downarrow}   \NC $\Downarrow$
\NC\NR
\NC \type{\lceil}       \NC $\lceil$
\NC \type{/}            \NC $/$
\NC \type{\uparrow}     \NC $\uparrow$
\NC \type{\updownarrow} \NC $\updownarrow$
\NC\NR
\NC \type{\rceil}       \NC $\rceil$
\NC \type{\backslash}   \NC $\backslash$
\NC \type{\Uparrow}     \NC $\Uparrow$
\NC \type{\Updownarrow} \NC $\Updownarrow$
\NC\NR
\stoptabulatie

In de dispay mode zouden we eigenlijk maar een
breuk moeten plaatsen of anders overgaan op $a/b$
schrijfwijze. We typen voor:

\startformule
a_0 + {\frac{a}{a_1 + \frac{1}{a_2}}}
\stopformule

dus niet \type{$$a_0+{\frac{a}{a_1+\frac{1}{a_2}}}$$}
maar we geven de voorkeur aan
\type{$$a_0 + {\frac{a}{a_1 + 1/a_2}}$$},
zodat we het volgende krijgen:

\startformule
a_0 + {\frac{a}{a_1 + 1/a_2}}
\stopformule

Bovendien kan ook het commando \type{\displaystyle} worden
gebruikt. Als we bijvoorbeeld
\type {$$a_0 + {\frac{a}{a_1 + \frac{1}{\strut \displaystyle a_2}}}$$}
typen, krijgen we:

\startformule
a_0 + {\frac{a}{a_1 + \frac{1}{\displaystyle a_2}}}
\stopformule

In de voorbeelden hieronder illustreren we verder zonder
enig commentaar de commando's \type{\matrix},
\type{\pmatrix}, \type{\ldots}, \type{\cdots} en
\type{\cases}.

\start
\steltypenin[na={\vskip1ex}]

\starttypen
$$A=\left(\matrix{
             x-\lambda & 1         & 0         \cr
             0         & x-\lambda & 1         \cr
             0         & 0         & x-\lambda \cr}\right)$$
\stoptypen

\startformule
A=\left(\matrix{
             x-\lambda & 1         & 0         \cr
             0         & x-\lambda & 1         \cr
             0         & 0         & x-\lambda \cr}\right)
\stopformule

\starttypen
$$A=\left|\matrix{x-\mu& 1     & 0    \cr
                  0    & x-\mu & 1    \cr
                  0    & 0     &x-\mu \cr}\right|$$
\stoptypen
\startformule
A=\left|\matrix{x-\mu& 1     & 0    \cr
                0    & x-\mu & 1    \cr
                0    & 0     &x-\mu \cr}\right|
\stopformule

\starttypen
$$A=\pmatrix{a_{11} & a_{12} & \ldots & a_{1n} \cr
             a_{21} & a_{22} & \ldots & a_{2n} \cr
             \vdots & \vdots & \ddots & \vdots \cr
             a_{m1} & a_{m2} & \ldots & a_{mn} \cr}$$
\stoptypen

\startformule
A=\pmatrix{a_{11} & a_{12} & \ldots & a_{1n} \cr
           a_{21} & a_{22} & \ldots & a_{2n} \cr
           \vdots & \vdots & \ddots & \vdots \cr
           a_{m1} & a_{m2} & \ldots & a_{mn} \cr}
\stopformule

\starttypen
$$|x|=\cases{ x, & als $x\geq0$; \cr
             -x, & anders        \cr}$$
\stoptypen

\startformule
|x|=\cases{ x, & als $x\geq0$; \cr
           -x, & anders        \cr}
\stopformule
\stop

Als we tekst in een mathematische expressie willen opnemen
moeten we daar enige moeite voor doen. In de eerste plaats
wordt de spatie niet gezet en moeten we deze afdwingen met
\type{ \ } (backslash). Daarnaast moeten we een
fontwisseling aangeven, omdat we de tekst niet in $math\
italic$ willen zetten, maar in de actuele font. In \CONTEXT\
typen we bijvoorbeeld \type{$$x^3+{\tf de\ lagere\ orde\
termen}$$} om het volgende te krijgen:

\startformule
x^3+{\tf de\ lagere\ orde\ termen}
\stopformule

Voor de mathematische functies als $\sin$ en $\tan$ die als
normale tekst in een mathematische expressie worden
weergegeven, is in \TEX\ een aantal functies
voorgedefinieerd:

\starttabulatie[|l|l|l|l|l|l|l|l|]
\NC \type{\arccos} \NC \type{\cos} \NC \type{\csc} \NC \type{\exp} \NC \type{\ker} \NC \type{\limsup} \NC \type{\min} \NC \type{\sinh} \NC\NR
\NC \type{\arcsin} \NC \type{\cosh} \NC \type{\deg} \NC \type{\gcd} \NC \type{\lg} \NC \type{\ln} \NC \type{\Pr} \NC \type{\sup} \NC\NR
\NC \type{\arctan} \NC \type{\cot} \NC \type{\det} \NC \type{\hom} \NC \type{\lim} \NC \type{\log} \NC \type{\sec} \NC \type{\tan} \NC\NR
\NC \type{\arg} \NC \type{\coth} \NC \type{\dim} \NC \type{\inf} \NC \type{\liminf} \NC \type{\max} \NC \type{\sin} \NC \type{\tanh} \NC\NR
\stoptabulatie

Als we voor een sinusfunctie
\type {$$\sin 2\theta=2\sin\theta\cos\theta$$}
of voor een limietfunctie
\type {$$\lim_{x\to0}{\frac{\sin x}{x}}=1$$} intypen,
dan krijgen we:

\startformule
\sin 2\theta=2\sin\theta\cos\theta
\quad {\tf of} \quad
\lim_{x\to0}{\frac{\sin x}{x}}=1
\stopformule

Bij afleidingen willen we soms mathematische uitdrukkingen
uitlijnen, bijvoorbeeld onder het \citeer{$=$}||teken. Dit
doen we met het commando \type{\eqalign}. We typen
bijvoorbeeld:

\startbuffer
$$\eqalign{
      ax^2+bx+c &= 0                                \cr
              x &= \frac{-b \pm \sqrt{b^2-4ac}}{2a} \cr}$$
\stopbuffer

\typebuffer

en we krijgen:

\startformule
\eqalign{
      ax^2+bx+c &= 0                                \cr
              x &= \frac{-b \pm \sqrt{b^2-4ac}}{2a} \cr}
\stopformule

Lange mathematische expressies willen we misschien op
meerdere plaatsen laten uitlijnen. Let in het volgende
voorbeeld op de tweede regel:

\startbuffer
$$\eqalign{
     ax+bx+\cdots+yx+zx &         = x(a +b+ \cdots \cr
                        &\phantom{= x(a~}+y+z)     \cr
                        &         = y              \cr}$$
\stopbuffer

\typebuffer

Het resultaat hiervan is:

\startformule
\eqalign{
     ax+bx+\cdots+yx+zx &         = x(a +b+ \cdots \cr
                        &\phantom{= x(a~}+y+z)     \cr
                        &         = y              \cr}
\stopformule

Naast het commando \type{\phantom} bestaan ook de commando's
\type{\hphantom} zonder hoogte en diepte en \type{\vphantom}
zonder breedte.

Spati\"ering binnen een mathematische expressie kan meestal
worden overgelaten aan \TEX. In situaties waar toch moet
worden ingegrepen, zijn de volgende commando's beschikbaar:

\starttabulatie[|l|r|]
\NC \type{\!} \NC $-\frac{1}{6}$\type{\quad} \NC\NR
\NC \type{\,} \NC $\frac{1}{6}$\type{\quad}  \NC\NR
\NC \type{\>} \NC $\frac{2}{9}$\type{\quad}  \NC\NR
\NC \type{\;} \NC $\frac{5}{18}$\type{\quad} \NC\NR
\stoptabulatie

De \citeer{spaties} zijn gerelateerd aan \type{\quad}. Deze
maat staat voor de breedte van een hoofdletter \citeer{M}.

Het gebruik van het commando \type{\prime} spreekt voor zich
als we het volgende intypen \type{$y_1^\prime+y_2^{\prime\prime}$}
krijgen we namelijk $y_1^\prime+y_2^{\prime\prime}$.

We verkrijgen $\root 3 \of {x^2+y^2}$ met
\type{$\root 3 \of {x^2+y^2}$}.

Tot slot wijzen we op het effect van het commando
\type{\mathstrut} waarmee consistentie kan worden
afgedwongen in de hoogte van bijvoorbeeld het wortel||teken.
Met \type{$\sqrt{\mathstrut a}+\sqrt{\mathstrut d}+\sqrt{\mathstrut y}$}
krijgen we $\sqrt{\mathstrut a}+\sqrt{\mathstrut d}+\sqrt{\mathstrut y}$
in plaats van $\sqrt{a}+\sqrt{d}+\sqrt{y}$.

Zie \in{bijlage}[overzichten] voor een overzicht van
een groot aantal math commando's.

\paragraaf{Het plaatsen van formules}

\index{formule}

\Command{\tex{plaatsformule}}
\Command{\tex{startformule}}
\Command{\tex{stelformulesin}}

Het plaatsen van genummerde formules doet u met:

\shortsetup{plaatsformule}
\shortsetup{startformule}

Een tweetal voorbeelden:

\startbuffer
\plaatsformule[formule:eenformule]
  $$y=x^2$$

\plaatsformule
  \startformule
    \int_0^1 x^2 dx
  \stopformule
\stopbuffer

\typebuffer

\haalbuffer

De \CONTEXT||commando's \type{\startformule} en
\type{\stopformule} vervangen de begin en eind \type{$$}.
Als u namelijk typt:

\startbuffer
$$
\int_0^1 x^2 dx
$$
\stopbuffer

\typebuffer

dan krijgt u een expressie die weliswaar op het midden van
de pagina wordt 'gedisplayed', maar niet optimaal wordt
gespatieerd.

\haalbuffer

Het commando \type{\plaatsformule[]} zorgt voor de witruimte
voor en na een formule en zorgt tevens voor het nummeren. De
vierkante haken zijn optioneel en worden gebruikt voor het
verwijzen naar de formule en voor het aan- en uitzetten van
nummeren.

\startbuffer
\plaatsformule[eerste]
\startformule
  y=x^2
\stopformule

\plaatsformule[middelste]
\startformule
  y=x^3
\stopformule

\plaatsformule[laatste]
\startformule
  y=x^4
\stopformule
\stopbuffer

\haalbuffer

\in{Formule}[middelste] is als volgt ingevoerd:

\startbuffer
\plaatsformule[middelste]
  \startformule
     y=x^3
  \stopformule
\stopbuffer

\typebuffer

Het label \type{[middelste]} wordt gebruikt voor het
verwijzen naar deze formule. Zo'n referentie wordt
opgeroepen met \type{\in{formule}[middelste]}.

Indien nummering ongewenst is, typt u:

\type{\plaatsformule[-]}

Het nummeren van formules wordt ingesteld met
\type{\stelnummerenin}. In deze handleiding is het nummeren
ingesteld met \type{\stelnummerenin[wijze=perhoofdstuk]}.
Dit betekent dat het hoofdstuknummer voorafgaat aan het
formulenummer en dat het nummeren van de formule bij ieder
nieuw hoofdstuk wordt gereset. Uit oogpunt van consistentie
wordt het nummeren van tabellen, figuren en formules met
hetzelfde commando ingesteld. Het commando
\type{\stelnummerenin} dient dan ook in het instelgebied van
uw invoerfile te staan.

Formules kunnen worden ingesteld met:

\shortsetup{stelformulesin}

\stoponderdeel
