\startonderdeel ma-cb-nl-tablesofcontent

\produkt ma-cb-nl

\hoofdstuk{Inhoudsopgave (lijsten)}

\index{inhoudsopgave}
\index{lijst}

\Command{\tex{volledigeinhoud}}
\Command{\tex{plaatsinhoud}}
\Command{\tex{definieerlijst}}
\Command{\tex{stellijstin}}
\Command{\tex{schrijfnaarlijst}}
\Command{\tex{schrijftussenlijst}}
\Command{\tex{definieersamengesteldelijst}}
\Command{\tex{stelsamengesteldelijstin}}

Een inhoudsopgave bevat hoofdstuktitels, hoofdstuknummers en
paginanummers en kan worden uitgebreid met de titels van
paragrafen, subparagrafen enz. Een inhoudsopgave wordt
automatisch gegenereerd door in te typen:

\starttypen
\plaatsinhoud
\stoptypen

Afhankelijk van de locatie in het document wordt vervolgens
een inhoudsopgave aangemaakt. Aan het begin van een document
wordt een volledige inhoudsopgave gegenereerd, bestaande uit
hoofdstukken, paragrafen, subparagrafen enz.
Aan het begin van een hoofdstuk echter:

\startbuffer
\hoofdstuk{Hasselt in de zomer}

\plaatsinhoud

\paragraaf{Hasselt in juli}

\paragraaf{Hasselt in augustus}
\stopbuffer

\typebuffer

wordt alleen een lijst van (sub)paragraaftitels en de
corresponderende paragraafnummers en paginanummers geplaatst.

Het commando \type{\plaatsinhoud} is beschikbaar nadat een
zogenaamde samengestelde lijst is gedefinieerd met:

\shortsetup{definieersamengesteldelijst}

Dit commando en \type{\definieerlijst} staan u toe om uw
eigen lijsten aan te maken die nodig zijn om een document te
structureren.

Het gebruik van deze commando's wordt ge\"{\i}llustreerd aan de
inhoudsopgave.

\startbuffer
\definieerlijst[hoofdstuk]
\stellijstin
   [hoofdstuk]
   [voor=\blanko,
    na=\blanko,
    letter=vet]

\definieerlijst[paragraaf]
\stellijstin
   [paragraaf]
   [variant=d]
\stopbuffer

\typebuffer

Tijdens het verwerken van de invoerfile worden twee lijsten
aangemaakt die gecombineerd worden tot \'e\'en inhoudsopgave
met \type{\definieersamengesteldelijst}.

\startbuffer
\defineersamengesteldelijst
   [inhoud]
   [hoofdstuk,paragraaf]
   [niveau=subparagraaf]
\stopbuffer

\typebuffer

Er zijn nu twee commando's beschikbaar \type{\plaatsinhoud}
en \type{\volledigeinhoud}. Met het tweede commando wordt
tevens de titel {\em inhoud} boven de inhoudsopgave geplaatst.

De vormgeving van de lijsten kan worden be\"{\i}nvloed met
de parameter \type{variant}.

\plaatstabel
  [hier,forceer]
  [tab:alternatieven]
  {Alternatieven voor lijstweergave.}
\starttabel[|c|l|]
\HL
\NC \bf Alternatief \NC \bf Weergave \NC\SR
\HL
\NC \type{a} \NC nummer -- titel -- paginanummer            \NC\FR
\NC \type{b} \NC nummer -- titel -- spaties -- paginanummer \NC\MR
\NC \type{c} \NC nummer -- titel -- dots -- paginanummer    \NC\MR
\NC \type{d} \NC nummer -- titel -- paginanummer (continu)  \NC\MR
\NC \type{e} \NC gereserveerd voor interactieve documenten  \NC\MR
\NC \type{f} \NC gereserveerd voor interactieve documenten  \NC\LR
\HL
\stoptabel

De lijsten worden ingesteld met:

\shortsetup{stellijstin}
\shortsetup{stelsamengesteldelijstin}

Indien u de layout van een inhoudsopgave wilt veranderen dan
moet u in gedachten houden dat het om een lijst gaat.

\startbuffer
\stelsamengesteldelijstin
  [inhoud]
  [variant=c,
   titeluitlijnen=nee,
   breedte=2.5cm]
\stopbuffer

\typebuffer

Het resultaat is een iets andere layout dan de standaard
layout.

Lijsten worden geplaatst met:

\shortsetup{plaatslijst}

Indien u een inhoud plaatst, kunt u bijvoorbeeld intypen:

\starttypen
\plaatslijst[inhoud][niveau=paragraaf]
\stoptypen

of:

\starttypen
\plaatsinhoud[niveau=paragraaf]
\stoptypen

Dan worden alleen de paragrafen in de inhoudsopgave opgenomen.
Een dergelijke optie komt van pas bij documenten waarin
bijvoorbeeld subsubsubsubsubparagrafen voorkomen die u niet
in de inhoudsopgave wilt opnemen.

Een lange lijst of een grote inhoudsopgave neemt meer dan
\'e\'en pagina in beslag. Om een pagina||overgang af te kunnen
dwingen is het volgende commando beschikbaar:

\starttypen
\volledigeinhoud[2.2,8.5,12.3.3]
\stoptypen

Een nieuwe pagina wordt gegenereerd na paragraaf 2.2 en 8.5
en subparagraaf~12.3.3.

Sporadisch heeft u wellicht de behoefte om teksten in uw
inhoudsopgave tussen te voegen. Dit wordt gedaan met:

\shortsetup{schrijfnaarlijst}
\shortsetup{schrijftussenlijst}

Indien u bijvoorbeeld een opmerking in uw inhoudsopgave wilt
maken vlak na een paragraaf met de titel {\em Hotels in
Hasselt} dan kunt u bijvoorbeeld intypen:

\startbuffer
\paragraaf{Hotels in Hasselt}
\schrijftussenlijst[paragraaf]{\blanko}
\schrijfnaarlijst[paragraaf]{}{---moet nog worden aangemaakt---}
\schrijftussenlijst[paragraaf]{\blanko}
\stopbuffer

\typebuffer

\stoponderdeel
