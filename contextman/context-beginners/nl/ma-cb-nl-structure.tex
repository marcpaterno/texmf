\startonderdeel ma-cb-nl-structure

\produkt ma-cb-nl

\hoofdstuk{Het defini\"eren van een document}

Ieder document wordt gestart met \type{\starttekst} en
afgesloten met \type{\stoptekst}. Alle overige invoer vindt
plaats tussen deze commando's en \CONTEXT\ verwerkt alleen
die informatie.

Instellingen die van toepassing zijn op het gehele document
worden in het instelgebied voor \type{\starttekst} gedaan.

\startbuffer
\stelkorpsin[12pt]
\starttekst
Dit is een eenregelig document.
\stoptekst
\stopbuffer

\typebuffer

Binnen \type{\starttekst} $\cdots$ \type{\stoptekst}
kan een document worden opgedeeld in vier secties:

\startopsomming[n,opelkaar]
\som inleidingen
\som hoofdteksten
\som uitleidingen
\som bijlagen
\stopopsomming

Deze secties worden gedefinieerd met:

\starttypen
\startinleidingen  ... \stopinleidingen
\starthoofdteksten ... \stophoofdteksten
\startuitleidingen ... \stopuitleidingen
\startbijlagen     ... \stopbijlagen
\stoptypen

In de in- en uitleidende secties produceert het commando
\type{\hoofdstuk} een niet genummerde titel in de
inhoudsopgave. Bovendien kan een afwijkende paginanummering
worden ingesteld. Binnen {\em inleidingen} worden meestal de
inhoudsopgave, lijsten met figuren of tabellen, het
voorwoord, het dankwoord enz. geplaatst.

De sectie bijlagen is (inderdaad) bedoeld voor bijlagen. De
koppen worden alfabetisch genummerd. Ook in deze sectie kan
gewoon \type{\hoofdstuk} worden gebruikt.

De secties worden ingesteld met:

\shortsetup{stelsectieblokin}

\stoponderdeel
