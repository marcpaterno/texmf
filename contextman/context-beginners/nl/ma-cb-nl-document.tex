\startonderdeel ma-cb-nl-document

\produkt ma-cb-nl

\hoofdstuk{Het maken van een document}

\index{invoerfile}

Laten we aannemen dat u een eenvoudig document wilt maken.
Het heeft enige structuur en bevat een titelpagina, een
aantal hoofdstukken en paragrafen. Natuurlijk is er een
inhoudsopgave en een index.

\CONTEXT\ maakt een dergelijk document automatisch als u de
juiste invoer aanreikt door middel van een file. U dient dus
eerst een zogenaamde invoerfile te maken. Een invoerfile
heeft een naam en een extensie. U kunt een willekeurige naam
kiezen, maar de extensie dient \type{tex} te zijn. Als u een
file met de naam \type{mijnfile.tex} maakt, zult u geen
problemen tegenkomen bij het runnen van \CONTEXT.

Een \paginareferentie[invoerfile] invoerfile zou er als
volgt uit kunnen zien:

\startbuffer
\starttekst

\startstandaardopmaak
\regelmidden{Een Document Titel}
\regelmidden{door}
\regelmidden{De Auteur}
\stopstandaardopmaak

\volledigeinhoud

\hoofdstuk{Inleiding}

... uw tekst\index{een indexwoord} ...

\hoofdstuk{Eerste hoofdstuk}

\paragraaf[eerste paragraaf]{De eerste paragraaf}

... uw tekst ...

\paragraaf{De tweede paragraaf}

\subparagraaf{de eerste subparagraaf}

... uw tekst\index{nog een indexwoord} ...

\subparagraaf{de tweede subparagraaf}

... uw tekst ...

\paragraaf{De derde paragraaf}

... uw tekst ...

\hoofdstuk{Nog een hoofdstuk}

... uw tekst ...

\hoofdstuk[laatste hoofdstuk]{Het laatste hoofdstuk}

... uw tekst ...

\volledigeindex

\stoptekst
\stopbuffer

{\switchnaarkorps[9pt]\typebuffer}

\CONTEXT\ verwacht een \ASCII\ invoerfile. Natuurlijk kunt u
iedere tekstverwerker gebruiken, maar u dient niet te
vergeten dat \CONTEXT\ alleen \ASCII\ invoer kan verwerken.
De meeste tekstverwerkers kunnen uw invoerfile exporteren
als standaard \ASCII\ ook wel {\em tekst} genoemd.

De invoerfile bevat de tekst die u wilt zetten en de
\CONTEXT||commando's. Een \CONTEXT||commando begint met een
backslash~\tex{}. Met het commando \type{\starttekst} geeft
u het begin van de tekst aan. Het gebied voor
\type{\starttekst} wordt het instelgebied genoemd en wordt
gebruikt voor het defini\"eren van nieuwe commando's en het
instellen van de layout van uw document.

Een commando wordt meestal gevolgd door een linker en
rechter vierkante haak \type{[]} en/of een linker en rechter
accolade \type{{}}. Het in het voorbeeld gegeven commando
\type{\hoofdstuk[laatste hoofdstuk]{Het laatste hoofdstuk}}
vertelt aan \CONTEXT\ een aantal acties uit te voeren ten
behoeve van layout, typografie en structuur. Die acties zijn
bijvoorbeeld:

\startopsomming[n,opelkaar]
\som begin op een nieuwe pagina
\som verhoog het hoofdstuknummer met \'e\'en
\som plaats het hoofdstuknummer voor de hoofdstuktitel
\som reserveer witruimte na de hoofdstuktitel
\som gebruik een grote letter voor de titel
\som plaats de titel en het nummer in de inhoudsopgave
\stopopsomming

Deze acties worden uitgevoerd op het argument dat tussen de
beide accolades staat: {\em Het laatste hoofdstuk}.

Tot nu toe is nog niet gesproken over \type{[laatste
hoofdstuk]} dat tussen het commando en de titel staat. Dit
is een label met een logische naam dat wordt gebruikt om
naar het bewuste hoofdstuk te verwijzen. Dit wordt gedaan
met het commando \type{\in[laatste hoofdstuk]} dat het
nummer teruggeeft of het commando \type{\over[laatste
hoofdstuk]} dat de titel genereert.

Nu kan de lijst met acties worden uitgebreid met:

\startopsomming[verder]
\som geef het label \type{laatste hoofdstuk} de waarde van
     het hoofdstuknummer (en sla het nummer en de titel
     op voor later gebruik)
\stopopsomming

Andere acties met betrekking tot voetteksten, resetten van
nummering en interactiviteit worden hier nog buiten
beschouwing gelaten.

Indien u \CONTEXT\ dit voorbeeld van een invoerfile laat
processen, krijgt u een zeer eenvoudig document met een
aantal genummerde hoofdstukken en paragrafen, een
inhoudsopgave en een register dat twee ingangen bevat.

Tijdens het processen van de invoerfile handelt \CONTEXT\
veel zaken af. Een van die zaken is bijvoorbeeld
paginanummering. Maar om een inhoudsopgave aan te maken (die
meestal aan het begin van het document moet worden
geplaatst) heeft \CONTEXT\ in eerste instantie te weinig
informatie. De invoerfile moet daarom twee maal worden
verwerkt.

\CONTEXT\ produceert een aantal hulpfiles om bepaalde
informatie op te slaan. Deze files worden verwerkt door
\TEXUTIL. In een enkel geval moet de invoerfile drie maal
worden verwerkt door \CONTEXT. U kunt \TEXEXEC\ gebruiken om
\CONTEXT\ van de commandline op te roepen. Dit \PERL||script
houdt ook bij hoe vaak een file moet worden verwerkt.
\TEXEXEC\ is onderdeel van de standaard \CONTEXT||distributie.

\stoponderdeel
