\startonderdeel ma-cb-nl-interactivity

\produkt ma-cb-nl

\hoofdstuk{Interactiviteit in elektronische documenten}

\paragraaf{Inleiding}

\index[reader]{\READER}
\index[exchange]{\EXCHANGE}
\index[distiller]{\DISTILLER}

Documenten kunnen elektronisch worden uitgegeven, zodat ze
op een computer kunnen worden geraadpleegd en op een scherm
kunnen worden weergegeven.

Interactiviteit betekent dat specifieke gebieden in het
document actief (hyperlinks) zijn gemaakt. Dit houdt in dat
die gebieden (meestal met de muis) kunnen worden
geselecteerd en aangeklikt. Het aanklikken resulteert in een
sprong naar het aangewezen doelgebied. Bij het raadplegen
van een index kan bijvoorbeeld op een ingang worden geklikt,
waarna naar de corresponderende pagina wordt gesprongen.

Interactie heeft betrekking op:

\startopsomming[opelkaar]
\som actieve hoofdstuknummers in inhoudsopgaven
\som actieve paginanummers in registers
\som actieve paginanummers, hoofdstuknummers en
     figuurnummers in verwijzingen naar pagina's,
     hoofdstukken, figuren enz.
\som actieve titels, paginanummer, hoofdstuknummers in
     externe verwijzingen naar andere interactieve
     documenten
\som actieve menu's ten behoeve van navigatie hulpmiddelen
\stopopsomming

De interactiviteit hangt af van het programma dat wordt
gebruikt voor het bekijken van het document. In deze
handleiding wordt ervan uitgegaan dat u \PDFTEX\ gebruikt
voor het produceren van uw \PDF||documenten of \DISTILLER\
gebruikt voor het omzetten van uw \POSTSCRIPT||documenten.
Vervolgens kunt u die documenten bekijken of raadplegen met
\READER, \EXCHANGE\ of \GHOSTVIEW.

\CONTEXT\ is een zeer goed hulpmiddel voor de produktie van
elektronische of interactieve \PDF||documenten. In deze
handleiding wordt slechts een klein deel van de
functionaliteit besproken. De auteurs hebben echter besloten
alle documenten rond \CONTEXT, inclusief de broncode, tevens
elektronisch beschikbaar te maken, zodat u een goed inzicht
kan krijgen in de mogelijkheden van \CONTEXT.

\paragraaf{Interactie}

\index{interactie}

\Command{\tex{stelinteractiein}}

Interactiviteit wordt geactiveerd door:

\shortsetup{stelinteractiein}

Bijvoorbeeld:

\startbuffer
\stelinteractiein
  [status=start,
   kleur=groen,
   letter=vet]
\stopbuffer

\typebuffer

De zogenaamde hyperlinks worden nu automatisch gegenereerd
en actieve woorden worden vetgroen weergegeven.

Het interactieve document is aanzienlijk groter (in
MegaBytes) dan zijn papieren tegenhanger, omdat hyperlinks
ruimte in beslag nemen. Ook de verwerkingstijd van een
document neemt toe. Het is daarom verstandig de interactie
pas te activeren als het document zijn eindstadium heeft
bereikt.

\paragraaf{Interactie binnen een document}

\Command{\tex{in}}
\Command{\tex{op}}
\Command{\tex{naar}}

In \in{hoofdstuk}[verwijzen] heeft u gezien dat u
verwijzingen kunt aanmaken met \type{\in} en \type{\op}. U
zult zich wellicht hebben afgevraagd waarom u ook {\em
hoofdstuk} moest intypen bij een verwijzing als
\type{\in{hoofdstuk}[introductie]}. In de eerste plaats
worden {\em hoofdstuk} en het corresponderende nummer niet
van elkaar gescheiden bij regelovergangen. In de tweede
plaats worden zowel het woord {\em hoofdstuk} als het
hoofdstuknummer in de interactieve toestand afwijkend gezet
(meestal vet groen) en worden beide aanklikbaar. Hierdoor
kan de gebruiker makkelijker een doelgebied selecteren.

Er is een commando dat alleen betekenis heeft
in een interactief document.

\shortsetup{naar}

De accolades bevatten tekst en de haken omsluiten de
verwijzing.

\startbuffer
In \naar{Hasselt}[fig:cityplan] zijn de straten
cirkelvormig aangelegd.
\stopbuffer

\typebuffer

In het interactieve document is {\em Hasselt} groen
en actief. Er wordt een sprong gerealiseerd naar een kaart
van Hasselt.

\paragraaf{Interactie tussen documenten}

\Command{\tex{uit}}
\Command{\tex{gebruikexterndocument}}

Het is mogelijk om van en naar meerdere documenten te
springen. Allereerst dient u het document te defini\"eren
waarnaar u wilt verwijzen.

\shortsetup{gebruikexterndocument}

De eerste haken bevatten een logische naam voor het externe
document, het tweede paar de filenaam zonder extensie en het
derde paar wordt gebruikt voor een titel van het document.

Vervolgens kunt u refereren naar het externe document met:

\shortsetup{uit}

De accolades bevatten tekst en de vierkante haken de
verwijzing. Hierna volgt een voorbeeld.

\startbuffer
\gebruikexterndocument[hia][hasboek][Hasselt in augustus]

De meeste toeristische attracties worden beschreven in
\uit[hia]. Een beschrijving van het Eui||feest wordt gegeven
in \uit[hia::euifeest]. Een beschrijving van het
\naar{Eui--feest}[hia::euifeest] vindt u in \uit[hia]. Het
eui||feest is beschreven op \op{pagina}[hia::euifeest] in
\uit[hia]. Zie voor meer informatie
\in{hoofdstuk}[hia::euifeest] in \uit[hia].
\stopbuffer

\typebuffer

Het commando \type{\gebruikexterndocument} wordt meestal in
het instelgebied van de invoerfile gedefinieerd.

De dubbele \type{::} geven aan dat het gaat om een
referentie naar een extern document.

Na het verwerken van uw invoerfile en de file
\type{hasboek.tex} (allebei ten minste twee maal verwerkt
ten behoeve van de referenties) heeft u twee
\PDF||documenten. De referenties hierboven hebben de
volgende betekenis:

\startopsomming[opelkaar]
\som \type{\uit[hia]} produceert een actieve titel die u in
     het derde hakenpaar van het commando
     \type{\gebruikexterndocument} heeft gedefinieerd en is
     gelinked (verwijst) naar de eerste pagina van
     \type{hasboek.pdf}
\som \type{\uit[hia::euifeest]} produceert een actieve titel
     en is gelinked (verwijst) naar de pagina waar hoofdstuk {\em
     Eui||feest} begint
\som \type{\naar{Eui--feest}[hia::euifeest]} produceert een
     actief woord {\em Eui||feest} en is gelinked (verwijst)
     naar de pagina waar hoofdstuk {\em Eui||feest} begint
\som \type{\op{pagina}[hia::euifeest]} produceert een actief
     woord {\em pagina} en paginanummer en is gelinked
     (verwijst) naar die pagina
\som \type{\in{hoofdstuk}[hia::euifeest]} produceert een
     actief woord {\em hoofdstuk} en hoofdstuknummer en is
     gelinked aan dat hoofdstuk
\stopopsomming

Zoals u ziet scheidt de \type{::} de (logische) filenaam
en het doelgebied.

\paragraaf{Menu's}

U kunt navigatiehulpmiddelen defini\"eren met:

\shortsetup{definieerinteractiemenu}

De eerste haakjes zijn bedoeld voor een logische naam van
het menu, waarmee het menu in een later stadium kan worden
opgeroepen. Het tweede paar wordt gebruikt om de plaats op
het scherm vast te leggen. Het derde paar bevat de
instellingen.

Een menudefinitie kan er als volgt uitzien:

\startbuffer
\stelkleurenin
  [status=start]

\stelinteractiein
  [status=start,
   menu=aan]

\definieerinteractiemenu
  [mijnmenu]
  [rechts]
  [status=start,
   uitlijnen=midden,
   achtergrond=raster,
   kader=aan,
   breedte=\margebreedte,
   letter=kleinvet,
   kleur=]

\stelinteractiemenuin
  [mijnmenu]
  [{Inhoud[inhoud]},
   {Index[index]},
   {\vfill},
   {Stoppen[VerlaatViewer]}]
\stopbuffer

\typebuffer

Deze definitie produceert een menu aan de rechterkant van
ieder scherm. De menuknoppen bevatten de teksten {\em
Inhoud}, {\em Index} en {\em Stoppen} en hebben
respectievelijk de volgende functies: een sprong naar de
inhoudsopgave, een sprong naar de index en het verlaten van
de viewer. De labels \type{inhoud} en \type{index} zijn
voorgedefinieerd. Andere voorgedefinieerde locaties zijn
\type{EerstePagina}, \type{LaatstePagina},
\type{VolgendePagina} en \type{VorigePagina}.

Een actie als \type{VerlaatViewer} is nodig om het
elektronische document zo onafhankelijk mogelijk te maken
van de viewer. Andere voorgedefinieerde acties zijn
\type{VorigeSprong}, \type{DoorzoekDocument} en
\type{PrintDocument}. De betekenis van deze acties spreken
voor zich.

Menu's worden ingesteld met:

\shortsetup{stelinteractiemenuin}

\stoponderdeel
