\startonderdeel ma-cb-nl-figures

\produkt ma-cb-nl

\hoofdstuk[figuren]{Figuren}

\index{figuur}
%\zieindex{picture}{figure}
\index{floating blocks}

\Command{\tex{plaatsfiguur}}
\Command{\tex{startfiguurtekst}}
\Command{\tex{stelfigurenin}}
\Command{\tex{gebruikexternfiguur}}
\Command{\tex{startcombinatie}}
\Command{\tex{stelplaatsblokkenin}}
\Command{\tex{stelblokkopjesin}}
\Command{\tex{externfiguur}}

Foto's en figuren worden in uw document geplaatst met het
volgende commando:

\startbuffer
\plaatsfiguur
   []
   [fig:kerk]
   {Stephanus Kerk.}
   {\externfiguur[ma-cb-24][breedte=.4\tekstbreedte]}
\stopbuffer

\typebuffer

Na verwerking wordt dit:

\haalbuffer

Het commando \type{\plaatsfiguur} zorgt voor nummering en
verticale witruimte voor en na het figuur. Bovendien
initieert het commando het floatmechanisme. Dit mechanisme
zorgt ervoor dat de figuur blijft 'zweven' tot de meest
optimale locatie in het document is gevonden. Meestal is dit
verplaatsen van een figuur nodig, omdat figuren relatief
groot zijn. Het floatmechanisme kan worden be\"{\i}nvloed binnen
het eerste hakenpaar.

Het commando \type{\plaatsfiguur} is een verbijzondering van
het commando:

\shortsetup{plaatsblok}

De mogelijke opties van \type{\plaatsfiguur} worden
beschreven in \in{tabel}[tab:plaatsfiguur].

\plaatstabel
  [hier]
  [tab:plaatsfiguur]
  {Opties in \type{\plaatsfiguur}.}
\starttabel[|l|l|]
\HL
\NC \bf Optie \NC \bf Betekenis \NC\SR
\HL
\NC hier       \NC  plaats figuur bij voorkeur op deze locatie  \NC\FR
\NC forceer    \NC  plaats de figuur altijd op deze lokatie     \NC\MR
\NC pagina     \NC  plaats figuur op een eigen pagina           \NC\MR
\NC boven      \NC  plaats de figuur boven aan de pagina        \NC\MR
\NC onder      \NC  plaats de figuur onder aan de pagina        \NC\MR
\NC links      \NC  plaats figuur tegen de linkermarge          \NC\MR
\NC rechts     \NC  plaats figuur tegen de rechtermarge         \NC\MR
\NC marge      \NC  plaats figuur in de marge                   \NC\LR
\HL
\stoptabel

Het tweede paar vierkante haken wordt gebruikt voor
verwijzingen naar de figuur. U kunt naar dit figuur
verwijzen door te typen:

\starttypen
\in{figuur}[fig:kerk]
\stoptypen

De eerste accolades van \type{\plaatsfiguur} worden gebruikt
voor de bijschrijft. Op die plaats kan iedere gewenste tekst
worden ingevoerd. Indien u geen nummering of bijschrift
wenst, typt u \type{{geen}}. Het label {\em figuur} wordt
ingesteld met \type{\stelblokkopjesin} en het nummeren wordt
ingesteld of hersteld met \type{\stelnummerenin} (zie
\in{paragraaf}[floatingblocks]).

Het tweede paar accolades wordt gebruikt om externe
figuurfiles aan te roepen of het figuur te defini\"eren.

In het volgende voorbeeld ziet u hoe
\inlijnd[hoogte=1em]{Hasselt} wordt gedefinieerd binnen het
laatste accoladepaar en wordt de functie van
\type{\plaatsfiguur{}{}} toegelicht.

\startbuffer
\plaatsfiguur
  {Een omlijnd Hasselt.}
  {\omlijnd{\tfd Hasselt}}

\stopbuffer

\typebuffer

Dit wordt:

\haalbuffer

Meestal worden figuren opgemaakt of opgewerkt in pakketten
als Corel Draw of Illustrator en foto's worden --- na inscannen
--- opgewerkt met PhotoShop. De figuren zijn dan beschikbaar als
files. Welke types kunnen worden gebruikt hangt af van het
gebruikte back||end. In geval van \PDFTEX\ zijn \type {JPG}, \type
{PNG} en \type {PDF} files terwijl ook \METAPOST\ uitvoer kan
worden afgehandeld. In een normale situatie zoekt \CONTEXT\ het
best beschikbare fileformaat uit.

In \in{figuur}[fig:grachten] ziet u een foto en een
opgemaakt plaatje in \'e\'en figuur.

\startbuffer
\plaatsfiguur
  [hier,forceer]
  [fig:grachten]
  {Grachten in Hasselt.}
  {\startcombinatie[2*1]
     {\externfiguur[ma-cb-03][breedte=.4\tekstbreedte]}
        {een bitmap foto}
     %{\externfiguur[gracht][breedte=.4\tekstbreedte]}
     {\externfiguur[ma-cb-00][width=.4\tekstbreedte]}
        {een vector plaatje}
   \stopcombinatie}
\stopbuffer

\haalbuffer

U kunt deze figuur als volgt maken:

\typebuffer

In de figuur worden twee plaatjes gecombineerd met:

\shortsetup{startcombinatie}

De commando's \type{\startcombinatie} $\cdots$
\type{\stopcombinatie} worden gebruikt voor het combineren
van twee of meer plaatjes in \'e\'en figuur. Het aantal
plaatjes wordt tussen de vierkante haakjes ingetypt. Wanneer
de plaatjes onder elkaar moeten staan, typt u \type{[1*2]}.
U kunt zich misschien voorstellen wat er gebeurt als u zes
plaatjes combineert met \type{[3*2]}
(\type{[kolommen*rijen]}).

De voorbeelden hiervoor zijn eigenlijk al voldoende om goed
ge\"{\i}llustreerde documenten te cre\"eren. Soms is het echter
wenselijk om figuur en tekst meer met elkaar te combineren.
U kunt daarvoor gebruik maken van het commando:

\shortsetup{startbloktekst}

Het effect van:

\startbuffer
\startfiguurtekst
  [links]
  [fig:inwoners]
  {geen}
  {\externfiguur[ma-cb-18][breedte=.5\zetbreedte]}
  Het inwonersaantal van Hasselt heeft altijd
  gevarieerd met de economische ontwikkelingen. De
  Dedemsvaart werd bijvoorbeeld rond 1810 gegraven. Dit
  kanaal loopt door Hasselt en de handel floreerde. De
  populatie nam binnen 10 jaar toe met ongeveer 40\%.
  Tegenwoordig heeft de Dedemsvaart geen economische waarde
  meer en zijn de grachten slechts een toeristische
  attractie. Toch vind je nog overal herinneringen aan die
  bloeiende tijd.
\stopfiguurtekst
\stopbuffer

\typebuffer

wordt in de onderstaande figuur getoond.

\start
\steltolerantiein[zeersoepel]
\haalbuffer
\stop

\shortsetup{externfiguur}

Het laatste paar accolades bevat het commando
\type{\externfiguur}. Dit commando geeft u de vrijheid om
met een figuur te doen wat u wilt. Het commando
\type{\externfiguur} heeft twee paren van vierkante haken.
Het eerste paar wordt gebruikt voor de filenaam van de
figuur zonder extensie, het tweede voor het fileformaat en
de dimensies. Het is niet moeilijk te voorspellen wat er
gebeurt als u het volgende intypt:\voetnoot{Zie
\op{pagina}[margefiguur].}

\startbuffer[margefiguur]
\inmarge
  {\externfiguur
     [ma-cb-23]
     [breedte=\margebreedte]}
\stopbuffer

\typebuffer[margefiguur]

De opmaak van figuren wordt ingesteld met:

\shortsetup{stelplaatsblokkenin}

De nummering en de labels worden ingesteld met:

\shortsetup{stelblokkopjesin}

De beide commando's worden in het instelgebied van de
invoerfile geplaatst en zijn geldig voor het totale
document.

\startbuffer
\stelplaatsblokkenin
  [plaats=rechts]

\stelblokkopjesin
  [plaats=boven,
   letter=vetschuin]

\plaatsfiguur
  {Een karakteristiek zicht op Hasselt.}
  {\externfiguur[ma-cb-12][breedte=8cm]}
\stopbuffer

\typebuffer

\start
\haalbuffer
\stop

\stoponderdeel
