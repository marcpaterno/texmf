\startonderdeel ma-cb-nl-footnotes

\produkt ma-cb-nl

\hoofdstuk{Voetnoten}

\index{voetnoten}

\Command{\tex{voetnoot}}
\Command{\tex{stelvoetnotenin}}

Indien u een tekst wilt annoteren met voetnoten gebruikt u
\type{\voetnoot}. Het commando ziet er als volgt uit:

\shortsetup{voetnoot}

De vierkante haken zijn optioneel en bedoeld voor een
logische naam waarmee het voetnootnummer meerdere malen kan
worden opgeroepen. De accolades bevatten de tekst die in de
voetnoot moet worden weergegeven.

Een voetnootnummer kan worden opgeroepen met:

\shortsetup{noot}

Indien u de onderstaande tekst heeft ingevoerd:

\startbuffer
De Hanze was een laat||middeleeuws commercieel verbond van
steden in het Noorden en rond de Baltische Zee. Het verbond
werd gesloten voor het bevorderen en de bescherming van de
handel van haar leden.\voetnoot[oorlog]{Dit was de bron van
veel jaloezie tussen steden onderling en veroorzaakte menige
oorlog.} In de Hanzeperiode werd levendig gehandeld in hout,
wol, metaal, stoffen, zout, wijn en bier. De handel
veroorzaakte grote groei in de Hanzesteden\voetnoot{Hasselt
was ook zo'n stad.} en het lokte vele oorlogen
uit.\noot[oorlog]
\stopbuffer

\typebuffer

Krijgt u:

\haalbuffer

Het nummeren van voetnoten gebeurt automatisch. De wijze van
weergave van voetnoten wordt ingesteld met het commando:

\shortsetup{stelvoetnotenin}

Voetnoten hoeven niet altijd onderaan een pagina te worden
geplaatst, maar kunnen bijvoorbeeld ook aan het eind van een
hoofdstuk worden geplaatst met het commando:

\shortsetup{plaatsvoetnoten}

U kunt voetnoten ook koppelen aan bijvoorbeeld een tabel. We
spreken dan van zogenaamde lokale voetnoten. De commando's
zijn:

\shortsetup{startlokalevoetnoten}

\shortsetup{plaatslokalevoetnoten}


Een voorbeeld
illustreert het gebruik:

\startbuffer
\startlokalevoetnoten[n=0]
  \plaatstabel
    {Teruggang van bedrijvigheid in Hasselt.}
     \starttabel[|l|c|c|c|c|]
       \HL
       \NC
       \NC Ovens\voetnoot{Bron: Uit de geschiedenis van Hasselt.}
       \NC Smidsen \NC Brouwketels \NC Pannebakkerijen \NC\SR
       \HL
       \NC 1682 \NC 15 \NC 9 \NC 3 \NC 2 \NC\FR
       \NC 1752 \NC ~6 \NC 4 \NC 0 \NC 0 \NC\LR
       \HL
       \stoptabel
  \plaatslokalevoetnoten
\stoplokalevoetnoten
\stopbuffer

\typebuffer

\haalbuffer

Ook kunnen voetnoten, in plaats van onderaan een bladzijde,
aan het eind van een tekst of hoofdstuk worden geplaatst.
Hiervoor zetten we \type {plaats} op \type {tekst} en
gebruiken het volgende commando om de noten te plaatsen:

\shortsetup{plaatsvoetnoten}

\stoponderdeel
