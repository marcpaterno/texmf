\startonderdeel ma-cb-nl-introduction

\produkt ma-cb-nl

\CONTEXT\ is een document produktie systeem gebaseerd op
\TEX. \TEX\ is een typografische programmeertaal \'en een
programma waarmee u documenten kunt vormgeven en produceren.
\CONTEXT\ maakt het werken met \TEX\ uitermate eenvoudig en
stelt u in staat zeer complexe (papieren en elektronische)
documenten te vervaardigen.

Deze handleiding beschrijft de mogelijkheden van \CONTEXT\
en de beschikbare commando's en functionaliteit. \voetnoot
{Alle papieren en elektronische produkten die \CONTEXT\
vergezellen zijn geproduceerd met \CONTEXT. Indien mogelijk
worden de bronteksten van alle handleidingen elektronisch
beschikbaar gesteld. Hierdoor wordt inzicht gegeven in de
wijze waarop \CONTEXT\ kan worden gebruikt.}

\CONTEXT\ is ontwikkeld voor en in de praktijk getest bij de
opmaak en produktie van eenvoudige boeken tot zeer
geavanceerde technische handleidingen of studieboeken in
elektronische of papieren vorm. Deze inleidende handleiding
behandelt de \CONTEXT\ functionaliteit, voor zover die van
belang is voor de toepassing van standaard tekstelementen in
een handleiding of studieboek. \CONTEXT\ kan echter veel
meer en voor gebruikers die meer willen zijn andere
\CONTEXT\ handleidingen en informatiebronnen beschikbaar.

\CONTEXT\ heeft een meertalige interface, zodat gebruikers
in hun eigen taal met \CONTEXT\ kunnen werken. \CONTEXT\ en
deze handleiding is beschikbaar in het nederlands, duits en
engels.

\stoponderdeel
