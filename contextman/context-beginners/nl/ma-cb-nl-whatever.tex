\startonderdeel ma-cb-nl-whatever

\produkt ma-cb-nl

\hoofdstuk{Diversen}

\paragraaf{Een titelpagina}

\index{titelpagina}

\Command{\tex{startstandaardopmaak}}
\Command{\tex{definieeropmaak}}

In het eerste voorbeeld van deze handleiding op
\op{pagina}[invoerfile] heeft u het volgende commando
gezien:

\shortsetup{startopmaak}

Dit commando wordt gebruikt om titelpagina's op te maken.
Kenmerkend voor een titelpagina is namelijk dat de layout
meestal afwijkt van die van het overige document. Met het
commando||paar \type{\start ... \stopstandaardopmaak} kan
binnen de bestaande zetspiegel dan ook een afwijkende pagina
worden opgemaakt.

Een eenvoudige titelpagina kan er als volgt uitzien:

\startbuffer
\startstandaardopmaak
\blanko
\regelrechts{\tfd Hasselt in de 21ste eeuw}
\blanko
\regelrechts{\tfb Een toekomstschets}
\vfill
\regelrechts{\tfa C. van Marle}
\regelrechts{Hasselt, 2001}
\stopstandaardopmaak
\stopbuffer

\typebuffer

In een dubbelzijdig opgemaakt document moet men zich enige
moeite getroosten om de \citeer{achterkant} te bedrukken.

\startbuffer
\startstandaardopmaak[dubbelzijdig=nee]
\blanko
\regelrechts{\tfd Hasselt in de 21ste eeuw}
\blanko
\regelrechts{\tfb Een toekomstschets}
\vfill
\regelrechts{\tfa C. van Marle}
\regelrechts{Hasselt, 2001}
\stopstandaardopmaak
\startstandaardopmaak[pagina=nee]
\vfill
\copyright 2001

Dit boek wordt opgedragen aan de Hasselter bevolking.
Speciale dank gaat uit naar J. Jonker voor het bewerken van
het fotomateriaal, waardoor men zich een nauwkeurig beeld
kan vormen van het toekomstige Hasselt.
\stopstandaardopmaak
\stopbuffer

\typebuffer

Eigen opmaken kunnen worden gedefinieerd en ingesteld met:

\shortsetup{definieeropmaak}

en

\shortsetup{stelopmaakin}

\paragraaf[floatingblocks]{Zwevende blokken / Floating blocks}

\index{floating blocks}
\index{zwevende blokken}

\Command{\tex{definieerplaatsblok}}
\Command{\tex{stelplaatsblokin}}
\Command{\tex{stelplaatsblokkenin}}
\Command{\tex{stelblokkopjesin}}
\Command{\tex{plaatsintermezzo}}

Een blok is in \CONTEXT\ een tekstelement, zoals een tabel
of een figuur die op een speciale manier wordt afgehandeld.
U heeft al het gebruik van \type{\plaatsfiguur} en
\type{\plaatstabel} gezien. Beide zijn voorbeelden van
floating blocks of zwevende blokken. Het floatmechanisme is
beschreven in \in{hoofdstuk}[figuren] en \in[tabellen].

Dergelijke blokken kunnen worden gedefinieerd met:

\shortsetup{definieerplaatsblok}

Tussen de vierkante haken wordt de naam in enkel- en meervoud
vermeld. Bijvoorbeeld:

\starttypen
\defineerplaatsblok[intermezzo][intermezzos]
\stoptypen

Na deze definitie zijn de volgende commando's beschikbaar:

\starttypen
\plaatsintermezzo[][]{}{}
\startintermezzotekst ... \stopintermezzotekst
\plaatslijstmetintermezzos
\volledigelijstmetintermezzos
\stoptypen

Het nieuw gedefinieerde (zwevende) blok kan worden ingesteld
door middel van:

\shortsetup{stelplaatsblokin}

De layout van dergelijke blokken wordt ingesteld met:

\shortsetup{stelplaatsblokkenin}

De nummering en de labels worden ingesteld met:

\shortsetup{stelblokkopjesin}

Deze commando's worden meestal in het instelgebied van de
invoerfile geplaatst en zijn geldig voor alle floating
blocks in het document.

\startbuffer
\stelplaatsblokkenin[plaats=midden]
\stelblokkopjesin[plaats=onder,kopletter=vetschuin]

\plaatsintermezzo{Een intermezzo.}
\startkadertekst
Aan het begin van deze eeuw liep er een tramlijn van Zwolle
naar Blokzijl via Hasselt. Andere vormen van transport
werden belangrijker en net voor de Tweede Wereldoorlog werd
de lijn opgeheven. Tegenwoordig zou zo'n tramlijn best
rendabel kunnen draaien.
\stopkadertekst
\stopbuffer

\typebuffer

\start
\haalbuffer
\stop

\paragraaf[tekstblokken]{Tekstblokken}

\index{tekstblokken}

\Command{\tex{definieerblok}}
\Command{\tex{gebruikblokken}}
\Command{\tex{verbergblokken}}
\Command{\tex{stelblokin}}

Een ander soort blok is het tekstblok. Een tekstblok is
meestal een stuk tekst dat meerdere malen in een document
wordt gebruikt (maar dat u maar eenmaal wilt invoeren).

Een tekstblok wordt gedefinieerd met:

\shortsetup{definieerblok}

De naam van het tekstblok wordt tussen vierkante haken
geplaatst. Het is ook mogelijk een lijst van namen op die
plaats in te voeren. De namen worden gescheiden door komma's.

U kunt bijvoorbeeld het volgende blok defini\"eren:

\starttypen
\definieerblok[nederlands]
\stoptypen

Vervolgens is na deze definitie het volgende commando||paar
beschikbaar:

\starttypen
\beginvannederlands ... \eindvannederlands
\stoptypen

Blokken worden gemanipuleerd met:

\shortsetup{verbergblokken}
\shortsetup{gebruikblokken}
\shortsetup{handhaafblokken}
\shortsetup{selecteerblokken}

Hieronder wordt met een voorbeeld de werking van
tekstblokken ge\"{\i}llustreerd. Tekstblokken worden voornamelijk
gebruikt bij vragen en antwoorden in studieboeken of in
meertalige documenten.

\startbuffer
\definieerblok[nederlands,engels]

\verbergblokken[nederlands,engels]

\beginvanengels[dedemsvaart-e]
Since 1810 the Dedemsvaart caused some prosperity in Hasselt. All
ships went through the canals of Hasselt and the shops on both
sides of the canals prospered.
\eindvanengels

\beginvannederlands[dedemsvaart-n]
Sinds 1810 zorgde de Dedemsvaart voor enige welvaart in Hasselt.
Alle schepen voeren door de grachten en de winkels aan weerszijden
van de gracht floreerden.
\eindvannederlands

\gebruikblokken[engels][dedemsvaart-e]
\stopbuffer

\typebuffer

Dit resulteert in:

\haalbuffer

Indien u dergelijke blokken consequent zou gebruiken kunt u
meertalige documenten maken. Voor dat doel is het dan ook
mogelijk tekstblokken in een aparte externe file op te slaan.
Dat ziet er als volgt uit:

\startbuffer
\stelblokin[nederlands][file=bewaar-n]
\stopbuffer

\typebuffer

De nederlandse tekstblokken worden bewaard in de file
\type{bewaar-n.tex} en de tekstfragmenten kunnen met hun
logische naam worden aangeroepen. Met \type{\stelblokin}
wordt de weergave ingesteld.

\paragraaf{Tekst bufferen (bewaren voor later gebruik)}

\index{tekst bufferen}

\Command{\tex{startbuffer}}
\Command{\tex{haalbuffer}}
\Command{\tex{typebuffer}}
\Command{\tex{stelbufferin}}

Informatie kan tijdelijk worden opgeslagen om later in het
document te worden gebruikt. Deze optie wordt het bufferen
van teksten genoemd.

\shortsetup{startbuffer}

Bijvoorbeeld:

\starttypen
\startbuffer[visite]
Als u wilt weten wat Hasselt u kan bieden, moet u dit stadje maar
eens komen bezoeken. Als u deze handleiding meeneemt, zult u
ongetwijfeld sommige locaties herkennen.
\stopbuffer

\haalbuffer[visite]
\stoptypen

Met \type{\haalbuffer[visite]} wordt de tekst opgeroepen. De
logische naam is optioneel. Met \type{\typebuffer[visite]}
wordt de getypte tekst van de tekstbuffer opgeroepen en
geplaatst.

Buffers worden ingesteld met:

\shortsetup{stelbufferin}

\paragraaf{Tekst verbergen}

\index{tekst verbergen}

\Command{\tex{startverbergen}}

Tekst worden verborgen met:

\shortsetup{startverbergen}

De tekst tussen dit commando||paar wordt niet verwerkt.

\paragraaf{Lijnen}

\index{lijnen}
\index{punten}

\Command{\tex{haarlijn}}
\Command{\tex{starttekstlijn}}
\Command{\tex{dunnelijn}}
\Command{\tex{dunnelijnen}}
\Command{\tex{steldunnelijnenin}}
\Command{\tex{punten}}
\Command{\tex{onderstrepen}}
\Command{\tex{doorstreep}}
\Command{\tex{invullijn}}
\Command{\tex{invulregels}}
\Command{\tex{stelinvullijnenin}}
\Command{\tex{stelinvulregelsin}}

Er zijn vele commando's om lijnen te tekenen. Om een enkele
lijn te trekken, typt u:

\shortsetup{haarlijn}

of:

\shortsetup{dunnelijn}

Meerdere lijnen worden opgeroepen met:

\shortsetup{dunnelijnen}

Tekst en lijnen kunnen ook worden gecombineerd.

\shortsetup{starttekstlijn}

Bijvoorbeeld:

\startbuffer
\starttekstlijn{Hasselt -- Amsterdam}
Als u een rechte lijn trekt van Hasselt naar Amsterdam moet
u een afstand overbruggen van bijna 145 \Kilo \Meter.
\stoptekstlijn

Als u twee rechte lijnen trekt van Hasselt naar Amsterdam
dan overbrugt u een afstand van 290 \Kilo \Meter.

Amsterdam \dunnelijnen[n=3] Hasselt
\stopbuffer

\haalbuffer

Deze voorbeelden zijn als volgt ingevoerd:

\typebuffer

Het tekenen van lijnen verdient altijd extra aandacht. De
witruimte voor en na de lijnen wil nog weleens anders worden
dan in eerste instantie mag worden verwacht.

De afstand tussen lijnen kunt u instellen met:

\shortsetup{steldunnelijnenin}

Er zijn enkele aanvullende commando's:

\shortsetup{stelinvullijnenin}
\shortsetup{stelinvulregelsin}

Deze commando's worden in voorbeelden ge\"{\i}llustreerd:

\startbuffer
\stelinvullijnenin[breedte=2cm]
\stelinvulregelsin[breedte=3cm]

\invullijnen[n=1]{\bf naam}
\invullijnen[n=3]{\bf adres}

\invulregel{Kunt u het \onderstreep{aantal}
huizen in Hasselt noemen?} \par

Streep \doorstrepen{Hasselt door in deze tekst}\punten[18]
\stopbuffer

\typebuffer

Dit wordt na verwerken:

\haalbuffer

Deze commando's zijn ontwikkeld voor vragenlijsten e.d.

Opgemerkt moet worden dat \TEX\ tekst die wordt doorgehaald
met \type{\doorstreep} of \type{\doorstrepen} niet afbreekt.

\paragraaf{Super- en subscript in tekst}

\index{subscript}
\index{superscript}

\Command{\tex{laag}}
\Command{\tex{hoog}}
\Command{\tex{laho}}

\startbuffer
Het is vrij eenvoudig om \hoog{superscript} en
\laag{subscript} in de tekst te plaatsen. Hoe dit
\laho{subscript}{superscript} wordt genoemd, is niet bekend,
maar het ziet er niet uit.
\stopbuffer

\haalbuffer

Deze tekst is gemaakt met \type{\laag{}}, \type{\hoog{}} en
\type{\laho{}{}}. De tekst wordt tussen de accolades
geplaatst.

\paragraaf{Datum}

\index{datum}

\Command{\tex{huidigedatum}}

De systeemdatum kan in uw document worden opgenomen met:

\starttypen
\huidigedatum
\stoptypen

\paragraaf{Positioneren}

\index{positioneren}

\Command{\tex{positioneer}}
\Command{\tex{stelpositionerenin}}

Voor zeer speciale toepassingen is het soms wenselijk tekst
op een pagina te positioneren. Positioneren gebeurt met:

\shortsetup{positioneer}

De haakjes omsluiten de $x,y$||co\"ordinaten, de accolades
bevatten de tekst die moet worden gepositioneerd.

Het $x,y$||stelsel wordt ingesteld met:

\shortsetup{stelpositionerenin}

Bij het instellen kan gebruik gemaakt worden van
schaalfactoren en eenheden. Een voorbeeld licht het commando
\type{\positioneer} verder toe.

\startbuffer
\def\dobbelvijf%
  {\omlijnd
     [breedte=42pt,hoogte=42pt,offset=0pt]
     {\stelpositionerenin
        [eenheid=pt,factor=12,xoffset=-11pt,yoffset=-8pt]%
      \startpositioneren
        \positioneer(1,1){$\bullet$}%
        \positioneer(1,3){$\bullet$}%
        \positioneer(2,2){$\bullet$}%
        \positioneer(3,1){$\bullet$}%
        \positioneer(3,3){$\bullet$}%
      \stoppositioneren}}

\plaatsfiguur{Dit is vijf.}{\dobbelvijf}
\stopbuffer

\typebuffer

Dit toch wel lastige voorbeeld komt er als volgt uit te
zien.

\haalbuffer

\paragraaf{Roteren van tekst, figuren en tabellen}

\index{roteren}

\Command{\tex{roteer}}

In een aantal gevallen is het noodzakelijk om teksten,
figuren of tabellen te roteren. Dergelijke objecten worden
geroteerd met:

\shortsetup{roteer}

De vierkante haken zijn optioneel en worden gebruikt om de
rotatie in te stellen: \type{rotatie=90}. De accolades
bevatten de tekst of het object dat geroteerd moet worden.

\startbuffer
Hasselt heeft haar stadsrechten in 1252 gekregen. Vanaf die
tijd bezat Hasselt het \roteer[rotatie=90]{recht}
om een eigen zegel op offici\"ele documenten te plaatsen. Het
zegel toont de Heilige Stephanus die bekend staat als een van
de eerste christelijke martelaren en als de beschermheilige
van \roteer[rotatie=270]{Hasselt}. Na de Reformatie werd het
zegel opnieuw ontworpen. Bovendien verloor Stephanus zijn
'heiligheid'. Vanaf die dag wordt hij dan ook afgebeeld zonder
aureool.
\stopbuffer

\typebuffer

Dit resulteert in een wel erg lelijke tekst:

\haalbuffer

Ook figuren kunt u roteren met:

\startbuffer
\plaatsfiguur
  [][fig:rotatie]
  {De Vispoort is 180 \Degrees\ geroteerd.}
  \roteer[rotatie=180]
  {\externfiguur[ma-cb-15][breedte=10cm]}
\stopbuffer

\typebuffer

U ziet in \in{figuur}[fig:rotatie] dat roteren een figuur
soms erg onduidelijk kan maken.

\haalbuffer

We kunnen roteren instellen met:

\shortsetup{stelroterenin}

\paragraaf{Nieuwe regel}

\index{nieuwe regel}

\Command{\tex{crlf}}
\Command{\tex{startregels}}

Een nieuwe regel kan worden afgedwongen met:

\shortsetup{crlf}

De afkorting \type{\crlf} staat voor carriage return en
linefeed.

Wanneer meerdere regels onder elkaar moeten worden geplaatst
en moeten worden afgebroken op een door u aangegeven plaats
kunt u dat als volgt doen:

\shortsetup{startregels}

\startbuffer
Op een houten paneel in het stadhuis kan men lezen:

\startregels
Heimelijcken haet
eigen baet
jongen raet
Door diese drie wilt verstaen
is het Roomsche Rijck vergaen.
\stopregels

Dit rijmpje waarschuwt magistraten van Hasselt ervoor dat
persoonlijke voordelen en gevoelens de besluitvorming
niet mogen be\"{\i}nvloeden.
\stopbuffer

\typebuffer

\haalbuffer

In enkele commando's worden regelovergangen gegenereerd met
\type{\\}. Als u het commando \type{\inmarge{in de\\marge}}
intypt wordt de margetekst over twee regels verdeeld.

\paragraaf{Afbrekingen}

\index{afbreking}
\index{taal}

\Command{\tex{hoofdtaal}}
\Command{\tex{taal}}
\Command{\tex{nl}}
\Command{\tex{en}}
\Command{\tex{fr}}
\Command{\tex{de}}
\Command{\tex{sp}}

Bij het schrijven van meertalige teksten dient u er rekening
mee te houden dat afbreekmechanismen per taal kunnen
verschillen.

Een afbreekmechanisme wordt geactiveerd met:

\shortsetup{hoofdtaal}

Tussen de vierkante haken typt u het taalgebied in
\type{nl}, \type{fr}, \type{en}, \type{de} en \type{sp}.

Om over te gaan van de ene taal op de andere kunt u de
volgende schrijfwijze hanteren:

\starttypen
\taal[nl] \taal[en]  \taal[de]  \taal[fr]  \taal[sp]
\stoptypen

of:

\starttypen
\nl  \en  \de  \fr  \sp
\stoptypen

Het voorbeeld hieronder geeft enkele overgangen van talen
weer:

\startbuffer
\en If you want to know more about Hasselt the best book to
read is {\nl \em Uit de geschiedenis van Hasselt} by
F.~Peereboom.
\stopbuffer

\typebuffer

\start
\haalbuffer
\stop

Het afbreekmechanisme van \TEX\ en dus ook van \CONTEXT\
is zeer goed. Indien het voorkomt dat \TEX\ een woord
verkeerd afbreekt, kunt u zelf een afbreekpatroon defini\"eren.
Dergelijke afbreekpatronen worden in het instelgebied van de
invoerfile gedefinieerd met:

\startbuffer
\hyphenation{ge-schie-de-nis}
\stopbuffer

\typebuffer

\paragraaf[externe texfiles]{Invoer van andere {\tt tex}||files}

\index{invoer van \TEX--files}

\Command{\tex{input}}

Informatie kan in meerdere \TEX||files worden ondergebracht
om vervolgens op de juiste plaats in de invoerfile te worden
geladen. Het kan bijvoorbeeld effici\"enter zijn om een
document op te splitsen in meerdere files, zodat partieel
verwerken mogelijk wordt.

Een andere \TEX||file (met de naam \type{eenfile.tex}) kan
in de invoerfile worden geladen met:

\type{\input eenfile.tex}

De extensie is optioneel, dus werkt dit ook:

\type{\input eenfile}

Het commando \type{\input} is een \TEX||commando.

\paragraaf{Commentaar in de invoerfile}

\index{commentaar}
\index[procent]{\% in input file}

Alle tekst tussen \type{\starttekst} en  \type{\stoptekst}
wordt tijdens de verwerking met \CONTEXT\ meegenomen en
gezet. Het kan echter zijn dat u tekstfragmenten wel wilt
bewaren, maar niet wilt laten verwerken. Ook kan het zijn dat
u uw opmaak wilt voorzien van commentaar.

Alle tekst die wordt voorafgegaan door een \type{%}||teken
wordt door \CONTEXT\ gezien als commentaar en wordt niet
verwerkt.

\startbuffer
% In grote documenten kunt u de verschillende onderdelen
% onderbrengen in meerdere files.
%
% Bijvoorbeeld:
%
% \input hass01.tex  % hoofdstuk 1 over Hasselt
% \input hass02.tex  % hoofdstuk 2 over Hasselt
% \input hass03.tex  % hoofdstuk 3 over Hasselt
\stopbuffer

\typebuffer

Als u de \type{%} zou weghalen voor de
\type{\input}||commando's dan worden de drie files geladen
en op die plek in het document geplaatst. Het commentaar dat
de inhoud van de files beschrijft wordt echter niet
meegenomen.

\stoponderdeel
