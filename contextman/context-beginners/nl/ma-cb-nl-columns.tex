\startonderdeel ma-cb-nl-columns

\produkt ma-cb-nl

\hoofdstuk{Kolommen}

\index{kolommen}

\Command{\tex{startkolommen}}
\Command{\tex{stelkolommenin}}
\Command{\tex{kolom}}

Eenvoudige recht||toe recht||aan tekst kan in kolommen worden
gezet. Als u het tekstfragment vooraf laat gaan door
\type{\startkolommen} en afsluit met \type{\stopkolommen}
wordt alles er tussenin in het opgegeven aantal kolommen
gezet.

\shortsetup{startkolommen}

\startbuffer
\startkolommen[n=3,tolerantie=zeersoepel]
Hasselt is een oude Hanzestad ongeveer 12 \Kilo \Meter\ ten
noorden van Zwolle aan de rivier het Zwartewater.

De stad heeft een lange historie sinds het verkrijgen van de
stadrechten in 1252. Herinneringen aan die rijke historie
zijn terug te vinden in de monumenten die nog steeds het
gezicht van het centrum van Hasselt bepalen.

De Sint Stephanuskerk domineert het aanzien van Hasselt. Het
is een laat||gothische kerk die dateert uit 1479 met een
magnifiek orgel. Het voormalige stadhuis staat op de Markt.
Gebouwd tussen 1500 en 1550 herbergt het een grote collectie
wapens, waaronder de grootste collectie haakbussen in de
wereld.

Verder is er een korenmolen 'De Zwaluw' te vinden die
dateert uit 1748 en de 'Stenendijk', een unieke gemetselde
dijk. Ook kunt u de nog steeds werkende kalkovens
bezichtigen.

Het stadscentrum met de vestingwerken, het Van Stolkspark en
de druk bezochte kades nodigen uit voor een wandeling.

Het gebied rond Hasselt moet ook worden genoemd. In de
winter huizen in de nabijgelegen polder Mastenbroek
duizenden ganzen. In de zomer groeit aan de oevers van het
Zwartewater de zeldzame Kievitsbloem. Als u die gaat
bekijken, komt u langs plaatsjes als Genne, Streukel en
Cellemuiden. Ideaal voor een wandel- of fietstocht.

Hasselt is ook een belangrijk watersportcentrum. De meren
van noord||west Overijssel, de IJssel, de Overijsselse Vecht
en de randmeren zijn goed bereikbaar vanuit de jachthaven
'De Molenwaard'. Zeilen, vissen, zwemmen en kano\"en zijn
sporten die in Hasselt goed kunnen worden beoefend.

Jaarlijks keren bepaalde festiviteiten terug. Zo is er aan
het eind van de maand augustus het Eui||festival.
\stopkolommen
\stopbuffer

\typebuffer

%\pagina[voorkeur] % zou niet moeten

Het resultaat is een drie||koloms tekst. Ten behoeve van het
uitlijnen wordt het afbreekmechanisme ingesteld met
\type{tolerantie=zeersoepel}.

{\switchnaarkorps[9pt]\haalbuffer}

Indien mogelijk kan een kolom worden afgedwongen met
\type{\kolom}. Het instellen van kolommen gebeurt met:

\shortsetup{stelkolommenin}

In de meeste gevallen krijgt u een beter resultaat
als u de tekst op het \citeer{grid} laat typesetten. Dit
gebeurt door \type{grid=ja} in te stellen bij het commando
\type {\stellayoutin}.

\stoponderdeel
