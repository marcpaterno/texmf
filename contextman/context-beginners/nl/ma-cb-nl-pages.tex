\startonderdeel ma-cb-nl-pages

\produkt ma-cb-nl

\hoofdstuk{Pagina's en paginanummering}

\index{paginanummering}

\Command{\tex{pagina}}
\Command{\tex{stelnummeringin}}
\Command{\tex{startuitstellen}}

Een pagina||overgang wordt afgedwongen of geblokkeerd met:

\shortsetup{pagina}

De eventuele opties worden tussen vierkante haakjes
geplaatst. De opties worden in \in{tabel}[tab:pagina opties]
toegelicht.

\plaatstabel
  []
  [tab:pagina opties]
  {Pagina opties.}
\starttabel[|l|l|]
\HL
\NC \bf Optie \NC \bf Betekenis \NC\SR
\HL
\NC \type{ja}            \NC gedwongen pagina--overgang \NC\FR
\NC \type{opmaak}        \NC gedwongen pagina--overgang zonder opvullen \NC\MR
\NC \type{nee}           \NC geen pagina--overgang \NC\MR
\NC \type{voorkeur}      \NC bij voorkeur hier een pagina--overgang \NC\MR
\NC \type{grotevoorkeur} \NC bij grote voorkeur hier een pagina--overgang \NC\MR
\NC \type{links}         \NC volgende pagina is een linker pagina \NC\MR
\NC \type{rechts}        \NC volgende pagina is een rechter pagina  \NC\MR
\NC \type{blokkeer}      \NC volgende commando geen effect \NC\MR
\NC \type{reset}         \NC volgende commando heeft effect \NC\MR
\NC \type{leeg}          \NC voeg een lege pagina toe \NC\MR
\NC \type{laatste}       \NC vul pagina's aan tot een even nummer \NC\MR
\NC \type{viertal}       \NC vul pagina's aan tot een viertal is bereikt \NC\LR
\HL
\stoptabel

Paginanummering vindt automatisch plaats, maar nummers
kunnen worden afgedwongen met:

\starttypen
\pagina[25]
\stoptypen

Soms is het beter om een relatief paginanummer in te voeren
(indien vooraf niet bekend is welk paginanummer de laatste
pagina heeft) \type{[+2]} of \type{[-2]}.

De positie van de paginanummers op een pagina hangen af van
uw eigen voorkeur en of het gaat om een enkel- of
dubbelzijdig document. Het nummeren van pagina's wordt
ingesteld met:

\shortsetup{stelnummeringin}

De voorkeuren worden tussen de vierkante haakjes geplaatst.

Tabellen en figuren nemen nogal wat ruimte in op een pagina.
Het plaatsen van dergelijke tekstelementen kan tijdelijk
worden uitgesteld, zodat u zelf kunt bepalen waar de figuren
en tabellen worden geplaatst. Dit wordt gedaan met:

\shortsetup{startuitstellen}

Als u bijvoorbeeld typt:

\startbuffer
\startuitstellen
\plaatsfiguur
  {Een uitgesteld figuur.}
  {\externfiguur[ma-cb-16][breedte=\tekstbreedte]}
\stopuitstellen
\stopbuffer

\typebuffer

De figuur wordt bovenaan de volgende pagina geplaatst,
waardoor de lopende tekst zo min mogelijk wordt verstoord.

\haalbuffer

\stoponderdeel
