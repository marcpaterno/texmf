\startonderdeel ma-cb-nl-frames

\produkt ma-cb-nl

\hoofdstuk{Omlijnde tekst}

\index{omlijnd+tekst}

\Command{\tex{omlijnd}}
\Command{\tex{stelomlijnenin}}
\Command{\tex{inlijnd}}

Tekst kan met \type{\omlijnd} worden \inlijnd{omlijnd}. Het
commando ziet er als volgt uit:

\shortsetup{omlijnd}

De vierkante haken zijn optioneel en bevatten de
instellingen. De accolades bevatten de te omlijnen tekst.

\startbuffer
\omlijnd[hoogte=3em,breedte=passend]{Hasselt krijgt de ruimte}
\stopbuffer

\typebuffer

Dit wordt:

\startregelcorrectie
\haalbuffer
\stopregelcorrectie

Een ander voorbeeld van \type{\omlijnd} met de instellingen
wordt hieronder getoond. We gebruiken hier overigens een
variant die rekening houdt met de regeldiepte:
\type{\inlijnd}.

\startbuffer
\regellinks
  {\inlijnd[breedte=passend]{Inwoners van Hasselt}}
\regelmidden
  {\inlijnd[hoogte=1.5cm,kader=uit]{hebben een}}
\regelrechts
  {\inlijnd[achtergrond=raster]{historische achtergrond}}
\stopbuffer

\typebuffer

Dit leidt tot:

\startregelcorrectie
\haalbuffer
\stopregelcorrectie

Het \type{\omlijnd}||commando is zeer geavanceerd en wordt
in zeer veel macro's ingezet. Het commando wordt ingesteld
met:

\shortsetup{stelomlijndin}

\stoponderdeel
