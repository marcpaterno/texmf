\startcomponent ma-cb-zh-document

\product ma-cb-zh

\chapter{如何創建文檔}

\index[shuruwenjian]{輸入文件}

假設你要創建一個簡單的文檔。它會有結構,有標題頁、幾個章、節和小節。
當然還有目錄和索引。

如果你能提供一個恰當的輸入文件,\CONTEXT\ 就可以自動創建這樣一個文檔。
所以,你首先必須創建一個輸入文件。輸入文件由文件名和擴展名組成。
你可以隨意取個文件名,不過擴展名一定要是\type{tex}。如果你創建
了一個名為\type{myfile.tex} 的文件,你將發現運行\CONTEXT\ 毫無
困難。

\pagereference[inputfile]輸入文件的形式如下:

\startbuffer
\starttext

\startstandardmakeup
  \midaligned{如何製作一份文檔。}
  \midaligned{作者}
  \midaligned{無名氏}
\stopstandardmakeup

\completecontent

\chapter{導言}

……這兒輸入你的文本\index{索引條目}……

\chapter{某一章}

\section[firstsection]{第一節}

……這兒輸入你的文本……

\section{第二節}

\subsection{第一小節}

……這兒輸入你的文本\index{另一個索引條目}……

\subsection{第二小節}

……這兒輸入你的文本……

\section{第三節}

……這兒輸入你的文本……

\chapter{又一章}

……這兒輸入你的文本……

\chapter[lastchapter]{最後一章}

……這兒輸入你的文本……

\completeindex

\stoptext
\stopbuffer

{\switchtobodyfont[9pt]\typebuffer}

\CONTEXT\ 讀入的輸入文件是無格式的\ASCII\ 文件。當然,你可以
用任何文本編輯器或者文字處理器來編輯,但是你必須要牢記,\CONTEXT\
衹認識\ASCII。大多數的文本編輯器或文字處理器都能夠導出無格式
的\ASCII\ 文件。

輸入文件應該包含你想讓\CONTEXT\ 和\CONTEXT\ 命令處理的文本
內容。\CONTEXT\ 命令是以反斜杠\tex{} 開頭的。\type{\starttext} 命令
表示正文開始了。在\type{\starttext} 之前的部分稱做設置區,
用來定義新命令和設置文檔的樣式。

命令後通常會跟有方括號對\type{[]} 或花括號對\type{{}},
或兼而有之。如{\processingverbatimtrue\type{\chapter[lastchapter]{最後一章}}}
這條命令中,\type{\chapter} 就告訴\CONTEXT\ 執行一些關於綱要、樣式以及結構
的操作。比如:

\startitemize[n,packed]
\item 換至新頁面
\item 章序號加一
\item 將章序號放至章標題前面
\item 留出些許豎直空隙
\item 切換至大號字體
\item 將章標題(及頁碼)放入目錄
\stopitemize

這一系列操作將作用於花括號之間的參數:{\em 最後一章}。

在方括號中的\type{[lastchapter]} 尚未提及。這是一個標簽,
可以用來引用指定的章。我們可以用些其他的\CONTEXT\ 命令來調用:
比如{\processingverbatimtrue\type{\in{第}{章}[lastchapter]}}
就可以顯示該章的序號,而\type{\about[lastchapter]} 可以獲得該
章的標題。

於是,上面的操作列表就有所擴充:

\startitemize[continue]
\item 將標簽\type{lastchapter} 設為章的序號和標題(並存儲起來
      以備日後使用)
\stopitemize

實際上還有些關於題頭、重置計數器和交互性沒有現在就談及。

如果你讓\CONTEXT\ 處理這個示例文件的話,你就可以得到一份很簡單的
文檔,其中有些標號的章節標題。

\CONTEXT\ 在處理該文件時顧及許多細節。例如其中之一就是頁碼
編號。不過\CONTEXT\ 首次運行的時候,它並不知道生成的目錄中
需要的頁碼是多少。因此你需要再度處理該文件。\CONTEXT\ 會生
成一些輔助文件來存儲此類信息。某些情況下,你必須三度編譯該輸入文件。
你可以在命令行下用\TEXEXEC\ 來運行\CONTEXT。這是個\RUBY\ 腳本,它能夠知曉
需要編譯多少次才能獲得最終結果。\TEXEXEC\ 是標準\CONTEXT\
套件的一部分。

\stopcomponent
