\startcomponent ma-cb-zh-columns

\product ma-cb-zh

\chapter{多欄排版}

\index{多欄}

\Command{\tex{startcolums}}
\Command{\tex{setupcolumns}}
\Command{\tex{column}}

我們可以將樸素簡潔的文檔排成多欄。如果你在一段文本片斷的開始之前輸入%
\type{\startcolumns},然後以 \type{\stopcolumns} 結束這段文本,
那麼這之間的所有内容將以多欄排版出來。

\shortsetup{startcolumns}

\startbuffer
\startcolumns[n=3,tolerance=verytolerant]
六王畢,四海一。蜀山兀,阿房出。覆壓三百餘里,隔離天日。驪山北構而西折,
直走咸陽。二川溶溶,流入宮牆。五步一樓,十步一閣。廊腰縵迴,簷牙高啄。
各抱地勢,鉤心鬥角。盤盤焉,囷囷焉,蜂房水渦,矗不知乎幾千萬落。長橋臥
波,未雲何龍?複道行空,不霽何虹?高低冥迷,不知西東。歌臺暖響,春光融
融。舞殿冷袖,風雨淒淒。一日之內,一宮之間,而氣候不齊。

妃嬪媵嬙,王子皇孫,辭樓下殿,輦來於秦。朝歌夜絃,為秦宮人。明星熒熒。
開妝鏡也;綠雲擾擾,梳曉鬟也。渭流漲膩,棄脂水也;煙斜霧橫,焚椒蘭也;
雷霆乍驚,宮車過也;轆轆遠聽,杳不知其所之也。一肌一容,盡態極姘;縵立
遠視,而望幸焉。有不得見者三十六年。

燕、趙之收藏,韓、魏之經營,齊、楚之精英,幾世幾年,剽掠其人,倚疊如山。
一旦不能有,輸來其閒。鼎鐺玉石,金塊珠礫,棄擲邐迤。秦人視之,亦不甚惜。

嗟乎!一人之心,千萬人之心也。秦愛紛奢,人亦念其家。奈何取之盡錙銖,用
之如泥沙!使負棟之柱,多於南畝之農夫;架梁之椽,多於機上之工女;釘頭磷
磷,多於在庾之粟粒;瓦縫參差,多於周身之帛縷;直欄橫檻,多於九土之城郭;
管絃嘔啞,多於市人之言語。使天下之人,不敢言而敢怒。獨夫之心,日益驕固。
戍卒叫,函谷舉。楚人一炬,可憐焦土。

嗚呼!滅六國者,六國也,非秦也;族秦者,秦也,非天下也。嗟夫!使六國各
愛其人,則足以拒秦;秦復愛六國之人,則遞三世可至萬世而為君,誰得而族滅
也。秦人不暇自哀,而後人哀之;後人哀之,而不鑑之,亦使後人而復哀後人也。
\stopcolumns
\stopbuffer

\typebuffer

結果文本將分爲三欄。

{\switchtobodyfont[9pt]\getbuffer}

如果想的話,可以用\type{\column} 來強行換一欄。你可以使用下面的命令來對欄設置:

\shortsetup{setupcolumns}

大多數情況下,如果你設置讓文本排列在“柵格”上,顯示結果會較好。我們可以在%
\type{\setuplayout} 命令中輸入參數\type{grid=yes} 來實現。

\stopcomponent
