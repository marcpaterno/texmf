% interface=nl

\starttext

% TODO: translate this stuff

\iffalse

\unprotect

\setuplayout
  [\c!kopwit=4cm,
   \c!hoogte=22cm,
   \c!hoofd=1.5cm,
   \c!voet=1cm,
   \c!bovenafstand=6pt,
   \c!hoofdafstand=12pt,
   \c!voetafstand=12pt,
   \c!onderafstand=6pt,
   \c!rugwit=5cm,
   \c!breedte=10.5cm,
   \c!linkermarge=0.75cm,
   \c!rechtermarge=1.5cm,
   \c!linkerrand=2cm,
   \c!rechterrand=1.75cm,
   \c!linkermargeafstand=18pt,
   \c!rechterrandafstand=6pt,
   \c!linkerrandafstand=12pt,
   \c!rechterrandafstand=12pt]

\stelboventekstenin [\v!rand]  [\tttf LLL] [\tttf RRR]
\stelboventekstenin [\v!marge] [\tttf LL]  [\tttf RR]
\stelboventekstenin [\v!tekst] [\tttf L]   [\tttf R]

\stelhoofdtekstenin [\v!rand]  [\tttf LLL] [\tttf RRR]
\stelhoofdtekstenin [\v!marge] [\tttf LL]  [\tttf RR]
\stelhoofdtekstenin [\v!tekst] [\tttf L]   [\tttf R]

\stelvoettekstenin  [\v!rand]  [\tttf LLL] [\tttf RRR]
\stelvoettekstenin  [\v!marge] [\tttf LL]  [\tttf RR]
\stelvoettekstenin  [\v!tekst] [\tttf L]   [\tttf R]

\stelondertekstenin [\v!rand]  [\tttf LLL] [\tttf RRR]
\stelondertekstenin [\v!marge] [\tttf LL]  [\tttf RR]
\stelondertekstenin [\v!tekst] [\tttf L]   [\tttf R]

\definieerbeeldmerk 
  [test oa] [\v!onder] [\v!linkerrand]   
  [\c!commando=\omlijnd{\type{lll}},\c!status=\v!start]
\definieerbeeldmerk 
  [test ob] [\v!onder] [\v!linkermarge]  
  [\c!commando=\omlijnd{\type{ll}},\c!status=\v!start]
\definieerbeeldmerk 
  [test oc] [\v!onder] [\v!links]        
  [\c!commando=\omlijnd{\type{l}},\c!status=\v!start]
\definieerbeeldmerk 
  [test od] [\v!onder] [\v!midden]       
  [\c!commando=\omlijnd{\type{m}},\c!status=\v!start]
\definieerbeeldmerk 
  [test oe] [\v!onder] [\v!rechts]       
  [\c!commando=\omlijnd{\type{r}},\c!status=\v!start]
\definieerbeeldmerk 
  [test of] [\v!onder] [\v!rechtermarge] 
  [\c!commando=\omlijnd{\type{rr}},\c!status=\v!start]
\definieerbeeldmerk 
  [test og] [\v!onder] [\v!rechterrand]  
  [\c!commando=\omlijnd{\type{rrr}},\c!status=\v!start]

\definieerbeeldmerk 
  [test ba] [\v!boven] [\v!linkerrand]   
  [\c!commando=\omlijnd{\type{lll}},\c!status=\v!start]
\definieerbeeldmerk 
  [test bb] [\v!boven] [\v!linkermarge]  
  [\c!commando=\omlijnd{\type{ll}},\c!status=\v!start]
\definieerbeeldmerk 
  [test bc] [\v!boven] [\v!links]        
  [\c!commando=\omlijnd{\type{l}},\c!status=\v!start]
\definieerbeeldmerk 
  [test bd] [\v!boven] [\v!midden]       
  [\c!commando=\omlijnd{\type{m}},\c!status=\v!start]
\definieerbeeldmerk 
  [test be] [\v!boven] [\v!rechts]       
  [\c!commando=\omlijnd{\type{r}},\c!status=\v!start]
\definieerbeeldmerk 
  [test bf] [\v!boven] [\v!rechtermarge] 
  [\c!commando=\omlijnd{\type{rr}},\c!status=\v!start]
\definieerbeeldmerk 
  [test bg] [\v!boven] [\v!rechterrand]  
  [\c!commando=\omlijnd{\type{rrr}},\c!status=\v!start]

\definieerbeeldmerk 
  [test va] [\v!voet]  [\v!linkerrand]   
  [\c!commando=\omlijnd{\type{lll}},\c!status=\v!start]
\definieerbeeldmerk 
  [test vb] [\v!voet]  [\v!linkermarge]  
  [\c!commando=\omlijnd{\type{ll}},\c!status=\v!start]
\definieerbeeldmerk 
  [test vc] [\v!voet]  [\v!links]        
  [\c!commando=\omlijnd{\type{l}},\c!status=\v!start]
\definieerbeeldmerk 
  [test vd] [\v!voet]  [\v!midden]       
  [\c!commando=\omlijnd{\type{m}},\c!status=\v!start]
\definieerbeeldmerk 
  [test ve] [\v!voet]  [\v!rechts]       
  [\c!commando=\omlijnd{\type{r}},\c!status=\v!start]
\definieerbeeldmerk 
  [test vf] [\v!voet]  [\v!rechtermarge] 
  [\c!commando=\omlijnd{\type{rr}},\c!status=\v!start]
\definieerbeeldmerk 
  [test vg] [\v!voet]  [\v!rechterrand]  
  [\c!commando=\omlijnd{\type{rrr}},\c!status=\v!start]

\definieerbeeldmerk 
  [test ha] [\v!hoofd] [\v!linkerrand]   
  [\c!commando=\omlijnd{\type{lll}},\c!status=\v!start]
\definieerbeeldmerk 
  [test hb] [\v!hoofd] [\v!linkermarge]  
  [\c!commando=\omlijnd{\type{ll}},\c!status=\v!start]
\definieerbeeldmerk 
  [test hc] [\v!hoofd] [\v!links]        
  [\c!commando=\omlijnd{\type{l}},\c!status=\v!start]
\definieerbeeldmerk 
  [test hd] [\v!hoofd] [\v!midden]       
  [\c!commando=\omlijnd{\type{m}},\c!status=\v!start]
\definieerbeeldmerk 
  [test he] [\v!hoofd] [\v!rechts]       
  [\c!commando=\omlijnd{\type{r}},\c!status=\v!start]
\definieerbeeldmerk 
  [test hf] [\v!hoofd] [\v!rechtermarge] 
  [\c!commando=\omlijnd{\type{rr}},\c!status=\v!start]
\definieerbeeldmerk 
  [test hg] [\v!hoofd] [\v!rechterrand]  
  [\c!commando=\omlijnd{\type{rrr}},\c!status=\v!start]

\stelachtergrondenin
  [\v!boven,\v!hoofd,\v!tekst,\v!voet,\v!onder]
  [\v!linkerrand,\v!linkermarge,\v!tekst,\v!rechtermarge,\v!rechterrand]
  [\c!achtergrond=\v!raster,
   \c!raster=.875]

\stelachtergrondenin
  [\v!pagina]
  [\c!achtergrond=\v!raster,
   \c!raster=.95,
   \c!offset=0pt,
   \c!diepte=0pt]

\protect

\fi

\placelogos

\showframe

\switchtobodyfont[8pt]

Here we show the complete layout, expressed in frames and
units. The settings can be generated with \type
{\showsetups}. The settings in this exampe are different
from those used in this manual. 

\startlinecorrection
\showsetups
\stoplinecorrection

The \type {\textheight} is calculated by subtracting the
footer and header dimensions from the height. The
multi||digit precision is due to the fact that we store the
dimensions in internal registers and thereby are bound by
\TEX's internal precission. 

\stoptext
