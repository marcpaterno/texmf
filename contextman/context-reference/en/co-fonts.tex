\startcomponent co-fonts
% we assume that there are fontfiles demofont.afm/pfb
%
% runtools mfonts.rb

% \loadmapfile[texnansi-test-test.map]

\environment contextref-env
\product contextref

% some typescript examples

\startbuffer[funny]
\definetypeface [funny] [rm] [serif] [palatino] [default] [encoding=texnansi]
\definetypeface [funny] [ss] [sans]  [palatino] [default] [encoding=texnansi]
\definetypeface [funny] [tt] [mono]  [palatino] [default] [encoding=texnansi]
\definetypeface [funny] [mm] [math]  [palatino] [default] [encoding=texnansi]
\stopbuffer

\startbuffer[joke]
\definetypeface [joke] [rm] [serif] [times]     [default] [encoding=texnansi]
\definetypeface [joke] [ss] [sans]  [helvetica] [default] [rscale=.9,
                                                           encoding=texnansi]
\definetypeface [joke] [tt] [mono]  [courier]   [default] [rscale=1.1,
                                                           encoding=texnansi]
\definetypeface [joke] [mm] [math]  [times]     [default] [encoding=texnansi]
\stopbuffer

\startbuffer[nojoke]
\definetypeface [nojoke] [rm] [serif] [times]     [default] [encoding=texnansi]
\definetypeface [nojoke] [ss] [sans]  [helvetica] [default] [encoding=texnansi]
\definetypeface [nojoke] [tt] [mono]  [courier]   [default] [encoding=texnansi]
\definetypeface [nojoke] [mm] [math]  [times]     [default] [encoding=texnansi]
\stopbuffer

\startbuffer[whow]
\definetypeface [whow] [rm] [serif] [modern] [latin-modern] [encoding=ec]
\definetypeface [whow] [ss] [sans]  [modern] [latin-modern] [encoding=ec]
\definetypeface [whow] [tt] [mono]  [modern] [latin-modern] [encoding=ec]
\definetypeface [whow] [mm] [math]  [modern] [latin-modern] [encoding=ec]
\stopbuffer

\getbuffer[funny,nojoke,joke,whow]

\chapter[fonts]{Fonts}
\index{font+definition}
\section{Introduction}
\index{fonts}

This chapter will cover the details of defining fonts and collections of fonts, and it
will explain how to go about installing fonts in both \MKII\ and \MKIV. It helps if
you know what a font is, and are familiar with the \CONTEXT\ font switching macros.

The original \CONTEXT\ font model was based on plain \TEX, but evolved into a more
extensive one primarily aimed at consistently typesetting \PRAGMA's educational
documents. The fact that pseudo caps had to be typeset in any font shape in the
running text as well as superscripts, has clearly determined the design. The font
model has been relatively stable since 1995.

Currently there are three layers of font definitions:

\startitemize

\item  simple font definitions: such definitions provide
       \type {\named} access to a specific font in a
       predefined size

\item  body font definitions: these result in a coherent
       set of fonts, often from a same type foundry or 
       designer, that can be used intermixed as a \quote{style}

\item  typescript definitions: these package serif, sans serif,
       mono spaced and math and other styles in such a way
       that you can conveniently switch between different
       combinations

\stopitemize

These three mechanisms are actually build on top of each
other and all rely on a low level mapping mechanism
that is responsible for resolving the real font file name
and the specific font encoding used.

When \TEX\ users install one of the \TEX\ distributions, like \TEX-live,
automatically a lot of fonts will be installed on their system.
Unfortunately it is not that easy to get a clear picture of what fonts
are there and what is needed to use them.  And although the \type {texmf}
tree is prepared for commercial fonts, adding newly bought fonts is not
trivial. To compensate this, \CONTEXT\ \MKII\ comes with
\type{texfont.pl}, a program that can install fonts for you. And if the
global setup is done correctly, \MKIV\ and \XETEX\ can use the fonts
installed in your operation system without the need for extra
installation work.

\section {Font files and synonyms}
\macro{\tex{definefontsynonym}}


In \CONTEXT, whenever possible you should define symbolic names for fonts.
The mapping from such symbolic names onto real font names can be done such
that it takes place unnoticed for the user. This is good since the name
depends on the encoding and therefore not seldom is obscure and hard to
remember. The trick is knowing how to use the \type{\definefontsynonym}
command.

\placefloat[][]{none}{\showsetup{definefontsynonym}}

The first argument is the synonym that is being defined or redefined.
Redefinition is not only allowed but often very useful. The second
argument is the replacement of the synonym.  This replacement can be a
real font name, but it can also be another synonym.  The optional third
argument can be used for to specify font settings.

There is no limit on the number of indirection levels, but the last one
in the chain has to be a valid font name. \CONTEXT\ knows it has reached
the bottom level when there is no longer any replacement possible. 

Font settings actually take place at the bottom level, since they are
closely related to specific instances of fonts.  Any settings that are
defined higher up in the chain perculate down, unless they are already
defined at the lower level.

\startbuffer[synonymargs]
\vskip 1ex
\starttabulate[|lT|p|]
\NC encoding \NC The font file encoding for \TFM-based (\MKII) fonts.\NC \NR
\NC handling \NC The font handling for \MKII\ (see  previous chapter).\NC \NR
\NC features \NC The font handling for \MKIV\ and \XETEX\ (see previous chapter).\NC \NR
\NC mapping  \NC letter case change mapping for \MKII\ that may be used in special
                 cases; never actually used in the \CONTEXT\ core. 
                 See~\in{chapter}[languages] on languages for details.\NC \NR
\stoptabulate
\vskip 1ex
\stopbuffer

\getbuffer[synonymargs]

Here is an example of the use of font synonyms:

\startexample
\starttyping
\definefontsynonym [Palatino] [uplr8t]   [encoding=ec]
\stoptyping
\stopexample

In this example, the argumnet \type{uplr8t} is the real font (the actual
file name is \type{uplr8t.tfm}, but file extensions are normally
omitted), and it contains the metrics for the Type~1 font URW Palladio L
in EC encoding.  From now on, the name \type{Palatino} can be used in
further font definitions to identify this font, instead of the
dreadfully low||level (and hard to remember) name \type{uplr8t} and its
accompanying encoding.

A note on font names: In \PDFTEX, the real font is the name of the \TEX\ metrics file,
minus the extension, as we saw already. In \XETEX\ and \MKIV\ a font name is a bit
more complex, because in both cases OpenType fonts can be accessed directly by their
official font name (but with any embedded spaces stripped out) as well as via the disk
file name.

In these two systems, \CONTEXT\ first attempts to find the font using the
official font name. If that doesn't work, then it tries to use the font by
file name as a fallback. Since this is not very efficient and also because
it may generate |<|harmless, but alarming looking|>| warnings it is possible to
force \CONTEXT\ into one or the other mode by using a prefix, so you will
most often see synonym definitions like this:

\startexample
\starttyping
\definefontsynonym [MSTimes]       [name:TimesNewRoman] [features=default]
\definefontsynonym [Iwona-Regular] [file:Iwona-Regular] [features=default]
\stoptyping
\stopexample

In \XETEX, the \type{file} prefix implies that \XETEX\ will search for an
OpenType font (with extension \type{otf} or \type{ttf}) and if that fails
it will try to find a \TEX\ font (with extension \type{tfm}). In \MKIV, the
list is a little longer: OpenType (\type{otf}, \type{ttf}), Type~1
(\type{afm}), \OMEGA\ (\type{ofm}), and finally \TEX\ (\type{tfm}).

The use of aliases to hide the complexity of true font names is already
very useful, but \CONTEXT\ goes further than that.  An extra synonym level
is normally defined that attaches this font name to a generic name like
\type{Serif} or \type{Sans}.

\startexample
\starttyping
\definefontsynonym [Serif] [Palatino]
\stoptyping
\stopexample

An important advantage of using names like \type {Serif} in macro and
style definitions is that it can easily be remapped onto a completely
different font than \type{Palatino}. This is often useful when you are
experimenting with a new environment file for a book or when you are
writing a \CONTEXT\ module.

In fact, inside an environment file it is useful to go even further and
define new symbolic names that map onto \type {Serif}.

\starttyping
\definefontsynonym [TitleFont] [Serif]
\stoptyping

By using symbolic names in the main document and in style and macro
definitions, you can make them independent of a particular font and let
them adapt automatically to the main document fonts. That is of course
assuming these are indeed defined in terms of \type{Serif}, \type{Sans},
etcetera. All the \CONTEXT\ predefined typescripts are set up this way, and
you are very much encouraged to stick to the same logic for your own font
definitions as well.

The list of \quote{standard} symbolic names is given
in~\in{table}[tab:symbolic font names]

\startbuffer[symfontnames]
\vskip 1ex
\starttabulate[|lT|lT|l|]
\NC \rm\bf name          \NC \rm \bf style, alternative \NC \rm \bf explanation \NC \NR
\NC Blackboard           \NC --       \NC Used by the \type{\bbd}  macro \NC \NR
\NC Calligraphic         \NC --       \NC Used by the \type{\cal}  macro \NC \NR
\NC Fraktur              \NC --       \NC Used by the \type{\frak} macro \NC \NR
\NC Gothic               \NC --       \NC Used by the \type{\goth} macro \NC \NR
\NC OldStyle             \NC --       \NC Used by the \type{\os} macro   \NC \NR
\NC MPtxtfont            \NC --       \NC The default font for \METAPOST \NC \NR
\NC Calligraphy          \NC cg,tf    \NC \NC \NR
\NC Handwriting          \NC hw,tf    \NC \NC \NR
\NC MathRoman(Bold)      \NC mm,mr(bf)\NC \NC \NR
\NC MathItalic(Bold)     \NC mm,mi(bf)\NC \NC \NR
\NC MathSymbol(Bold)     \NC mm,sy(bf)\NC \NC \NR
\NC MathExtension(Bold)  \NC mm,ex(bf)\NC \NC \NR
\NC MathAlpha(Bold)      \NC mm,ma(bf)\NC \NC \NR
\NC MathBeta(Bold)       \NC mm,mb(bf)\NC \NC \NR
\NC MathGamma(Bold)      \NC mm,mc(bf)\NC \NC \NR
\NC MathDelta(Bold)      \NC mm,md(bf)\NC \NC \NR
\NC Mono                 \NC tt,tf    \NC \NC \NR
\NC MonoBold             \NC tt,bf    \NC \NC \NR
\NC MonoItalic           \NC tt,it    \NC \NC \NR
\NC MonoBoldItalic       \NC tt,bi    \NC \NC \NR
\NC MonoSlanted          \NC tt,sl    \NC \NC \NR
\NC MonoBoldSlanted      \NC tt,bs    \NC \NC \NR
\NC MonoCaps             \NC tt,sc    \NC \NC \NR
\NC Sans                 \NC ss,tf    \NC \NC \NR
\NC SansBold             \NC ss,bf    \NC \NC \NR
\NC SansItalic           \NC ss,it    \NC \NC \NR
\NC SansBoldItalic       \NC ss,bi    \NC \NC \NR
\NC SansSlanted          \NC ss,sl    \NC \NC \NR
\NC SansBoldSlanted      \NC ss,bs    \NC \NC \NR
\NC SansCaps             \NC ss,sc    \NC \NC \NR
\NC Serif                \NC rm,tf    \NC \NC \NR
\NC SerifBold            \NC rm,bf    \NC \NC \NR
\NC SerifItalic          \NC rm,it    \NC \NC \NR
\NC SerifBoldItalic      \NC rm,bi    \NC \NC \NR
\NC SerifSlanted         \NC rm,sl    \NC \NC \NR
\NC SerifBoldSlanted     \NC rm,bs    \NC \NC \NR
\NC SerifCaps            \NC rm,sc    \NC \NC \NR
\stoptabulate
\vskip 1ex
\stopbuffer

\placetable
  [here]
  [tab:symbolic font names]
  {Standard symbolic font names, and the style||alternative pair they belong to.}
  {\getbuffer[symfontnames]}


As mentioned earlier, the items in the third argument of
\type{\definefontsynonym} perculate down the chain of
synonyms. Occasionally, you may want to splice some settings into that
chain, and that is where \type{\setupfontsynonym} comes in handy. 

\showsetup{setupfontsynonym}

For example, the predefined \MKII\ typescripts for font handling that we saw in
the previous chapter contain a sequence of commands like this:

\startexample
\starttyping
\setupfontsynonym [Serif]       [handling=pure]
\setupfontsynonym [SerifBold]   [handling=pure]
\setupfontsynonym [SerifItalic] [handling=pure]
....
\stoptyping
\stopexample

\section {Simple font definitions}
\macro{\tex{definefont}}
\macro{\tex{definedfont}}

The most simple font definition takes place with \type {\definefont}. 

\showsetup{definefont}

This macro defines a font with the same name as the first argument and
you can use its name as an identifier to select that font. The second
argument works in the same way as the second argument to
\type{\definefontsynonym}: you can use either a font synonym or a real
font. There is an optional third argument that can be either a bare
number like \type{1.5 }, or a named setup
(see~\in{section}[sec:setups]). In case of a bare number, that is a local
setting for the interline space. In case of a setup, that setup can do
whatever it wants.

\startbuffer [dingbat]
\loadmapfile [koeieletters]
\definefont  [ContextLogo] [koeielogos at 72pt]
\ContextLogo \char 2
\stopbuffer

For instance:

\startexample
\typebuffer[dingbat]
\stopexample

will result in 

\startreality
\getbuffer[dingbat]
\stopreality

If you want a fixed size font like in the example above, you can define a
font using the primitive \TEX\ \type {at} or \type {scaled} modifiers.

Be warned that \type {at} is often useful, but \type {scaled} is somewhat
unreliable since it scales the font related to its internal design size,
and that is often unknown.  Depending on the design size is especially
dangerous when you use symbolic names, since different fonts have
different design sizes, and designers differ in their ideas about what a
design size is. Compare for instance the 10pt instance of a Computer
Modern Roman with Lucida Bright (which more looks like a 12pt then).

\starttyping
\definefont [TitleFont] [Serif scaled 2400]
\stoptyping

Hardcoded sizes can be useful in many situations, but they can be
annoying when you want to define fonts in such a way that their
definitions adapt themselves to their surroundings.  That is why
\CONTEXT\ provides an additional way of scaling:

\starttyping
\definefont [TitleFont] [Serif sa 2.4]
\stoptyping

The \type {sa} directive means as much as \quote {scaled at
the body font size}. Therefore this definition will lead to
a 24pt scaling when the (document) body font size equals
10pt. Because the definition has a lazy nature, the font
size will adapt itself to the current body font size.

There is an extra benefit to using \type{sa} instead of \type{at}.
Instead of a numeric multiplier, you can also use the identifiers that
were defined in the body font environment that specified the related
dimensions. For example, this scales the font to the \type {b} size,
being 1.440 by default:

\starttyping
\definefont [TitleFont] [Serif sa b]
\stoptyping

In fact, if you use a bare name like in 

\starttyping
\definefont [TitleFont] [Serif]
\stoptyping

it will internally be converted to

\starttyping
\definefont [TitleFont] [Serif sa *]
\stoptyping

which in turn expands into the current actual font size, after the
application of size corrections for super|| and subscripts etc. 

For example

\startbuffer[saex]
\definefont [TitleFont] [Sans]
{\TitleFont test} and {\tfc \TitleFont test}
\stopbuffer

\startexample
\typebuffer[saex]
\stopexample

gives

\startreality
\getbuffer[saex]
\stopreality

A specialized alternative to \type {sa} that is sometimes useful is \type
{mo}. Here the size maps onto to body font size only after it has passed
through an optional size remapping. Such remappings are defined by the
macro \type{\mapfontsize}:

\showsetup{mapfontsize}

Such remapping before applying scaling is sometimes handy for math fonts,
where you may want to use slightly different sizes than the ones given in
the body font environment. In the \CONTEXT\ distribution, this happens
only with the Math Times fonts, where the predefined typescript contains
the following lines:

\starttyping
\mapfontsize [5pt]    [6.0pt]
\mapfontsize [6pt]    [6.8pt]
\mapfontsize [7pt]    [7.6pt]
\mapfontsize [8pt]    [8.4pt]
\mapfontsize [9pt]    [9.2pt]
\mapfontsize [10pt]   [10pt]
\mapfontsize [11pt]   [10.8pt]
\mapfontsize [12pt]   [11.6pt]
\mapfontsize [14.4pt] [13.2pt]
\stoptyping

As we have seen, \type{\definefont} creates a macro name for a font
switch. For ease of use, there is also a direct method to access a 
font:

\showsetup{definedfont}

Where the argument has exactly the same syntax as the second argument to
\type{\definefont}. In fact, this macro executes \type{\definefont} 
internally, and then immediately switches to the defined font.


\section {Defining body fonts}
\macro{\tex{definebodyfont}}


In older versions of \CONTEXT, the model for defining fonts that will be
described in this section was the top||level user interface.  These days,
typescripts are used at the top||level, and the body font definitions are
wrapped inside of those. 

Most commercial fonts have only one design size, and when you create a
typescript for such fonts, you can simply reuse the predefined size
definitions. Later on we will see that this means you can just refer to a
\type{default} definition.

Still, you may need (or want) to know the details of body font definitions
if you create your own typescripts, especially if the fonts are not all
that standard.  For example, because Latin Modern comes in design sizes,
there was a need to associate a specific font with each bodyfont size.  You
may find yourself in a similar situation when you attempt to create a
typescript for a \quote{professional} commercial font set.

The core of this intermediate model is the \type{\definebodyfont} command
that is used as follows:

\starttyping
\definebodyfont [10pt] [rm] [tf=tir at 10pt]
\stoptyping

This single line actually defines two font switches \type{\tf} for use
after a \type{\rm} command, and \type{\rmtf} for direct access.

\blank

As one can expect, the first implementation of a font model in \TEX\ is
also determined and thereby complicated by the fact that the Computer
Modern Roman fonts come in design sizes. As a result, definitions can look
rather complex and because most \TEX\ users start with those fonts, font
definitions are considered to be complex. 

Another complicating factor is that in order to typeset math, even more
(font) definitions are needed. Add to that the fact that sometimes fonts
with mixed encodings have to be used, i.e.\ with the glyphs positioned in
different font slots, and you can understand why font handling in \TEX\
is often qualified as \quote {the font mess}. Flexibility simply has its
price.

Like most other \TEX\ users, Hans Hagen started out using the Computer
Modern Roman fonts. Since these fonts have specific design sizes,
\CONTEXT\ supports extremely accurate \type{\definebodyfont} definitions with
specific font names and sizes for each combination. The following is an
example of that:

\startbuffer[font-6]
\definebodyfont [12pt] [rm]
    [ tf=cmr12,
     tfa=cmr12 scaled \magstep1,
     tfb=cmr12 scaled \magstep2,
     tfc=cmr12 scaled \magstep3,
     tfd=cmr12 scaled \magstep4,
      bf=cmbx12,
      it=cmti12,
      sl=cmsl12,
      bi=cmbxti10 at 12pt,
      bs=cmbxsl10 at 12pt,
      sc=cmcsc10 at 12pt]
\stopbuffer

\typebuffer[font-6]

It should be clear to you that for fonts with design sizes, similar
\type{\definebodyfont} commands will have to be written for each of the
requested body font sizes. But many commercial fonts do not come in
design sizes at all.  In fact, many documents have a rather simple design
and use only a couple of fonts for all sizes.

The previous example used the available \TEX||specifications \type {scaled}
and \type {at}, but (as we say already) \CONTEXT\ supports special keyword
that is a combination of both: \type {sa} (scaled at).

For example, for the Helvetica Type~1 font definition we could define:

\startbuffer[font-7]
\definebodyfont [12pt] [ss]
  [tf=hv  sa 1.000,
   bf=hvb sa 1.000,
   it=hvo sa 1.000,
   sl=hvo sa 1.000,
  tfa=hv  sa 1.200,
  tfb=hv  sa 1.440,
  tfc=hv  sa 1.728,
  tfd=hv  sa 2.074,
   sc=hv  sa 1.000]
\stopbuffer

\typebuffer[font-7]

The scaling is done in relation to the bodyfont size. In analogy with
\TEX's \type {\magstep} we can use \type {\magfactor}: instead of \type
{sa 1.440} we could specify \type {sa \magfactor2}. 

If you are happy with the relative sizes as defined in the body font
environment (and there is no reason not to), the \type{\definebodyfont}
can be four lines shorter. That is because \CONTEXT\ predeclares a whole
collection of names that combine the styles \type{rm}, \type{ss}, \type{tt},
\type{tf}, \type{hw} and \type{cg} with the alternatives \type{bf}, 
\type{it}, \type{sl}, \type{bi}, \type{bs}, and  \type{sc} with the
postfixes \type{a}, \type{b}, \type{c}, \type{d}, \type{x} and \type{xx}.

For the combination of \type{ss} and \type{sl}, the following identifiers
are predeclared: 

\starttyping
\ss    \ssa   \ssb   \ssc   \ssd    \ssx  \ssxx
\sl   \sla  \slb  \slc  \sld   \slx \slxx
\sssl  \sssla \ssslb \ssslc \sssld 
\stoptyping


And because there are no more sizes in the definition any more, we can
just as well combine all of the requested sizes in a single
\type{\definebodyfont} by using a list of sizes as the first
argument. This means exactly the same as repeating that whole list five
(or more) times, but saves a lot of typing:

\starttyping
\definebodyfont [12pt,11pt,10pt,9pt,8pt] [ss]
  [tf=hv  sa 1.000,
   bf=hvb sa 1.000,
   it=hvo sa 1.000,
   sl=hvo sa 1.000,
   sc=hv  sa 1.000]
\stoptyping

Because the font names (may) depend on the encoding vector, we had better
use the previously discussed method for mapping symbolic names. So, any one
of the three following lines can be used, but the third one is best:

\starttyping
\definebodyfont [10pt,11pt,12pt] [ss] [tf=hv        sa 1.000]
\definebodyfont [10pt,11pt,12pt] [ss] [tf=Helvetica sa 1.000]
\definebodyfont [10pt,11pt,12pt] [ss] [tf=Sans      sa 1.000]
\stoptyping


And in the actual \CONTEXT\ core, the default body fonts are in fact
defined with commands like this:

\startbuffer[font-5]
\definebodyfont [default] [rm]
  [ tf=Serif       sa 1,
    ...
    it=SerifItalic sa 1,
    ... ]
\stopbuffer

\typebuffer[font-5]

We saw that \type {\tf} is the default font. Here \type {\tf} is defined as
\typ{Serif sa 1} which means that it is a serif font, scaled to a normal
font size. This \type {Serif} is mapped elsewhere on for example \typ
{Palatino} which in turn is mapped on the actual filename \type {uplr8t},
as demonstrated earlier.

\showsetup{definebodyfont}

The macro syntax for \type{\definebodyfont} is a bit abbreviated. Besides
the two||letter keys that are listed for the third argument, it is also
possible to assign values to font identifiers with the alphabetic suffixes
\type{a} through \type{d} like \type{tfa} as well as the ones with an
\type{x} or \type{xx} suffix like \type{bfx}. You can even define totally new
keywords, if you want that.

As an example we will define a bigger fontsize of \type {\tf}:

\startbuffer
\definebodyfont [10pt,11pt,12pt] [rm] 
    [tfe=Serif at 48pt,
     ite=SerifItalic at 48pt]
\tfe Big {\it Words}.
\stopbuffer

\startexample
\typebuffer
\stopexample

This becomes:

\startreality
\startlinecorrection
\getbuffer
\stoplinecorrection
\stopreality

Note that there is a small trick here: the assignment to \type{ite} is
needed for the command \type{\it} to work properly. Without that, the 
command \type{\it} would run the \quote{normal} version of \type{it} and
that has a size of 11pt.


The keywords \type{mr}, \type {ex}, \type {mi}, \type {sy}, \type {ma},
\type {mb},\type {mc} and \type{md} all relate to math families. As was
already hinted at in~\in{table}[tab:symbolic font names], these have
extended relatives suffixed by \type{bf} for use within bold math
environments. 

Calls of \type{\definebodyfont} for the \type{mm} style look quite
different from the other styles, because they set up these special
keywords, and nothing else.  The first four keys are required in all math
setups just to do basic formula typesetting, the other four (\type{ma}
\dots \type{md}) can be left undefined. Those are normally used for fonts
with special symbols or alphabets like the AMS symbol fonts \type{msam}
and \type{msbm}.

Here is what a setup for a fairly standard \type{mm} could look like:

\startexample
\starttyping
\definebodyfont [10pt] [mm]
    [mr=cmr10,
     ex=cmex10,
     mi=cmmi10,
     sy=cmsy10]

\definebodyfont [17.3pt,14.4pt,12pt,11pt,10pt,9pt] [mm]
    [ma=msam10 sa 1,
     mb=msbm10 sa 1]
\stoptyping
\stopexample

The keys \type{mc} and \type{md} are left undefined. This example
explicitly shows how multiple \type{\definebodyfont}s are combined by
\CONTEXT\ automatically and that there is no need to do everything 
within a single definition (in fact this was already implied by the
\type{tfe} trick above.)

Apart from the calling convention as given in the macro syntax that has
already been shown, there are a few alternative forms of
\type{\definebodyfont} that can be used to defined and call body fonts by
name:

\showsetup{definebodyfontDEF}

This was used in the default serif font defintion shown above: the first
argument to \type{\definebodyfont} was the identifier \type{default}
because these definitions were to be used from within other definitions.  

An actual size will be provided by the commands at the top||level in the
calling chain, the third argument in that \type{\definebodyfont} call will
also be \type{default} instead of actually specifying settings.

\showsetup{definebodyfontREF}

The use of the \type{default} actually happens deep inside \CONTEXT\ so
there is clear code that can be shown, but if it was written out, a call
would for example look like this:

\startexample
\starttyping
\definebodyfont
  [17.3pt,14.4pt,12pt,11pt,10pt,9pt,8pt,7pt,6pt,5pt,4pt]
  [rm,ss,tt,mm]
  [default]
\stoptyping
\stopexample

To end this section: for advanced \TEX\ users there is the
dimension||register \type {\bodyfontsize}. This variable can be used to set
fontwidths. The number (rounded) points is available in \type
{\bodyfontpoints}.

This way of defining fonts has been part of \CONTEXT\ from
the beginning, but as more complicated designs started to
show up, we felt the need for a more versatile mechanism.

\section {Typescripts and typefaces}

\macro{\tex{usetypescript}}
\macro{\tex{usetypescriptfile}}
\macro{\tex{starttypescript}}
\macro{\tex{definetypeface}}

On top of the existing traditional font module, \CONTEXT\ now provides a
more abstract layer of typescripts and building blocks for definitions
and typefaces as font containers. The original font definition files have
been regrouped into such typescripts thereby reducing the number of files
involved.

As we saw earlier, \quote{using} a typescript is done via the a call to the
macro \type{\usetypescript}. Here is the macro syntax setup again:

\showsetup{usetypescript}

Typescripts are in fact just organized definitions, and \quote{using} a
typescript therefore actually means nothing more than executing the set of
definitions that is contained within a particular typescript.

The main defining command for typescripts is a start||stop pair that wraps
the actual macro definitions. 

\starttyping
\starttypescript [...] [...] [...]
   ....
\stoptypescript
\stoptyping

As with \type{\usetypescript}, there can be up to three arguments, and
these two sets of arguments are linked to eachother: the values of the 
first and second argument in the call to \type{\starttypescript} of

\startexample
\starttyping
\starttypescript [palatino] [texnansi,ec,qx,t5,default]
  ...
\stoptypescript
\stoptyping
\stopexample

are what make the \MKII-style call to \type{\usetypescript}

\startexample
\starttyping
\usetypescript [palatino] [ec]
...
\stoptyping
\stopexample

possible and meaningful: the first argument in both cases is the same so
that this matches, and the second argument of \type{\usetypescript} appears
in the list that is the second argument of \type{\starttypescript}, so this
also matches. \CONTEXT\ will execute all matching blocks it knows about:
there may be more than one.

To perform the actual matching, \CONTEXT\ scans through the list of known
\type{\starttypescript} blocks for each of the combinations of items in 
the specified arguments of \type{\usetypescript}. These blocks can be
preloaded definitions in \TEX's memory, or they may come from a file.

There is a small list of typescript files that is tried always, and by
using \type{\usetypescriptfile} you actually add extra ones at the end of
this list.

The automatically loaded files for the three possible engines are, 
in first to last order:

\starttabulate[|lT|lT|lT|p|]
\NC \rmbf pdftex \NC \rmbf xetex \NC \rmbf luatex  \NC \rmbf explanation \NC \FR
\NC type-tmf     \NC type-tmf    \NC type-tmf      \NC Core \TEX\ community fonts\NC \NR
\NC type-siz     \NC type-siz    \NC type-siz      \NC Font size setups\NC \NR
\NC type-one     \NC             \NC               \NC Type~1 free fonts\NC \NR
\NC              \NC type-otf    \NC type-otf      \NC OpenType free fonts\NC \NR
\NC              \NC type-xtx    \NC               \NC MacOSX font support\NC \NR
\NC type-akb     \NC             \NC               \NC Basic Adobe Type~1 mappings\NC \NR
\NC type-loc     \NC type-loc    \NC type-loc      \NC A user configuration file\NC \NR
\stoptabulate

Extra arguments to \type{\usetypescript} are ignored, and that is why that
same two-argument call to \type{\usetypescript} works correctly in \MKIV\
as well, even tough the typescript itself uses only a single argument:

\startexample
\starttyping
\starttypescript [palatino]
  ...
\stoptypescript
\stoptyping
\stopexample

On the other hand, extra arguments to \type{\starttypescript} are not
ignored: a \type{\starttypescript} with two specified arguments will not
be matched by a \type{\usetypescript} that has only one specified
argument. 

However, you can force any key at all to match by using the special
keyword \type{all} in your \type{\usetypescript} or
\type{\starttypescript}.   We will see later that this use of a 
wildcard is sometimes handy.


\subsection{A typescript in action}

Before we can go on and explain how to write \type{\starttypescript}
blocks, we have to step back for a moment to the macro
\type{\definetypeface}, and especially to the third, fourth and fifth
argument:

\startexample
\starttyping
\starttypescript [palatino] [texnansi,ec,qx,t5,default]
\definetypeface[palatino] [rm] [serif] [palatino] [default]
...
\stoptyping
\stopexample

Remember how in the previous chapter there were the tables that listed all
the predefined combinations? It was said there that these \quote{\dots are
nothing more than convenience names that are attached to a group of fonts
by the person that wrote the font definition}. 

Here is how that works: these arguments of \type{\definetypeface} are
actually used as parts of \type{\usetypescript} calls. To be preciese,
inside the macro definition of \type{\definetypeface}, there are the
following lines:

\startexample
\starttyping
\def\definetypeface 
  ...
  \usetypescript[#3,map][#4][name,default,\typefaceencoding,special]
  \usetypescript[#3][#5][size]
  ...
\stoptyping
\stopexample

In our example \type{#3} is \type{serif}, \type{#4} is \type{palatino},
and \type{#5} is default. The value of \type{\typefaceencoding} is
inherited from the calling \type{\usetypescript}. That means that the two
lines expand into:

\starttyping
\usetypescript[serif,map][palatino] [name,default,ec,special]
\usetypescript[serif][default][size]
\stoptyping

And those typescripts will be searched for. This example is using \MKII,
so the list of typescript files is \type{type-tmf}, \type{type-siz}, 
\type{type-one}, \type{type-akb}, and \type{type-loc}. The first two 
arguments of \type{\usetypescript} are handled depth first, so first all
\quote{serif} typescripts are tried against all the files in the list and
then all the \quote{map} typescripts.

Not all of the seached typescript blocks are indeed present in the list
of files that have to be scanned, but a few are, and one apparently even
more than once:

\starttabulate[|lT|l|l|l|]
\NC type-tmf.tex \NC serif \NC palatino \NC name                     \NC\NR
\NC type-one.tex \NC serif \NC palatino \NC texnansi,ec,8r,t5        \NC\NR
\NC type-one.tex \NC serif \NC palatino \NC ec,texnansi,8r           \NC\NR
\NC type-one.tex \NC map   \NC all      \NC --                       \NC\NR
\NC type-siz.tex \NC serif \NC default  \NC size                     \NC\NR
\stoptabulate

All of the found blocks are executed, so let's look at them in order

\starttyping
\starttypescript [serif] [palatino] [name]
    \definefontsynonym [Serif]            [Palatino]
    \definefontsynonym [SerifBold]        [Palatino-Bold]
    \definefontsynonym [SerifItalic]      [Palatino-Italic]
    \definefontsynonym [SerifSlanted]     [Palatino-Slanted]
    \definefontsynonym [SerifBoldItalic]  [Palatino-BoldItalic]
    \definefontsynonym [SerifBoldSlanted] [Palatino-BoldSlanted]
    \definefontsynonym [SerifCaps]        [Palatino-Caps]
\stoptypescript
\stoptyping

This block has mapped the standard symbolic names to names in the
\quote{Palatino} family, one of the standard font synonym actions as
explained in the beginning of this chapter.

\starttyping
\starttypescript [serif] [palatino] [texnansi,ec,8r,t5]
\definefontsynonym [Palatino]             
    [\typescriptthree-uplr8a]  [encoding=\typescriptthree]
\definefontsynonym [Palatino-Italic]      
    [\typescriptthree-uplri8a] [encoding=\typescriptthree]
\definefontsynonym [Palatino-Bold]        
    [\typescriptthree-uplb8a]  [encoding=\typescriptthree]
\definefontsynonym [Palatino-BoldItalic]  
    [\typescriptthree-uplbi8a] [encoding=\typescriptthree]
\definefontsynonym [Palatino-Slanted]     
    [\typescriptthree-uplr8a-slanted-167] [encoding=\typescriptthree]
\definefontsynonym [Palatino-BoldSlanted]
    [\typescriptthree-uplb8a-slanted-167] [encoding=\typescriptthree]
\definefontsynonym [Palatino-Caps]        
    [\typescriptthree-uplr8a-capitalized-800] [encoding=\typescriptthree]

\loadmapfile[\typescriptthree-urw-palatino.map]
\stoptypescript
\stoptyping

This maps the Palatino names onto the actual font files. Some further
processing is taking place here: the calling \type{\usetypescript} that
was called from within the \type{\definetypeface} knows that it wants
\type{ec} encoding. Because this is the third argument, it becomes the
replacement of \type{\typescriptthree}. The body of the typescript
therefore reduces to:

\begingroup
\hfuzz10pt
\starttyping
\definefontsynonym[Palatino]            [ec-uplr8a]                [encoding=ec]
\definefontsynonym[Palatino-Italic]     [ec-uplri8a]               [encoding=ec]
\definefontsynonym[Palatino-Bold]       [ec-uplb8a]                [encoding=ec]
\definefontsynonym[Palatino-BoldItalic] [ec-uplbi8a]               [encoding=ec]
\definefontsynonym[Palatino-Slanted]    [ec-uplr8a-slanted-167]    [encoding=ec]
\definefontsynonym[Palatino-BoldSlanted][ec-uplb8a-slanted-167]    [encoding=ec]
\definefontsynonym[Palatino-Caps]       [ec-uplr8a-capitalized-800][encoding=ec]

\loadmapfile[ec-urw-palatino.map]
\stoptyping
\endgroup

Incidentally, this also loads a font map file.  In earlier versions of \CONTEXT, this
was done by separate typescripts in the file \type{type-map.tex}, but nowadays all map
loading is combined with the definition of the synonyms that link to the true fonts on
the harddisk. This way, there is a smaller chance of errors creeping
in. See~\in{section}[sec:map files] for more details on font map files.

The third match is a block that sets sets up \quote{TeXPalladioL} font synonyms. These
will not actually be used, but it is a match so it will be executed anyway.

\starttyping
\starttypescript [serif] [palatino] [ec,texnansi,8r]
\definefontsynonym[TeXPalladioL-BoldItalicOsF]
    [\typescriptthree-fplbij8a][encoding=\typescriptthree]
...
\stoptypescript
\stoptyping

The next matched entry loads the font map files for the default fonts:

\starttyping
\starttypescript [map] [all]
    \loadmapfile[original-base.map]
    \loadmapfile[original-ams-base.map]
\stoptypescript
\stoptyping

this will not really be needed for the palatino \type{\rm} typescript, but it ensures
that even if there is something horribly wrong with the used typescripts, at least 
\PDFTEX\ will be able to find the Latin Modern (the default font set) on the
harddisk.

The last match is the missing piece of the font setup: 

\starttyping
\starttypescript [serif] [default] [size]
  \definebodyfont
    [4pt,5pt,6pt,7pt,8pt,9pt,10pt,11pt,12pt,14.4pt,17.3pt]
    [rm] [default]
\stoptypescript
\stoptyping

and now the typescript is complete.  

As explained earlier, that last block references a named
\type{\definebodyfont} that is defined in \type{type-unk.tex}:

\starttyping
\definebodyfont [default] [rm]
  [tf=Serif sa 1,
   bf=SerifBold sa 1,
   it=SerifItalic sa 1,
   sl=SerifSlanted sa 1,
   bi=SerifBoldItalic sa 1,
   bs=SerifBoldSlanted sa 1,
   sc=SerifCaps sa 1]
\stoptyping

similar \type{default} blocks are defined for the other five font styles also.

Looking back, you can see that the Palatino-specific typescripts did
actually do anything except definining font synonyms, loading a map file,
and calling a predefined \type{bodyfont}. 

\subsection{Some more information}

As we saw already, typescripts and its invocations have up to three specifiers.  An
invocation matches the script specification when the three arguments have common
keywords, and the special keyword \type {all} is equivalent to any match.

Although any keyword is permitted in any of the three arguments, the
current definitions (and macros like \type{\definetypeface}) make heavy
use of some keys in particular:

\starttabulate[|lT|p|]
\HL
\NC \bf pattern \NC \bf application \NC \NR
\HL
\NC [serif] [*] [*]   \NC serif fonts             \NC \NR
\NC [sans] [*] [*]    \NC sans serif fonts        \NC \NR
\NC [mono] [*] [*]    \NC mono spaced fonts       \NC \NR
\NC [math] [*] [*]    \NC math fonts              \NC \NR
\NC [*] [*] [size]    \NC size specifications     \NC \NR
\NC [*] [*] [name]    \NC symbolic name mapping   \NC \NR
\NC [*] [*] [special] \NC special settings        \NC \NR
\NC [*] [all] [*]     \NC default case(s)         \NC \NR
\NC [map] [*] [*]     \NC map file specifications \NC \NR
\HL
\stoptabulate

When you take a close look at the actual files in the distribution you
will notice a quite a few other keywords.  One in particular is worth
mentioning: instead of the predefined sizes in \type {default}, you can
use the \type {dtp} size scripts with their associated body font
environments by using

\starttyping
\usetypescript [all] [dtp] [size]
\stoptyping

or 

\starttyping
\definetypeface[palatino] [rm] [serif] [palatino] [dtp]
\stoptyping

\blank

In the top||level typescript for the palatino, we had a bunch of
\type{\definetypeface} commands, as follows:

\typebuffer[funny]

Once these commands are executed (wether or not as part of a typescript), \type
{\funny} will enable this specific collection of fonts. In a similar way we can 
define a collection \type {\joke}.

\typebuffer[joke]

And the familiar Computer Modern Roman as \type {\whow}:

\typebuffer[whow]


\startbuffer[sample]
Who is {\it fond} of fonts?
Who claims that $t+e+x+t=m+a+t+h$?
Who {\ss can see} {\tt the difference} here?
\stopbuffer

Now has become possible to switch between these three font collections at will.  
Here is a sample of some text and a little bit of math:

\startexample
\typebuffer[sample]
\stopexample

When typeset in \type {\funny}, \type {\joke}, and \type{whow}, the samples look like:

\startlines
{\funny \getbuffer[sample]}
\stoplines

\startlines
{\joke  \getbuffer[sample]}
\stoplines

\startlines
{\whow  \getbuffer[sample]}
\stoplines

With \type {\showbodyfont} you can get an overview of this font.

\placefigure
  {The \type {funny} typeface collection.}
  {\showbodyfont[funny]}

\placefigure
  {The \type {joke} typeface collection.}
  {\showbodyfont[joke]}

\placefigure
  {The \type {whow} typeface collection.}
  {\showbodyfont[whow]}

When defining the joke typeface collection, we used a scale
directive. The next sample demonstrates the difference
between the non scaled and the scaled alternatives.

\startlines
{\nojoke \getbuffer[sample]}
\stoplines

\startlines
{\joke   \getbuffer[sample]}
\stoplines


It may not be immediately clear from the previous examples, but a big
difference between using typeface definitions and the old method of
redefining over and over again, is that the new method uses more
resources. This is because each typeface gets its own name space
assigned. As an intentional side effect, the symbolic names also follow
the typeface. This means that for instance:

\startbuffer[big]
\definefont[MyBigFont][Serif sa 1.5] \MyBigFont A bit larger!
\stopbuffer

\typebuffer[big]

will adapt itself to the currently activated serif font
shape, here \type {\funny}, \type {\joke} and \type {\whow}.

\startlines
{\funny \getbuffer[big]}
{\joke  \getbuffer[big]}
{\whow  \getbuffer[big]}
\stoplines

\subsection{A bit more about math}

Math is kind of special in the sense that it has its own set
of fonts, either or not related to the main text font. By
default, a change in style, for instance bold, is applied to
text only.

\startbuffer[math]
$        \sqrt{625} =     5\alpha$
$\bf     \sqrt{625} =     5\alpha$
$        \sqrt{625} = \bf 5\alpha$
$\bfmath \sqrt{625} =     5\alpha$
\stopbuffer

\typebuffer[math]

The difference between these four lines is as follows:

\startlines
\funny \getbuffer[math]
\stoplines

In order to get a bold $\alpha$ symbol, we need to define bold math
fonts. \footnote {Bold math is already prepared in the core modules, so
normally one can do with less code} Assuming the font's typescripts
support bold math, the most convenient way of doing this is the following:

\startbuffer
\definetypeface [whow] [mm]
  [math,boldmath] [modern] [default] [encoding=texnansi]
\stopbuffer

\typebuffer \getbuffer

Bold math looks like this:

\startlines
\whow \getbuffer[math]
\stoplines

The definitions are given on the next page. Such definitions are normally collected in
the project bound file, for instance called \type {typeface.tex}, that is then manually
added to the list of typescript files:

\starttyping
\usetypescriptfile[typeface] % project scripts
\stoptyping

%An example of such a file is shown below:

%\start
%\switchtobodyfont[9pt] \typefile{typeface}
%\stop

It is also possible to avoid typescripts. When definitions are
used only once, it makes sense to use a more direct method.
We will illustrate this with a bit strange example.

Imagine that you want some math formulas to stand out, but
that you don't have bold fonts. In that case you can for
instance scale them. A rather direct method is the following.

\startbuffer
\definebodyfont
  [funny]
  [12pt,11pt,10pt,9pt,8pt,7pt] [mm]
  [mrbf=MathRoman     mo 2,
   exbf=MathExtension mo 2,
   mibf=MathItalic    mo 2,
   sybf=MathSymbol    mo 2]
\stopbuffer

\typebuffer \getbuffer

Our math sample will now look like:

\startlines
\funny \getbuffer[math]
\stoplines

We can also use an indirect method:

\startbuffer
\definebodyfont
  [smallmath] [mm]
  [mrbf=MathRoman     mo .5,
   exbf=MathExtension mo .5,
   mibf=MathItalic    mo .5,
   sybf=MathSymbol    mo .5]

\definebodyfont
  [funny]
  [12pt,11pt,10pt,9pt,8pt,7pt]
  [mm] [smallmath]
\stopbuffer

\typebuffer \getbuffer

This method is to be preferred when we have to define more
typefaces since it saves keystrokes.

\startlines
\funny \getbuffer[math]
\stoplines

For efficiency reasons, the font definitions (when part of
a typeface) are frozen the first time they are used. Until
that moment definitions will adapt themselves to changes in
for instance scaling and (mapped) names. Freezing
definitions is normally no problem because typefaces are
defined for a whole document and one can easily define
more instances. When you redefine it, a frozen font is
automatically unfrozen.

\section{Predefined font, style and alternative keywords}
\macro{\tex{definebodyfontswitch}}
\macro{\tex{definefontstyle}}
\macro{\tex{definealternativestyle}}

Some of the internal commands are worth mentioning because they define keywords and
you may want to add to the list.

Font size switching is done with keywords like \type{twelvepoint} and commands like
\type {\twelvepoint} or \type {\xii}, which is comparable to the way it is done in
plain \TEX. These commands are defined with:


\startbuffer[font-10]
\definebodyfontswitch [fourteenpointfour] [14.4pt]
\definebodyfontswitch [twelvepoint]       [12pt]
\definebodyfontswitch [elevenpoint]       [11pt]
\definebodyfontswitch [tenpoint]          [10pt]
\definebodyfontswitch [ninepoint]         [9pt]
\definebodyfontswitch [eightpoint]        [8pt]
\definebodyfontswitch [sevenpoint]        [7pt]
\definebodyfontswitch [sixpoint]          [6pt]
\definebodyfontswitch [fivepoint]         [5pt]
\definebodyfontswitch [fourpoint]         [4pt]
\definebodyfontswitch [xii]  [12pt]
\definebodyfontswitch [xi]   [11pt]
\definebodyfontswitch [x]    [10pt]
\definebodyfontswitch [ix]   [9pt]
\definebodyfontswitch [viii] [8pt]
\definebodyfontswitch [vii]  [7pt]
\definebodyfontswitch [vi]   [6pt]
\stopbuffer

\typebuffer[font-10]

But be warned that  \type{\xi} is later redefined as a greek symbol.

The keys in \type {\setupbodyfont} are defined in terms of:

\startbuffer[font-11]
\definefontstyle [rm,roman,serif,regular]    [rm]
\definefontstyle [ss,sansserif,sans,support] [ss]
\definefontstyle [tt,teletype,type,mono]     [tt]
\definefontstyle [hw,handwritten]            [hw]
\definefontstyle [cg,calligraphic]           [cg]
\stopbuffer

\typebuffer[font-11]

In many command setups we encounter the parameter \type {style}. In those situations
we can specify a key. These keys are defined with \type {\definealternativestyle}. The
third argument is only of importance in chapter and section titles, where, apart from
\type {\cap}, we want to obey the font used there.

\startbuffer[font-12]
\definealternativestyle [mediaeval]         [\os]               []
\definealternativestyle [normal]            [\tf]               []
\definealternativestyle [bold]              [\bf]               []
\definealternativestyle [type]              [\tt]               []
\definealternativestyle [mono]              [\tt]               []
\definealternativestyle [slanted]           [\sl]               []
\definealternativestyle [italic]            [\it]               []
\definealternativestyle [boldslanted,
                         slantedbold]       [\bs]               []
\definealternativestyle [bolditalic,
                         italicbold]        [\bi]               []
\definealternativestyle [small,
                         smallnormal]       [\tfx]              []
\definealternativestyle [smallbold]         [\bfx]              []
\definealternativestyle [smalltype]         [\ttx]              []
\definealternativestyle [smallslanted]      [\slx]              []
\definealternativestyle [smallboldslanted,
                         smallslantedbold]  [\bsx]              []
\definealternativestyle [smallbolditalic,
                         smallitalicbold]   [\bix]              []
\definealternativestyle [sans,
                         sansserif]         [\ss]               []
\definealternativestyle [sansbold]          [\ss\bf]            []
\definealternativestyle [smallbodyfont]     [\setsmallbodyfont] []
\definealternativestyle [bigbodyfont]       [\setbigbodyfont]   []
\definealternativestyle [cap,
                         capital]           [\smallcapped]      [\smallcapped]
\definealternativestyle [smallcaps]         [\sc]               [\sc]
\definealternativestyle [WORD]              [\WORD]             [\WORD]
\stopbuffer

\typebuffer[font-12]

In \in{section}[emphasize] we have already explained how
{\em emphasizing} is defined. With oldstyle digits this is
somewhat different. We cannot on the forehand in what font
these can be found. By default we have the setup:


\startbuffer[font-13]
\definefontsynonym [OldStyle] [MathItalic]
\stopbuffer

\typebuffer[font-13]

As we see they are obtained from the same font as the math italic characters. The
macro \type{\os} fetches the runtime setting by executing
\type{\symbolicfont{OldStyle}}, which is just a low-level version of
\type{\definedfont[OldStyle sa *]}. A few other macros behave just like that:

\starttabulate[|lT|l|l|]
\NC \rm\bf macro  \NC \bf synonym     \NC \bf default value   \NC \NR
\NC \tex{os}      \NC OldStyle        \NC MathItalic (lmmi10) \NC \NR
\NC \tex{frak}    \NC Fraktur         \NC eufm10              \NC \NR
\NC \tex{goth}    \NC Gothic          \NC eufm10              \NC \NR
\NC \tex{cal}     \NC Calligraphic    \NC cmsy10 (lmsy10)     \NC \NR
\NC \tex{bbd}     \NC Blackboard      \NC msbm10              \NC \NR
\stoptabulate

In addition to all the alrady mentioned commands there are others, for example macros
for manipulating accents. These commands are discussed in the file \type
{font-ini}. More information can also be found in the file \type {core-fnt} and
specific gimmicks in the file \type {supp-fun}. So enjoy yourself.

\section {Symbols and glyphs}

Some day you may want to define your own symbols, if
possible in such a way that they nicely adapt themselves to
changes in style and size. A good example are the \symbol
[euro] symbols. You can take a look in \type {symb-eur.tex}
to see how such a glyph is defined.

\starttyping
\definefontsynonym [EuroSerif]     [eurose]
\definefontsynonym [EuroSerifBold] [euroseb]
...
\definefontsynonym [EuroSans]      [eurosa]
\definefontsynonym [EuroSansBold]  [eurosab]
...
\definefontsynonym [EuroMono]      [euromo]
\definefontsynonym [EuroMonoBold]  [euromob]
\stoptyping

Here we use the free Adobe euro fonts, but there are
alternatives available. The symbol itself is defined as:

\starttyping
\definesymbol [euro] [\getglyph{Euro}{\char160}]
\stoptyping

You may notice that we only use the first part of the
symbolic name. \CONTEXT\ will complete this name according
to the current style. You can now access this symbol with
\typ {\symbol [euro]}

\starttabulate[|l|c|c|c|c|c|c|]
\NC           \NC \tex{tf}    \NC \tex{bf}    \NC \tex{sl}
              \NC \tex{it}    \NC \tex{bs}    \NC \tex{bi}    \NC\NR
\NC \rm Serif \NC \rm\tf\euro \NC \rm\bf\euro \NC \rm\sl\euro
              \NC \rm\it\euro \NC \rm\bs\euro \NC \rm\bi\euro \NC\NR
\NC \ss Sans  \NC \ss\tf\euro \NC \ss\bf\euro \NC \ss\sl\euro
              \NC \ss\it\euro \NC \ss\bs\euro \NC \ss\bi\euro \NC\NR
\NC \tt Mono  \NC \tt\tf\euro \NC \tt\bf\euro \NC \tt\sl\euro
              \NC \tt\it\euro \NC \tt\bs\euro \NC \tt\bi\euro \NC\NR
\stoptabulate

More details on defining symbols and symbol sets can be
found in the documentation of the symbol modules.

\section{Encodings}
\macro{\tex{startencoding}}
\macro{\tex{startmapping}}
\macro{\tex{definecharacter}}
\macro{\tex{defineaccent}}
\macro{\tex{definecommand}}
\macro{\tex{definecasemap}}

\todo{Add macro syntax definition blocks}

Until now we assumed that an~\type {a} will become an~a
during type setting. However, this is not always the case.
Take for example~\"a or~\ae. This character is not available
in every font and certainly not in the Computer Modern
Typefaces. Often a combination of characters \type {\"a} or a
command \type {\ae} will be used to produce such a character.
In some situation \TEX\ will combine characters
automatically, like in \type {fl} that is combined to fl and
not \hbox{f}\hbox{l}. Another problem occurs in converting
small print to capital print and vice versa.

Below you see an example of the \type {texnansi} mapping:


\startbuffer[enco-1]
\startmapping[texnansi]
  \definecasemap 228 228 196  \definecasemap 196 228 196
  \definecasemap 235 235 203  \definecasemap 203 235 203
  \definecasemap 239 239 207  \definecasemap 207 239 207
  \definecasemap 246 246 214  \definecasemap 214 246 214
  \definecasemap 252 252 220  \definecasemap 220 252 220
  \definecasemap 255 255 159  \definecasemap 159 255 159
\stopmapping
\stopbuffer

\typebuffer[enco-1]

This means so much as: in case of a capital the character
with code 228 becomes character 228 and in case of small
print the character becomes character 196.

These definitions can be found in \type {enco-ans}. In this
file we can also see:


\startbuffer[enco-2]
\startencoding[texnansi]
  \defineaccent " a 228
  \defineaccent " e 235
  \defineaccent " i 239
  \defineaccent " o 246
  \defineaccent " u 252
  \defineaccent " y 255
\stopencoding
\stopbuffer

\startbuffer[enco-3]
\startencoding[texnansi]
  \definecharacter ae 230
  \definecharacter oe 156
  \definecharacter o  248
  \definecharacter AE 198
\stopencoding
\stopbuffer

\typebuffer[enco-2]

and

\typebuffer[enco-3]

As a result of the way accents are placed over characters we
have to approach accented characters different from normal
characters. There are two methods: \TEX\ does the accenting
itself {\em or} prebuild accentd glyphs are used. The
definitions above take care of both methods. Other
definitions are sometimes needed. In the documentation of
the file \type {enco-ini} more information on this can be
found.

\section[sec:map files]{Map files}
\macro{\tex{loadmapfile}}
\macro{\tex{setupencoding}}

\todo{This section is too informal}

If you're already sick of reading about fonts, you probably
don't want read this section. But alas, \DVI\ post processors
and \PDFTEX\ will not work well if you don't provide them
\type {map} files that tell them how to handle the files
that contain the glyphs.

In its simplest form, a definition looks as follows:

\starttyping
usedname < texnansi.enc < realname.pfb
\stoptyping

This means as much as: when you want to include a file that
has the \type {tfm} file \type {usedname}, take the outline
file \type {realname.pfb} and embed it with the \type
{texnansi} encoding vector. Sometimes you need more
complicated directives and you can leave that to the
experts. We try to keep up with changes in the map file
syntax, the names of fonts, encodings, locations in the
\TEX\ tree, etc. However, it remains a troublesome area.

It makes sense to take a look at the \type {cont-sys.rme} file
to see what preferences make sense. If you want to speed up
the typescript processing, say (in \type {cont-sys.tex}:

\starttyping
\preloadtypescripts
\stoptyping

If you want to change the default encoding, you should add
something:

\starttyping
\setupencoding [default=texnansi]
\stoptyping

You can let \CONTEXT\ load the map files for \PDFTEX:

\starttyping
\autoloadmapfilestrue
\stoptyping

The following lines will remove existing references to map
files and load a few defaults.

\starttyping
\resetmapfiles
\loadmapfile[original-base.map]
\loadmapfile[original-ams-base.map]
\loadmapfile[original-public-lm.map]
\stoptyping

As said, map files are a delicate matter.

\section{Installing fonts}

\todo{Document use of \MKIV and \XETEX\ and in particular OSFONTDIR}

Most \TEX\ distributions come with a couple of fonts, most
noticeably the Computer Modern Roman typefaces. In order to
use a font, \TEX\ has to know its characteristics. These are
defined in \type {tfm} and \type {vf} files. In addition to
these files, on your system you can find a couple of more
file types.

\starttabulate[|cT|lp|]
\HL
\NC \bf suffix \NC \bf content \NC \NR
\HL
\NC tfm \NC \TEX\ specific font metric files that, in
            many cases, can be generated from \type {afm}
            files \NC \NR
\NC vf  \NC virtual font files, used for building glyph
            collections from other ones \NC \NR
\NC afm \NC Adobe font metric files that are more limited
            than \type {tfm} files (especially for math
            fonts) \NC \NR
\NC pfm \NC Windows specific font metric files, not used
            by \TEX\ applications \NC \NR
\NC pfb \NC files that contain the outline specification of
            the glyphs fonts, also called Type 1 \NC \NR
\NC enc \NC files with encoding vector specifications \NC \NR
\NC map \NC files that specify how and what font files
            are to be included \NC \NR
\HL
\stoptabulate

On your disk (or cdrom) these files are organized in such a
way that they can be located fast. \footnote {If you have
installed \TETEX\ or \FPTEX\ (possibly from the \TEX live
\CDROM) you will have many thousands of font files on your
system.} The directory structure normally is as follows:

\starttyping
texmf / fonts  / tfm    / vendor   / name / *.tfm
               / afm    / vendor   / name / *.afm
               / pfm    / vendor   / name / *.pfm
               / vf     / vendor   / name / *.vf
               / type1  / vendor   / name / *.pfb
      / pdftex / config /                   *.cfg
               / config /                   *.map
               / config / encoding /        *.enc
\stoptyping

The \type {texmf-local} or even better \type {texmf-fonts}
tree normally contains your own fonts, so that you don't
have to reinstall them when you reinstall the main tree.
The \type {pdftex} directory contains the files that
\PDFTEX\ needs in order to make decisions about the fonts
to include. The \type {enc} files are often part of
distributions, as is the configuration \type {cfg} file.
When you install new fonts, you often also have to add or
edit \type {map} files.

\CONTEXT\ comes with a \PERL\ script \type {texfont.pl}
that you can use to install new fonts. Since its usage is
covered by a separate manual, we limit ourselves to a short
overview.

Say that you have just bought a new font. A close look at
the files will reveal that you got at least a bunch of \type
{afm} and \type {pfb} files and if you're lucky \type {tfm}
files.

Installing such a font can be handled by this script. For
this you need to know (or invent) the name of the font
vendor, as well as the name of the font. The full set of
command line switches is given below: \footnote {there are a
couple of more switches described in the manual \type
{mtexfonts}.}

\starttabulate[|lT|lp|]
\HL
\NC \bf switch \NC \bf meaning \NC \NR
\HL
\NC fontroot   \NC texmf font root (automatically determined) \NC \NR
\NC vendor     \NC vendor name (first level directory) \NC \NR
\NC collection \NC font collection (second level directory)\NC \NR
\NC encoding   \NC encoding vector (default: texnansi) \NC \NR
\NC sourcepath \NC when installing, copy from this path \NC \NR
\NC install    \NC copy files from source to font tree \NC \NR
\NC makepath   \NC when needed, create the paths \NC \NR
\NC show       \NC run tex on \type{*.tex} afterwards \NC \NR
\HL
\stoptabulate

You seldom need to use them all. In any case it helps if you
have a local path defined already. The next sequence does
the trick:

\starttyping
texfont --ve=FontFun --co=FirstFont --en=texnansi --ma --in
\stoptyping

This will generate the \type {tfm} files from the \type
{afm} files, and copy them to the right place. The Type~1
files (\type {pfb}) will be copied too. The script also
generates a \type {map} file. When this is done successfully,
a \TEX\ file is generated and processed that shows the font
maps. If this file looks right, you can start using the
fonts. The \TEX\ file also show you how to define the fonts.

This script can also do a couple of more advanced tricks.
Let us assume that we have bought (or downloaded) a new
font package in the files \type {demofont.afm} and \type
{demofont.pfb} which are available on the current (probably
scratch) directory. First we make sure that this font is
installed (in our case we use a copy of the public Iwona
Regular):

\starttyping
texfont --ve=test --co=test --ma --in demofont
\stoptyping

We can now say:

% \loadmapfile[demofont.map] % in case we forgot to generate

\startbuffer
\loadmapfile[texnansi-test-test.map]
\definefontsynonym[DemoFont][texnansi-demofont]
\ruledhbox{\definedfont[DemoFont at 50pt]Interesting}
\stopbuffer

\typebuffer \startlinecorrection \getbuffer \stoplinecorrection

From this font, we can derive a slanted alternative by
saying:

\starttyping
texfont --ve=test --co=test --ma --in --sla=.167 demofont
\stoptyping

The map file is automatically extended with the entry needed.

\startbuffer
\definefontsynonym[DemoFont-Slanted][texnansi-demofont-slanted-167]
\ruledhbox{\definedfont[DemoFont-Slanted at 50pt]Interesting}
\stopbuffer

\typebuffer \startlinecorrection \getbuffer \stoplinecorrection

We can also create a wider version:

\starttyping
texfont --ve=test --co=test --ma --in --ext=1.50 demofont
\stoptyping

When you use the \type {--make} and \type {--install} switch,
the directories are made, fonts installed, and entries
appended to the map file if needed.

\startbuffer
\definefontsynonym[DemoFont-Extended][texnansi-demofont-extended-1500]
\ruledhbox{\definedfont[DemoFont-Extended at 50pt]Interesting}
\stopbuffer

\typebuffer \startlinecorrection \getbuffer \stoplinecorrection

Instead of using pseudo caps in \TEX\ by using \type
{\kap}, you can also create a pseudo small caps font.

\starttyping
texfont --ve=test --co=test --ma --in --cap=0.75 demofont
\stoptyping

This method is much more robust but at the cost of an extra
font.

\startbuffer
\definefontsynonym[DemoFont-Caps][texnansi-demofont-capitalized-750]
\ruledhbox{\definedfont[DemoFont-Caps at 50pt]Interesting}
\stopbuffer

\typebuffer \startlinecorrection \getbuffer \stoplinecorrection

\starttabulate[|lT|lp|]
\HL
\NC \bf switch \NC \bf meaning \NC \NR
\HL
\NC extend=factor \NC stretch the font to the given factor \NC \NR
\NC narrow=factor \NC shrink the font to the given factor \NC \NR
\NC slant=factor  \NC create a slanted font \NC \NR
\NC caps=factor   \NC replace lowercase characters by small uppercase ones \NC \NR
\NC test          \NC use test/test as vendor/collection \NC \NR
\HL
\stoptabulate

When manipulating a font this way, you need to provide a
file name. Instead of a factor you can give the keyword
\type {default} or a \type {*}.

\starttyping
texfont --test --auto --caps=default demofont
\stoptyping

The previous example runs create fonts with the rather
verbose names:

\starttyping
demofont
demofont-slanted-167
demofont-extended-150
demofont-capitalized-750
\stoptyping

This naming scheme makes it possible to use more instances
without the risk of conflicts.

In the distribution you will find an example batch file
\type {type-tmf.dat} which creates metrics for some free
fonts for the encoding specified. When you create the
default font metrics this way, preferably \type
{texmf-fonts}, you have a minimal font system tuned for you
prefered encoding without the risk for name clashes. When
you also supply \type {--install}, the font outlines will be
copied from the main tree to the fonts tree, which sometimes
is handy from the perspective of consistency.

\section {Getting started}

\todo{This section needs to be modernized}

The way \TEX\ searches for files (we're talking \WEBC\ now)
is determined by the configuration file to which the \type
{TEXMFCNF} environment variable points (the following
examples are from my own system):

\starttyping
set TEXMFCNF=T:/TEXMF/WEB2C
\stoptyping

When searching for files, a list of directories is used:

\starttyping
set TEXMF={$TEXMFFONTS,$TEXMFPROJECT,$TEXMFLOCAL,!!$TEXMFMAIN}
\stoptyping

Here we've added a font path, which itself is set with:

\starttyping
set TEXMFMAIN=E:/TEX/TEXMF
set TEXMFLOCAL=E:/TEX/TEXMF-LOCAL
set TEXMFFONTS=E:/TEX/TEXMF-FONTS
\stoptyping

Now you can generate metrics and map files. The batch file is
searched for at the \CONTEXT\ data path in the \TEXMF\ tree
or on the local path.

\starttyping
texfont --encoding=ec --batch type-tmf.dat
\stoptyping

If you want to play with encoding, you can also generate
more encodings, like \type {8r} or \type {texnansi}.

\starttyping
texfont --encoding=texnansi --batch type-tmf.dat
texfont --encoding=8r       --batch type-tmf.dat
\stoptyping

After a while, there will be generated \type {tfm}, \type
{vf}, and \type {map} files. If you let \CONTEXT\ pass
the map file directives to \PDFTEX, you're ready now.
Otherwise you need to add the names of the mapfiles to the
file \type {pdftex.cfg}. You can best add them in front of
the list, and, if you use \CONTEXT\ exclusively, you can
best remove the other ones.

As a test you can process the \TEX\ files that are generated
in the process. These also give you an idea of how well the
encoding vectors match your expectations.

Now, the worst that can happen to you when you process your
files, is that you get messages concerning unknown \type
{tfm} files or reports on missing fonts when \PDFTEX\
writes the file. In that case, make sure that you indeed
{\em have} the right fonts (generated) and|/|or that the map
files are loaded. As a last resort you can load all map
files by saying:

\starttyping
\usetypescript [map] [all]
\stoptyping

and take a look at the log file and see what is reported.

In due time we will provide font generation scripts for
installation of other fonts as well as extend the
typescript collection.

\section {Remarks}

It really makes sense to take a look at the font and type
definition files (\type {font-*.tex} and \type {type-*.tex}).
There are fallbacks defined, as well as generic definitions.
Studying styles and manual source code may also teach you a
few tricks.

\stopcomponent
