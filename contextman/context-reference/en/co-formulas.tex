\startcomponent co-formulas

\environment contextref-env
\product contextref

\input supp-mps % for \MPdivten (chemic)
\let \ifgrTEXgraphic \ifforceMPTEXgraphic

\chapter[formulas]{Formulas}

\section{Introduction}

For what reason do we need a complete chapter on formulas?
The reason is obvious: a considerable part of the
functionality of \TEX\ relates to math typesetting since
the main reason for developing \TEX\ was the need for
typesetting math.

In \CONTEXT\ math typesetting is not really an isue.
\CONTEXT\ was developed for typesetting educational
materials and not necessarily math. Therefore more attention
was paid to chemical formulas and consistent use of units
than to math. Math was available anyhow.

In \CONTEXT\ the functionality is more oriented towards the
educational disciplines and these can be found in specific
modules. A module will not supply basic functionality
because it can be found in the core.

There are modules for chemical stuff, units and
flow||charts, which all have their own manual. The same goes
for the math module. This module contains the same
functionality as the macros developed by the {\em American
Mathematical Society}. Those macros are well||known in the
\TEX\ community. Most extensions concern the interface and
consistent spacing. In this chapter we pay attention to the
standard functionality in \CONTEXT.

\section{Basic commands}
\index{placing+formulas}
\index{overviews+formulas}
\index{numbering+formulas}
\index{formulas+placing}
\index{formulas+overviews}
\macro{\tex{setupformulae}}
\macro{\tex{placeformula}}
\macro{\tex{placesubformula}}
\macro{\tex{subformulanumber}}
\macro{\tex{formulanumber}}
\macro{\tex{startformula}}

Typesetting formulas is one of the strong points of \TEX.
Special commands are available for typesetting math. These
commands are enclosed by single or double dollar signs.

In the running text we use single dollar signs:
\type {$a=b^2+1/c$} becomes  $a=b^2+1/c$. In conjunction
with in||line||math there is display||math, or rather
formulas surrounded by whitespace. Those formulas are
frequently numbered. The location and way of numbering can
be set with:

\showsetup{setupformulas}

With \type{left} and \type{right} characters on the left
or right side of the formula number are set up. Default
these are \type{(}~and~\type{)}.

A (numbered) formula is defined with the commands:

\showsetup{placeformula}

\showsetup{placesubformula}

The reference and subnumber are optional. Below we give
some examples of formulas. In the margin we display the
references. Typing the formula number manually is necessary
when we make use of tables, matrices and \TEX||commando's
like \type{\displaylines}. In the examples we use \type
{$$} to save some space; however we advise you to use the
command \type {\startformula}.

\start

\def\ShowMathExample[#1]%
  {\bgroup
   \setupinmargin[location=left]
   \def\doShowMathExample##1{\in{##1:}[##1]\\}%
   \doifsomething{#1}
     {\margintitle{\processcommalist[#1]\doShowMathExample\unskip}}
   \typebuffer
   \getbuffer
   \egroup}

\startbuffer
\placeformula          $$ a + b = c                            $$
\stopbuffer
\ShowMathExample[]

\startbuffer
\placeformula          $$ a + b = c                            $$
\stopbuffer
\ShowMathExample[]

\startbuffer
\placesubformula   {a} $$ a + b = c                            $$
\stopbuffer
\ShowMathExample[]

\startbuffer
\placesubformula   {b} $$ a + b = c                            $$
\stopbuffer
\ShowMathExample[]

\startbuffer
\placeformula   [a]    $$ a + b = c                            $$
\stopbuffer
\ShowMathExample[a]

\startbuffer
\placeformula   [b]    $$ a + b = c                            $$
\stopbuffer
\ShowMathExample[b]

\startbuffer
\placesubformula[c]{a} $$ a + b = c                            $$
\stopbuffer
\ShowMathExample[c]

\startbuffer
\placesubformula[d]{b} $$ a + b = c                            $$
\stopbuffer
\ShowMathExample[d]

\startbuffer
\placeformula          $$ a + b = c \quad \formulanumber       $$
\stopbuffer
\ShowMathExample[]

\startbuffer
\placeformula          $$ a + b = c \quad \formulanumber       $$
\stopbuffer
\ShowMathExample[]

\startbuffer
\placesubformula   {a} $$ a + b = c \quad \formulanumber       $$
\stopbuffer
\ShowMathExample[]

\startbuffer
\placesubformula   {b} $$ a + b = c \quad \formulanumber       $$
\stopbuffer
\ShowMathExample[]

\startbuffer
\placeformula   [e]    $$ a + b = c \quad \formulanumber       $$
\stopbuffer
\ShowMathExample[e]

\startbuffer
\placeformula   [f]    $$ a + b = c \quad \formulanumber       $$
\stopbuffer
\ShowMathExample[f]

\startbuffer
\placesubformula[g]{a} $$ a + b = c \quad \formulanumber       $$
\stopbuffer
\ShowMathExample[g]

\startbuffer
\placesubformula[h]{b} $$ a + b = c \quad \formulanumber       $$
\stopbuffer
\ShowMathExample[h]

\startbuffer
\placesubformula      $$ a + b = c \quad \formulanumber   {a} $$
\stopbuffer
\ShowMathExample[]

\startbuffer
\placesubformula       $$ a + b = c \quad \formulanumber   {b} $$
\stopbuffer
\ShowMathExample[]

\startbuffer
\placeformula          $$ a + b = c \quad \formulanumber[i]    $$
\stopbuffer
\ShowMathExample[i]

\startbuffer
\placeformula          $$ a + b = c \quad \formulanumber[j]    $$
\stopbuffer
\ShowMathExample[j]

\startbuffer
\placesubformula       $$ a + b = c \quad \formulanumber[k]{a} $$
\stopbuffer
\ShowMathExample[k]

\startbuffer
\placesubformula       $$ a + b = c \quad \formulanumber[l]{b} $$
\stopbuffer
\ShowMathExample[l]

\startbuffer
\placeformula   [m]    $$ a + b = c \quad \formulanumber   {a} $$
\stopbuffer
\ShowMathExample[m]

\startbuffer
\placeformula   [n]    $$ a + b = c \quad \formulanumber   {b} $$
\stopbuffer
\ShowMathExample[n]

\startbuffer
\placesubformula[o]    $$ a + b = c \quad \formulanumber   {a} $$
\stopbuffer
\ShowMathExample[o]

\startbuffer
\placesubformula[p]    $$ a + b = c \quad \formulanumber   {b} $$
\stopbuffer
\ShowMathExample[p]

\startbuffer
\placeformula   [q]    $$ a + b = c \quad \formulanumber[r]{a} $$
\stopbuffer
\ShowMathExample[q,r]

\startbuffer
\placeformula   [s]    $$ a + b = c \quad \formulanumber[t]{b} $$
\stopbuffer
\ShowMathExample[s,t]

\startbuffer
\placesubformula[u]    $$ a + b = c \quad \formulanumber[v]{a} $$
\stopbuffer
\ShowMathExample[u,v]

\startbuffer
\placesubformula[w]    $$ a + b = c \quad \formulanumber[x]{b} $$
\stopbuffer
\ShowMathExample[w,x]

\stop

When we want {\em no} numbers we have to indicate that
explicitly by means of \type{[-]}:

\startbuffer
\placeformula[-]
  $$\displaylines
      {ab=ba\hfill\cr
       ac+bc=(a+b)c\hfill\cr}$$
\stopbuffer

\startexample
\typebuffer
\stopexample

This results in:

\getbuffer

We also could have used here \type {\startformula...}\type
{\stopformula}:

\startbuffer
\placeformula[-]
  \startformula
  \displaylines{ab=ba\hfill\cr ac+bc=(a+b)c\hfill\cr}
  \stopformula
\stopbuffer

\startexample
\typebuffer
\stopexample

The use of the \type{\start...}\type{\stop}||pair
has the advantage that we can test symmetry in some
wordprocessors. The disadvantage is we can not see
immediately that we work in math mode.

\showsetup{startformula}

The next examples does use numbers. In this example \type
{[that's it]} is a logical name, a label, for future
referencing.

\startbuffer
\placeformula
  \startformula
    \displaylines
        {a\times b=b\times a\hfill\formulanumber\cr
         a+b=b+a\hfill\subformulanumber\cr
         ac+bc=(a+b)c\hfill\formulanumber[that's it]{x}\cr}
  \stopformula
\stopbuffer

\startexample
\typebuffer
\stopexample

This becomes:

\getbuffer

\section{Legends}
\index{formulas+legends}
\macro{\tex{startfact}}
\macro{\tex{startlegend}}

In case of physics formulas you may want to explain the
meaning of the used symbols. There are two commands to do
that:

\showsetup{startlegend}

\showsetup{startfact}

A legend and facts are coded as follows:

\startbuffer
\placeformula[for:force]$$F = m a$$

\startlegend
\leg F \\ force        \\ N       \\
\leg m \\ mass         \\ kg      \\
\leg a \\ acceleration \\ m/{s^2} \\
\stoplegend

Determine by means of formula~\in[for:force] the acceleration~$a$
when given is that:

\startfact
\fact mass  \\ m \\ 10~kg  \\
\fact force \\ F \\ 1500~N \\
\stopfact
\stopbuffer

\startexample
\typebuffer
\stopexample

This results in:

\getbuffer

A combination is also possible:

\startbuffer
\startlegend[two]
\leg F \\      \\ force        \\ N       \\
\leg m \\   10 \\ mass         \\ kg      \\
\leg a \\ 1500 \\ acceleration \\ m/{s^2} \\
\stoplegend
\stopbuffer

\getbuffer

This was specified in this way:

\typebuffer

\section[units]{Units}
\index{overviews+units}
\index{formulas+units}
\index{units}
\macro{\tex{unit}}
\macro{\tex{mathematics}}

A unit can be typeset with:

\startexample
\starttyping
10~$\rm m^3$
\stoptyping
\stopexample

For the purpose of consistent typesetting the command
\type{\unit} is available. This is an example of the use of
synonyms as described in \in{section}[synonyms].

\startexample
\starttyping
\unit {strange} {m^3\!/s^2} {a strange unit}
\stoptyping
\stopexample

In this case the \type{\!} takes care of backskipping
the~$/$ in such a way that in stead of $\rm m^3/s^2$ we get
$\rm m^3\!/s^2$. In fact we can do without these kind of
cryptic typing, because the unit module offers a better
alternative. The module is loaded in the set up area of your
source file with:

\starttyping
\usemodule[unit]
\stoptyping

After that you can type the recall unit by typing them. For
example:

\starttyping
... 10 \Meter \Per \Second\ ...
... 33 \Kilo \Gram \Per \Square \Meter\ ...
\stoptyping

At this point we advise you to read the manual that comes
with this module for more examples.

When we use math commands there may occur problems as soon
as we use~\type{$} in a nested way. When we are in math mode
and we use a~\type{$} for the purpose of switching to math
mode we just end math mode like this:

\startexample
\starttyping
$a $\times$ b$
\stoptyping
\stopexample

\TEX\ will produce an error because \type{\times} is typed
outside math mode. In this example we saw what goes wrong
but the problem is less obvious in the next example:

\startexample
\starttyping
\def\multiply{$\times$}
$a \multiply b$
\stoptyping
\stopexample

This seems correct but with \type{\multiply} we leave math
mode. We can prevent errors by defining \type{\multiply}
as follows:

\startexample
\starttyping
\def\multiply{\ifmmode \times \else $\times$ \fi}
\stoptyping
\stopexample

The next commands does just that:

\showsetup{mathematics}

We can use this command in nested situations:

\startbuffer
\mathematics{a\mathematics{b\mathematics{c\mathematics{d\mathematics{e}}}}}
\stopbuffer

\startexample
\typebuffer
\stopexample

and it will result in a correct output:

\startreality
\getbuffer
\stopreality

so do not use this:

\startbuffer
$a$b$c$d$e$
\stopbuffer

\startreality
\getbuffer
\stopreality

which we would have obtained by typing:

\startexample
\typebuffer
\stopexample

\section{Chemicals}
\index[PPCHTEX]{\PPCHTEX}
\index{chemical formulas}
\macro{ch}
\macro{startchemical}
\macro{chemical}
\macro{usemodule}

Earlier we stated that in this chapter we also describe the
module for chemical typesetting. This module is loaded with:

\starttyping
\usemodule[chemic]
\stoptyping

The first version of this module used \PICTEX\ for
positioning text and drawing the chemical structures, the
current version uses \METAPOST\ for drawing the graphics.
The results are better and the files are more compact.

\startbuffer
\startchemical[with=fit,height=fit]
  \chemical
    [SIX,B,C,ADJ1,
     FIVE,ROT3,SB34,+SB2,-SB5,Z345,DR35,SR4,CRZ35,SUB1,
     ONE,OFF1,SB258,Z0,Z28]
    [C,N,C,O,O,
     CH,COOC_2H_5,COOC_2H_5]
\stopchemical
\stopbuffer

\startlinecorrection
\getbuffer
\stoplinecorrection

This chemical structure was typed as follows:

\typebuffer

The interface (syntax) looks rather cryptic but after some
practice its compactness is an asset. There is an extensive
manual and a collection of examples available.

One characteristic of chemical typesetting is the fact that
all super|| respectively subscripts are at the same height.
This is not the case in math typesetting where the location
of the super|| and subscripts depend on the available
vertical space. The command \type {\chemical} takes this
into account. When you want to put a chemical formula in a
math formula |<|for example when you want to display an
expression for a chemical equilibrium|>| there is the
command \type{\ch}. This command has one argument and adapts
automatically to its context: \type
{$\frac{\ch{N}}{\ch{O}}$}

\section{Math}
\index{fraktur}
\index{gothic}
\index{calligraphy}
\index{fractions}
\macro{\tex{fraction}}
\macro{\tex{frac}}
\macro{\tex{maframed}}
\macro{\tex{inmaframed}}
\macro{\tex{frak}}
\macro{\tex{goth}}
\macro{\tex{cal}}
\macro{\tex{fraktur}}
\macro{\tex{gothic}}
\macro{\tex{calligraphic}}

We limit ourselves only to those commands that are
available by default. In addition to the commands mentioned
here, the math module implements many more:

\starttyping
\usemodule[math]
\stoptyping

The extra commands are described in a separate manual.

Like in  plain \TEX\ we offer the next commands
for switching to some specialized fonts:

\starttabulate[|l|l|l|]
\NC \type {\frak} \NC fraktur      \NC \frak ABC \NC\NR
\NC \type {\goth} \NC gothic       \NC \goth abc \NC\NR
\NC \type {\cal}  \NC calligraphic \NC \cal  ABC \NC\NR
\stoptabulate

Alternatively one can use the commands \type {\fraktur},
\type {\gothic} and \type {\calligraphic} which each take
one argument, like in \type {\fraktur {TEXT}}.

These are typical fonts meant for math typesetting and
special characters.

Fractions can occur quite often so we also added the command
\type {\frac} on request: \type {$\frac{a}{b}$} results as
expected $\frac{a}{b}$. This command adapts to its
surroundings as good as possible.

For instructional purposes a frame or a background can be
useful to indicate the specific math symbol. There is a
special version of \type {\framed}: \type {\maframed}. We
give some examples:

\startbuffer
\startformula
  y + \maframed{y} + y^{2} + y^{\maframed{2}}
\stopformula

\startformula
  x \times \maframed{y} \times y^{\maframed{z}_{\maframed{z}}}
\stopformula
\stopbuffer

\typebuffer

\getbuffer

In this example we can see that the superscript~2 is rather
big. This can be prevented by using the commands \type
{\super} and \type {\suber} in stead of \type {^} and \type
{_}.

\startbuffer
\startformula
  x \times \maframed{y} \times y\super{\maframed{z}\suber{\maframed{z}}}
\stopformula
\stopbuffer

\typebuffer

\getbuffer

If you want to use \type {^} and \type {_} anyhow, than you
can use \type {\enablesupersub} to reach the same effect.

To obtain a good spacing in framed math texts the \type
{offset} equals \type {overlay}. The offset is produced by
giving \type {frameoffset} an adequate value. Other setups
are also possible:

\startbuffer
\startformula
  x \times y\super{\maframed[framecolor=red]{z}\suber{z}}
\stopformula
\stopbuffer

\typebuffer

\getbuffer

For in||line math the command \type {\inmaframed}
is available.

It is possible to typeset fractions without switching to
math mode with the command:

\showsetup{fraction}

The braces are essential in the next example.

\startbuffer
If \fraction{123}{456} equals \fraction{x}{y}, then \fraction{y}{x} equals
\fraction{456}{123}.
\stopbuffer

\startexample
\typebuffer
\stopexample

results in:

\startreality
\getbuffer
\stopreality


\section {Math collection}

Math is a complicated matter and therefore we will not spend
that many words on the gory details. For the user it is
enough to know that you can mix different math fonts in a
comfortable way and that \CONTEXT\ will take care of the
proper mapping on specific math fonts.

Because the wide range of math symbols can come from
different fonts, math characters are organized into so
called math collections. Normally such a collection is
chosen automatically when you load a font definition, just
as with font encodings. The \AMS\ math fonts extend the
default math collection, which gives you a comfortable fall
back. More information can be found in the documentation of
the math module.

You can generate a list of the current math character set
with the command \type {\showmathcharacters}.

\showmathcharacters

\stopcomponent

