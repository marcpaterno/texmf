\documentclass{article}
\usepackage{times,europs}

\setlength{\parindent}{0mm}
\setlength{\parskip}{2\baselineskip}

\newcommand{\eutab}[1]{\begin{tabular}{l|cc}%
       & upright & italic \\%
\hline%
medium & \textup{\textmd{#1}} & \textit{\textmd{#1}} \\%
bold   & \textup{\textbf{#1}} & \textit{\textbf{#1}}%
\end{tabular}}

\begin{document}

\def\sampletext{%
The Euro symbol \eur{} comes in different shapes, depending on the
\textbf{current context \eur}. Font changes \textit{can \eur{} be followed,
\textbf{even if \eur{} they are nested}}. \textsf{Changes of \eur{} the font
family} \texttt{are handled \eur{} differently.}
}

\let\eur\EUR
\sampletext

\textrm{EuroSerif}:\\
\eutab{\EURtm}\\
\let\eur\EURtm
\sampletext

\textsf{EuroSans}:\\
\eutab{\EURhv}\\
\let\eur\EURhv
\sampletext

\texttt{EuroMono}:\\
\eutab{\EURcr}\\
\let\eur\EURcr
\sampletext

\textrm{official}:\\
\eutab{\EURofc}\\
\let\eur\EURofc
\sampletext

\end{document}
