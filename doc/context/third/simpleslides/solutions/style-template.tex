% This file is a template to create your own style for use with the
% simpleslides module. Modify the "dummy"-settings in it, save it under a
% name simpleslides-s-YOURNAME.tex, and use it by choosing style=YOURNAME
% in the \tex{usemodule} command.

% Provide the name of your style here; replace "template" with your unique
% name for your style.

\startmodule[simpleslides-s-template]

\unprotect

% Here, you set the layout for your style. You will probably have to fiddle
% with the values until everything is exactly the way you want.

\setuplayout [width=fit,
	      height=middle,
              margin=0cm,
              height=fit,
	      margindistance=0cm,
              header=0cm, 
              footer=0cm, 
              topspace=1cm,
	      bottomspace=2cm,
              backspace=1.5cm,
              location=singlesided]

% Some presentations need a different layout for "horizontal" and "vertical"
% slides. If, e.g., you want a larger header for you horizontal slides (in
% order to accomodate the titles of your slides), you would set this up like
% so: 

\setuplayout [simpleslides:layout:horizontal][header=1.4cm]
\setuplayout [simpleslides:layout:vertical]  [header=0cm]
\setuplayout [simpleslides:layout:title]     [header=0cm]

% The following command defines the position of the slidetitle layer; the
% values x= and y= determine the distance from the top left edge.

\setuplayer[simpleslides:layer:slidetitle]
            [x=15mm]

% Sometimes, your vertical arrangement is set up in a way that your picture
% frames should not take up the full textheight and less than half of the
% textwidth. You can set these values here; they will be used internally. 

\define\NormalHeight        {\textheight}
\define\NormalWidth         {.5\textwidth}
\define\PictureFrameHeight  {\textheight}
\define\PictureFrameWidth   {.5\textwidth}

% Now comes an important part: defining your color scheme. You will need at
% least two colors: a backgroundcolor and a contrastcolor. The
% simpleslides:itemize:color is used to typeset numbers and symbols of
% itemizations; in many styles, the contrastcolor is used. 

\definecolor [simpleslides:backgroundcolor]    [s=.95]
\definecolor [simpleslides:contrastcolor]      [s=.3]
\definecolor [simpleslides:variantcolor]       [s=.1]
\definecolor [simpleslides:itemize:color]      [simpleslides:contrastcolor]

% Now comes the part which is of greatest importance to the visual appearance
% of your presentation: defining the background of your slides.  One
% possibility is using \METAPOST\ to calculate it, using the colors you have
% just defined. If you want to set up such a background, have a look at chapter
% 6 of the Metafun manual, where many nifty tricks are explained. One thing
% that is especially useful is the StartPage ... StopPage environment
% (explained in chapter 6.4), which gives you access to many variables
% pertaining to page dimensions. Since there are three different types of
% slides (for the title page, for "horizontal" and for "vertical" slides), it
% may make sense to define three different backgrounds, but that is entirely up
% to you.

\startuseMPgraphic{simpleslides:MP:ornament} 
StartPage ;
fill Page withcolor \MPcolor{simpleslides:backgroundcolor} ;
draw Page withcolor \MPcolor{simpleslides:contrastcolor} ;
StopPage ;
\stopuseMPgraphic 

% The actual setup of our backgrounds is done here: we define overlays for
% different types of slides. You can now make use of the MPgraphics that you
% just defined, but you could also put external images as the background to
% your slides. By default, the following backgrounds are used:
%
% title page: simpleslides:background:title
%
% horizontal: simpleslides:background:horizontal and simpleslides:background:ornament
%
% vertical:   simpleslides:background:vertical and simpleslides:background:ornament


\defineoverlay 
  [simpleslides:background:ornament] 
  [\useMPgraphic{simpleslides:MP:ornament}] 

\defineoverlay 
  [simpleslides:background:title] 
  [\useMPgraphic{simpleslides:MP:ornament}] 

% The setupTitle and setupSlideTitle define the look of your titles; lots of
% setup commands are available to determine what they should look like.

\setupTitle
  [\c!headcolor={simpleslides:contrastcolor}]

\setupSlideTitle
  [\c!color={simpleslides:contrastcolor},
   \c!alternative=layer,
   \c!align=\v!center,
   \c!width=\textwidth,
   \c!height=3cm,
   \c!after=]

% You're almost done! At long last, we define a symbol for the first level of
% itemizations and make sure that itemizations use the right color. 

\definesymbol[1][\useMPgraphic{simpleslides:itemize:square}]
\setupitemize[1][color={simpleslides:itemize:color}]

\protect
\stopmodule

% And this should be enough!

\endinput

